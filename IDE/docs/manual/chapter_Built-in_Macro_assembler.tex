\newcommand{\no}{\color{red}{\textbf{no}}}
\newcommand{\yes}{\color{black}{\textbf{yes}}}

In this chapter, we will be concerned with the build-in macro assembler. With syntax of its statements, directives, and PicoBlaze instructions. We assume that the reader is familiar with general concepts of the assembly language programming and with PicoBlaze architecture.

MDS macro-assembler for PicoBlaze is a two stage fast macro-assembler inspired by Intel assemblers and the C language. It supports wide range of output data formats and a number of advanced features like macro processing, conditions, loops, etc. it is meant to give you means to write code with run-time efficiency typical for the assembly language while giving you some of the comfort of a more high-level language like C. For instance you may use ``\verb'if ( S0 == S1 )''' to easy write conditions instead of compares and conditional jumps, ``\verb'for ( S0, 0 .. 9 )''' for loops, ``\verb'#if my_constant > 20''' for conditional compilation, ``\verb'abc macro x, y, z''' for defining your own macros, and more. MDS assembler is enhanced with these features in hope it will help you save your time and make your work a little bit easier. MDS assembler also has a number of features for smooth and transparent debugging and it is regularly subjected to extensive automated testing to ensure its very functionality and to provide high reliability. Also this assembler supports all PicoBlaze versions publicly available at the time of its release. For these reasons we are convinced that this is the most advanced assembler for PicoBlaze currently available on the market.

\paragraph{Supported processors}
    \begin{itemize}
        \item PicoBlaze 6
        \item PicoBlaze 3
        \item PicoBlaze II
        \item PicoBlaze CPLD
        \item PicoBlaze
    \end{itemize}

\paragraph{Note about this documentation:}
    to write a document properly describing any programming language, including the assembly language, is always a challenging task. If you find a mistake or improperly covered area in this documentation, please let us know, we will fix that and provide you with the fixed version as soon as possible.

\clearpage
\section{General}
    \subsection{Main differences from the Xilinx assembler}
        \begin{description}
            \item[Default radix is decimal, not hexadecimal.]~\\
                You can use different radix for each numerical literal, if you do not specify radix, it is decimal by default. For hexadecimal radix use the `0x' prefix: 0x1a, 0xbc, 0x23, ...; for octal radix use the `0' (zero) prefix: 076, 011, 027, ...; for binary radix use the `0b' prefix: 0b11001100, 0b10101010, 0b11111111, ...; for ASCII value put the character in single quotes: 'a', 'A', '3', ... Suffix notation is also supported: 80h (hex.), 128d (dec.), 200q (oct.), 10000000b (bin.), for ASCII characters, you can also use C language escape sequences: \verb"'\0'"~(NUL), \verb"'\n'"~(LF), \verb"'\r"~(CR), \verb"'\t"~(TAB), ...
            \item[Addressing mode specification is mandatory.]~\\
                For immediate addressing use the `\#' prefix: \texttt{LOAD~S0,~\#0xAB}; for indirect addressing use the ``@'' prefix: ``\texttt{STORE~S0,~@S1}''; and for direct addressing do not use any prefix: ``\texttt{LOAD~S0,~S1~+~3}'' (this loads S0 with S4).
            \item[This assembler is case insensitive.]~\\
                ``\texttt{load~S0,~S1}'' is the same as ``\texttt{LOAD~s0,~s1}'', or ``\texttt{Load~S0,~s1}''.
            \item[This assembler supports user defined macro instructions, and expressions.]~\\
                ``\texttt{LOAD~S0,~\#(2~+~3~*~8)}'', etc. please refer to the Tutorial Project, and the Quick User Guide for brief introduction, or to later pages in this manual for detailed description.
        \end{description}

    \clearpage
    \subsection{Statements}
        Source code files for this assembler have to be text files in the following format:
        ~\\\\
        \verb'[ label: ]  [ instruction  [ operand  [ , operand  ] ]    [ ;comment ]'\\
        \verb'[ label: ]  directive      [ argument [ , argument ] ]    [ ;comment ]'\\
        \verb'[ symbol ]  directive      [ argument [ , argument ... ]  [ ;comment ]'\\
        \verb'            directive      symbol, argument               [ ;comment ]'\\

        Tokens in square brackets are optional. Compilation does not continue beyond line containing the \texttt{END} directive, after that directive the code does not have to be syntactically correct. Empty lines are valid, as well as lines containing only comment or label. Statements are separated by white space (spaces or tabs). Statements are case insensitive and their length is not limited, overall line length is also not limited. Allowed line delimiters are: LF, CR, and sequence CRLF. Expected encoding is UTF-8 but it is not necessarily required.

    \subsection{Comments}
        Comments in source code are ignored by the assembler (they are supposed to serve only as notes to the developer). MDS assembler uses two types of comments: single-line comment starting with ``\texttt{;}'' (semicolon) character and ending with the end of line, and multi-line comments starting with ``\texttt{/*}'' sequence (slash followed by asterisk) and ending with ``\texttt{*/}'' sequence (asterisk followed by slash), like in the C language.

        \paragraph{Example}~\\
            \verb'    ; Single-line comment'\\
            \verb'    '\\
            \verb'    /*'\\
            \verb'     * Multi-line comment ...'\\
            \verb'     */'\\
            \verb'    '\\
            \verb'    LOAD /* Comment, */ S0, /* another comment, */ S1 ; more comment.'

    \clearpage
    \subsection{Numbers}
        MDS assembler supports multiple radixes for numeric constants, radix is specified using prefix or suffix notation. Default radix is decimal (like in the C language).

        ~\\
        \verb"    ; Hexadecimal."\\
        \verb"    ld          S0, #0xAB       ; prefix notation (`0x...')"\\
        \verb"    ld          S0, #0ABh       ; suffix notation (`...h')"\\
        \verb""\\
        \verb"    ; Decimal."\\
        \verb"    ld          S0, #123        ; no prefix - default radix - decimal"\\
        \verb"    ld          S0, #123d       ; suffix notation (`d')"\\
        \verb""\\
        \verb"    ; Octal."\\
        \verb"    ld          S0, #076        ; prefix notation (`0...')"\\
        \verb"    ld          S0, #76q        ; suffix notation (`...q')"\\
        \verb"    ld          S0, #76o        ; suffix notation (`...o')"\\
        \verb""\\
        \verb"    ; Binary."\\
        \verb"    ld          S0, #0b1100     ; prefix notation (`0b...')"\\
        \verb"    ld          S0, #1100b      ; suffix notation (`...b')"\\
        \verb""\\
        \verb"    ; ASCII."\\
        \verb"    ld          S0, #'A'        ; ASCII value of capital A."\\
        \verb"    ld          S0, #'\t'       ; ASCII value of the tab character (escape sequence)."

        \subsubsection{Escape sequences}
            \index{Escape sequences} \label{Escape sequences}
            In place of ASCII value you can also use a single escape sequence, escape sequences can also be used in strings (\verb'"abc\r\n"'). Escape sequences in MDS assembler are exactly the same as in the C language.

            \begin{table}[h!]
                \centering
                \begin{tabular}{|c|l|c|}
                    \hline
                    \textbf{Sequence} & \textbf{Description} & \textbf{Value} \\\hline
                    \verb'\a'         & alarm (bell)         & 0x07 \\\hline
                    \verb'\b'         & backspace            & 0x08 \\\hline
                    \verb"\'"         & single quote         & 0x27 \\\hline
                    \verb'\"'         & double quote         & 0x22 \\\hline
                    \verb'\?'         & question mark        & 0x3F \\\hline
                    \verb'\\'         & backslash            & 0x5C \\\hline
                    \verb'\f'         & form feed            & 0x0C \\\hline
                    \verb'\n'         & line feed            & 0x0A \\\hline
                    \verb'\r'         & carriage return      & 0x0D \\\hline
                    \verb'\t'         & horizontal tab       & 0x09 \\\hline
                    \verb'\v'         & vertical tab         & 0x0B \\\hline
                    \verb'\e'         & escape               & 0x1B \\\hline
                \end{tabular}
                \caption{Escape sequences.}
            \end{table}

            In addition to the escape sequences sequences described in the table above, MDS assembler also supports these escape sequences:
            \begin{description}
                \item[$\backslash$NNN]~\\
                    To write a character whose numerical value is given by \texttt{NNN} interpreted as an octal number (e.g. \verb'\0', \verb'\22', \verb'\377', etc.).
                \item[$\backslash$xHH]~\\
                    To write a character whose numerical value is given by \texttt{HH} interpreted as an hexadecimal number (e.g. \verb'\x0F', \verb'\x2A', \verb'\xC7', etc.).
                \item[$\backslash$uXX]~\\
                    To write a unicode character specified by max. 4 hexadecimal digits.
                \item[$\backslash$uXXXX]~\\
                    To write a unicode character specified by max. 8 hexadecimal digits.
            \end{description}

    \subsection{Symbols}
        \index{Symbols}
        Symbols are user defined symbolic names for numbers and addresses used in the program. Symbol names consist of upper and lower case letters, digits, and underscore character (``\_''), their length is not limited, they are case insensitive and they have to be different from language keywords. Symbol names cannot start with digit. Be aware of that there cannot coexists two or more symbols which differs only by letter casing, for instance ``\texttt{abc}'' and ``\texttt{ABC}'' are considered by this assembler to be the same symbol. Symbols have to be defined before they are used.

        \paragraph{Example}~\\
            \verb'    first_symbol      EQU     0b100111 ; Binary radix, number.'\\
            \verb'    second_symbol     SET     047      ; Octal radix, number.'\\
            \verb'    third_symbol      REG     39       ; Decimal radix, register address.'\\
            \verb'    fourth_symbol     DATA    0x27     ; Hexadecimal radix, scratch-pad ram address.'\\
            \verb'    fifth_symbol      CODE    0x27     ; Hexadecimal radix, program memory address.'\\
            \verb'    my_label:                          ; Program memory address, defined using label.'

        \subsubsection{Special Symbols}
            MDS assembler supports only one special symbol, this symbol is always automatically defined on every line containing an instruction (or the \texttt{DB} directive), it is symbol ``\texttt{\$}''. ``\texttt{\$}'' contains address in the program memory where ``this instruction'' is going to be placed by the assembler.

            \paragraph{Example}~\\
                \verb'    ; Since $ conatins the address of the JUMP instruction,'\\
                \verb'    ; this jump results in infinite loop.'\\
                \verb'    JUMP    $'\\
                \verb''\\
                \verb'    ; Skip the next instruction ($+1 would "skip" only the jump itself).'\\
                \verb'    JUMP    $+2'\\
                \verb'    LOAD    S0, S1   ; <-- This instruction is goigh to be skipped.'\\
                \verb'    LOAD    S2, S3'

        \clearpage
        \subsubsection{Predefined Symbols}
            \paragraph{Registers}~\\
                These symbols are defined by default when you specify the target device (using the ``\texttt{DEVICE}'' directive). Symbols \texttt{S10}..\texttt{S1F} are defined for KCPSM2 only, symbols \texttt{S8}..\texttt{S1F} are not defined for KCPSM1-CPLD but defined for every other PicoBlaze. All these symbols are defined as register addresses.

                \begin{table}[h!]
                    \centering
                    \begin{tabular}{|cc|cc|}
                        \hline
                        \textbf{Symbol} & \textbf{Value} & \textbf{Symbol} & \textbf{Value} \\\hline
                        \texttt{S0}     & 0x00           & \texttt{S10}    & 0x10           \\\hline
                        \texttt{S1}     & 0x01           & \texttt{S11}    & 0x11           \\\hline
                        \texttt{S2}     & 0x02           & \texttt{S12}    & 0x12           \\\hline
                        \texttt{S3}     & 0x03           & \texttt{S13}    & 0x13           \\\hline
                        \texttt{S4}     & 0x04           & \texttt{S14}    & 0x14           \\\hline
                        \texttt{S5}     & 0x05           & \texttt{S15}    & 0x15           \\\hline
                        \texttt{S6}     & 0x06           & \texttt{S16}    & 0x16           \\\hline
                        \texttt{S7}     & 0x07           & \texttt{S17}    & 0x17           \\\hline
                        \texttt{S8}     & 0x08           & \texttt{S18}    & 0x18           \\\hline
                        \texttt{S9}     & 0x09           & \texttt{S19}    & 0x19           \\\hline
                        \texttt{SA}     & 0x0A           & \texttt{S1A}    & 0x1A           \\\hline
                        \texttt{SB}     & 0x0B           & \texttt{S1B}    & 0x1B           \\\hline
                        \texttt{SC}     & 0x0C           & \texttt{S1C}    & 0x1C           \\\hline
                        \texttt{SD}     & 0x0D           & \texttt{S1D}    & 0x1D           \\\hline
                        \texttt{SE}     & 0x0E           & \texttt{S1E}    & 0x1E           \\\hline
                        \texttt{SF}     & 0x0F           & \texttt{S1F}    & 0x1F           \\\hline
                    \end{tabular}
                    \caption{Predefined symbols for registers}
                \end{table}

            \enlargethispage{2\baselineskip}
            \paragraph{Other predefined symbols}
                \begin{description}
                    \item[\_\_MDS\_VERSION\_\_]~\\
                        This constant contains current version of your MDS installation. It is a 16 bit BCD coded integer containing 3 numbers: VMMP where V stands for major version number, MM stands for minor version number, and P stands for patch number. For example, if your MDS version was for instance 1.2.3, value of this constant would be 0x1023; or if your MDS version was 3.2.1, the value would be 0x3021.
                    \item[\_\_DATE\_\_]~\\
                        This is a string\footnote{Please see the \texttt{STRING} directive (\ref{STRING directive}) for details on what the strings are and what they can be used for.} containing the current date formatted as "Jan 19 2015, it contains 11 characters; and if the day of the month is less than 10, it is padded with a space on the left. Its the same as in the ISO C language.
                    \item[\_\_TIME\_\_]~\\
                        This is a string containing the current time formatted as "13:18:06, it contains 8 characters. Its the same as in the ISO C language.
                    \item[\_\_FILE\_\_]~\\
                        A string containing the name of the current input file including its full path.
                    \item[\_\_LINE\_\_]~\\
                        This string contains line number in the current input file.
                \end{description}

    \subsection{Expressions}
        \index{Expressions}
        Mathematical expressions are evaluated at compilation time and replaced with constants corresponding to their resulting values. Expression calculation is performed on 32-bit unsigned integers, the resulting values then are trimmed from left to 16-bits at most (with exception for the \texttt{DB} directive where values might be trimmed to 18 bits). Expression comprises of arithmetical operators, numeric literals, and symbols. Examples of such expressions include:

        \begin{itemize}
            \item \texttt{2+1}
            \item \texttt{(2 + 4) - ABC}
            \item \texttt{A \& B}
            \item \texttt{X / 0FF00h}
            \item \texttt{X * Y + X \% Y}
        \end{itemize}

        When operators with different priority levels appear in the expression, operations are evaluated according to priorities. When operators of the same priority appear in the expression, operations are evaluated from left to right. Parenthesis may be used to force a different order of evaluation, for example \texttt{2 + 2 * 2} is evaluated as 6 because multiplication has higher priority than addition, but \texttt{( 2 + 2 ) * 2} results in 8 because addition is enclosed by parenthesis and therefore it is evaluated prior to the multiplication.

        The table below shows priorities for all supported operators, priority 1 is the highest priority, priority 9 is the lowest priority. Arithmetic operators have higher priority than relational operators, and relational operators have higher priority than logical operators.

        \enlargethispage{6\baselineskip}
        \begin{table}[h!]
            \centering
            \begin{tabular}{|c|c|l|l|}
                \hline
                \textbf{Priority} & \textbf{Operator} & \textbf{Description} & \textbf{Example} \\\hline
                1        & \texttt{+}    & unary plus sign         & \texttt{+12}               \\\hline
                1        & \texttt{-}    & unary minus sign        & \texttt{-5}                \\\hline
                2        & \texttt{\~{}} & bitwise NOT             & \texttt{\~{}0a55ah}        \\\hline
                2        & \texttt{!}    & logical NOT             & \texttt{!0a55ah}           \\\hline
                3        & \texttt{*}    & unsigned multiplication & \texttt{11 * 12}           \\\hline
                3        & \texttt{/}    & unsigned division       & \texttt{11 / 12}           \\\hline
                3        & \texttt{\%}   & unsigned modulo         & \texttt{13 \% 11}          \\\hline
                4        & \texttt{+}    & unsigned addition       & \texttt{3 + 5}             \\\hline
                4        & \texttt{-}    & unsigned subtraction    & \texttt{20 - 4}            \\\hline
                5        & \texttt{<{}<} & binary shift left       & \texttt{21 <{}< 4}         \\\hline
                5        & \texttt{>{}>} & binary shift right      & \texttt{32 >{}> 2}         \\\hline
                6        & \texttt{<}    & less than               & \texttt{11 < 12}           \\\hline
                6        & \texttt{<=}   & less or equal than      & \texttt{11 <= 11}          \\\hline
                6        & \texttt{>}    & greater than            & \texttt{12 > 11}           \\\hline
                6        & \texttt{>=}   & greater or equal than   & \texttt{12 >= 11}          \\\hline
                7        & \texttt{==}   & equal to                & \texttt{11 == 11}          \\\hline
                7        & \texttt{!=}   & not equal to            & \texttt{A != B}            \\\hline
                7        & \texttt{<>}   & not equal to            & \texttt{A <> B}            \\\hline
                8        & \texttt{\&}   & bitwise AND             & \texttt{48 \& 16}          \\\hline
                8        & \texttt{|}    & bitwise OR              & \texttt{370q | 7}          \\\hline
                9        & \texttt{\&\&} & logical AND             & \texttt{48 \&\& 16}        \\\hline
                9        & \texttt{||}   & logical OR              & \texttt{370q || 7}         \\\hline
                9        & \texttt{\^{}} & bitwise XOR             & \texttt{00fh \^{} 005h}    \\\hline
            \end{tabular}
            \caption{Operators priorities.}
        \end{table}

        \subsubsection{Special operators}
            Special operators can appear only at certain places and serve special purposes.
            \begin{description}
                \item[high(...)]~\\
                    Operator \texttt{HIGH(...)} extracts the high order byte from 16 bit value. For example \texttt{LOAD  S0, \#high(0x2233)} would load S0 with immediate value of 0x22.
                \item[low(...)]~\\
                    Operator \texttt{LOW(...)} extracts the low order byte from 16 bit value. For example \texttt{LOAD  S0, \#low(0x2233)} would load S0 with immediate value of 0x33.
                \item[at]~\\
                    Operator \texttt{AT} is used only in conjunction with \texttt{AUTOREG} and \texttt{AUTOSPR} directives to set address counter to the specified value.
                \item[..]~\\
                    Operator \texttt{..} is used only with the \texttt{FOR} directive to specify the loop iteration interval.
            \end{description}

    \clearpage
    \subsection{Reserved keywords}
        Please remember that the assembler is case \textbf{in-}sensitive.
        \subsubsection{Instruction mnemonics}
            \begin{table}[h!]
                \centering
                \texttt{
                    \begin{tabular}{|l|l|l|l|l|l|}
                        \hline
                        cpl2 & cpl  & inc  & dec  & setr & clrr \\\hline
                        setb & clrb & notb & djnz & ijnz & nop  \\\hline
                    \end{tabular}
                }
                \caption{Pseudo-instructions}
            \end{table}

            \begin{table}[h!]
                \centering
                \texttt{
                    \begin{tabular}{|l|l|l|l|l|l|l|}
                        \hline
                        ldret & ena  & dis & retie & retid & cmp & in  \\\hline
                        out   & outk & ld  & cmpcy & st    & ft  & ret \\\hline
                    \end{tabular}
                }
                \caption{Instruction shortcuts}
            \end{table}

            \begin{table}[h!]
                \centering
                \texttt{
                    \begin{tabular}{|l|l|l|l|l|l|l|}
                        \hline
                        add     & addcy & sub    & subcy   & compare & load      & return \\\hline
                        and     & or    & xor    & test    & store   & fetch     & jump   \\\hline
                        sr0     & sr1   & srx    & sra     & rr      & sl0       & call   \\\hline
                        sl1     & slx   & sla    & rl      & input   & output    &        \\\hline
                        hwbuild & star  & testcy & outputk & jump    & comparecy &        \\\hline
                    \end{tabular}
                }
                \caption{Regular instructions}
            \end{table}

        \subsubsection{Directives}
            Regular directives and special macros directives can be prefixed with ``\texttt{.}'' (period) character without any effect on their meaning and function.

            \begin{table}[h!]
                \centering
                \texttt{
                    \begin{tabular}{|l|l|l|l|l|}
                        \hline
                        if       & elseif & else & endif & while  \\\hline
                        endwhile & endw   & for  & endf  & endfor \\\hline
                    \end{tabular}
                }
                \caption{Special macros}
            \end{table}

            \begin{table}[h!]
                \centering
                \texttt{
                    \begin{tabular}{|l|l|l|l|l|l|}
                        \hline
                        \#if         & \#ifn   & \#ifdef  & \#ifndef  & \#elseifb   & \#endwhile \\\hline
                        \#elseifnb   & \#else  & \#elseif & \#elseifn & \#elseifdef & \#endw     \\\hline
                        \#elseifndef & \#endif & \#ifnb   & \#ifb     & \#while     &            \\\hline
                    \end{tabular}
                }
                \caption{Conditional compilation \& the \texttt{\#WHILE}}
            \end{table}

            \enlargethispage{6\baselineskip}

            \begin{table}[h!]
                \centering
                \texttt{
                    \begin{tabular}{|l|l|l|l|l|l|}
                        \hline
                        failjmp & device   & limit    & reg       & string   & default\_jump \\\hline
                        namereg & address  & org      & define    & undefine & undef         \\\hline
                        equ     & constant & set      & variable  & port     & portin        \\\hline
                        portout & data     & code     & string    & db       & error         \\\hline
                        warning & list     & message  & messg     & nolist   & skip          \\\hline
                        title   & expand   & noexpand & local     & endmacro & endm          \\\hline
                        exitm   & repeat   & rept     & endrepeat & endr     & autoreg       \\\hline
                        autospr & orgspr   & initspr  & mergespr  & macro    & end           \\\hline
                    \end{tabular}
                }
                \caption{Regular directives}
            \end{table}

\clearpage
\section{Instructions}
    \newcommand{\no}{\color{red}{\textbf{no}}}
\newcommand{\yes}{\color{black}{\textbf{yes}}}

\definecolor{instruction_bg}{rgb}{0.7, 0.7, 1.0}
\newcommand{\instruction}[1]{~\\[7pt]\addcontentsline{toc}{subsection}{#1}\colorbox{instruction_bg}{\parbox{\dimexpr\textwidth-2\fboxsep}{\color{black}\textbf{#1}}}\bigskip}

\paragraph{Legend}
    \begin{description}
        \item[sX, sY]~\\
            Direct address in register file, e.g. \texttt{S0}, \texttt{S1}, \texttt{S2}, ...
        \item[@sX, @sY]~\\
            Indirect address, e.g. \texttt{@S0}, \texttt{@S1}, \texttt{@S2}, ...
        \item[\#kk]~\\
            Immediate value, e.g. \texttt{\#0x21} (hex.), \texttt{\#26} (dec.), \texttt{\#'A'} (ascii).
        \item[aaa]~\\
            Address in program memory.
        \item[pp]~\\
            8-bit port address, i.e. in the range from 0x00 to 0xFF.
        \item[p]~\\
            4-bit port address, i.e. in the range from 0x0 to 0xF.
        \item[ss]~\\
            Address in scratch-pad RAM.
    \end{description}

\paragraph{Addressing modes}
    \index{Addressing modes}
    \begin{description}
        \item[Immediate]~\\
            Instead of reading operand value from memory, the operand value is taken from the instruction opcode itself.
        \item[Direct]~\\
            Effective address of the operand is address as given in the instruction opcode.
        \item[Indirect]~\\
            Effective address of the operand is taken from contents of register which address is given in the instruction opcode
    \end{description}

\clearpage
\subsection{Register Loading}
    \instruction{LOAD, LD}
        The \texttt{LOAD} instruction provides a method for specifying the contents of any register. The new value can be a taken from an immediate constant, or contents of any other register. \texttt{LD} is only shorthand for \texttt{LOAD}, these two mnemonics can be used interchangeably.

        \paragraph{Syntax}
            ~\\
            \verb'    LOAD sX, #kk '; Load register sX with immediate value kk.\\
            \verb'    LD   sX, #kk '; The same a above.\\
            \verb''\\
            \verb'    LOAD sX, sY  '; Load register sX with content of register sY.\\
            \verb'    LD   sX, sY  '; The same a above.

        \paragraph{Examples}
            ~\\
            \verb'    LD      S0, #0x55   ; Load register S0 with 0x55.'\\
            \verb'    LD      S1, S0      ; Copy content of S0 to S1.'

        \paragraph{Flags}
            ~\\\indent
            \begin{tabular}{ll}
                Z & no effect \\
                C & no effect
            \end{tabular}

        \paragraph{Availability}
            ~\\\indent
            \begin{tabular}{ccccc}
                PicoBlaze 6 & PicoBlaze 3 & PicoBlaze II & PicoBlaze & PicoBlaze CPLD \\
                \yes        & \yes        & \yes         & \yes      & \yes
            \end{tabular}

    \instruction{STAR}
        Instruction \texttt{STAR} performs the same operation as the \texttt{LOAD} instruction except for that the target register is in inactive register bank while source is in active bank.

        \paragraph{Syntax}
            ~\\
            \verb'    STAR sX, sY   '; Load sX in inactive bank with content of sY in active bank.\\
            \verb'    STAR sX, #kk  '; Load sX in inactive bank with immediate value kk.

        \paragraph{Examples}
            ~\\
            \verb'    STAR    S0, #0x55   ; Load register S0 in inactive bank with 0x55.'\\
            \verb'    STAR    S1, S0      ; Copy S0 in active bank to S1 in inactive bank.'

        \paragraph{Flags}
            ~\\\indent
            \begin{tabular}{ll}
                Z & no effect \\
                C & no effect
            \end{tabular}

        \paragraph{Availability}
            ~\\\indent
            \begin{tabular}{ccccc}
                PicoBlaze 6 & PicoBlaze 3 & PicoBlaze II & PicoBlaze & PicoBlaze CPLD \\
                \yes        & \no         & \no          & \no       & \no
            \end{tabular}

\clearpage
\subsection{Logical}
    \instruction{OR}
        The \texttt{OR} instruction performs bit-wise logical inclusive-OR operation.

        \paragraph{Syntax}
            ~\\
            \verb'    OR sX, sY     '; Perform OR on sX and sY registers, and store the result in sX.\\
            \verb'    OR sX, #kk    '; Perform OR on sX register and immediate value kk, put result in sX.

        \paragraph{Examples}
            ~\\
            \verb'    OR      S0, #0x55   ; S0 = S0 | 0x55.'\\
            \verb'    OR      S1, S0      ; S1 = S1 | S0.'

        \paragraph{Flags}
            ~\\\indent
            \begin{tabular}{ll}
                Z & 1 if result is zero, 0 otherwise \\
                C & 0
            \end{tabular}

        \paragraph{Availability}
            ~\\\indent
            \begin{tabular}{ccccc}
                PicoBlaze 6 & PicoBlaze 3 & PicoBlaze II & PicoBlaze & PicoBlaze CPLD \\
                \yes        & \yes        & \yes         & \yes      & \yes
            \end{tabular}

    \instruction{XOR}
        The \texttt{XOR} instruction performs bit\-wise logical exclusive-OR operation.

        \paragraph{Syntax}
            ~\\
            \verb'    XOR sX, #kk   '; Perform XOR on sX and sY registers, and store the result in sX.\\
            \verb'    XOR sX, sY    '; Perform XOR on sX register and immediate value kk, put result in sX.

        \paragraph{Examples}
            ~\\
            \verb'    XOR     S0, #0x55   ; S0 = S0 ^ 0x55.'\\
            \verb'    XOR     S1, S0      ; S1 = S1 ^ S0.'

        \paragraph{Flags}
            ~\\\indent
            \begin{tabular}{ll}
                Z & 1 if result is zero, 0 otherwise \\
                C & 0
            \end{tabular}

        \paragraph{Availability}
            ~\\\indent
            \begin{tabular}{ccccc}
                PicoBlaze 6 & PicoBlaze 3 & PicoBlaze II & PicoBlaze & PicoBlaze CPLD \\
                \yes        & \yes        & \yes         & \yes      & \yes
            \end{tabular}

\clearpage
    \instruction{AND}
        The \texttt{AND} instruction performs bit\-wise logical AND operation.

        \paragraph{Syntax}
            ~\\
            \verb'    AND sX, #kk   '; Perform AND on sX and sY registers, and store the result in sX.\\
            \verb'    AND sX, sY    '; Perform AND on sX register and immediate value kk, put result in sX.

        \paragraph{Examples}
            ~\\
            \verb'    AND     S0, #0x55   ; S0 = S0 & 0x55.'\\
            \verb'    AND     S1, S0      ; S1 = S1 & S0.'

        \paragraph{Flags}
            ~\\\indent
            \begin{tabular}{ll}
                Z & 1 if result is zero, 0 otherwise \\
                C & 0
            \end{tabular}

        \paragraph{Availability}
            ~\\\indent
            \begin{tabular}{ccccc}
                PicoBlaze 6 & PicoBlaze 3 & PicoBlaze II & PicoBlaze & PicoBlaze CPLD \\
                \yes        & \yes        & \yes         & \yes      & \yes
            \end{tabular}

\clearpage
\subsection{Arithmetic}
    \instruction{ADD, ADDCY}
        The \texttt{ADD} instruction performs an 8-bit addition of two values.

        \paragraph{Syntax}
            ~\\
            \verb'    ADD sX, #kk    '; Add immediate value \#kk to sX register (without carry).\\
            \verb'    ADD sX, sY     '; Add content of sY register to sX register (without carry).\\
            \verb'    ADDCY sX, #kk  '; Add immediate value \#kk to sX register (with carry).\\
            \verb'    ADDCY sX, sY   '; Add content of sY register to sX register (with carry).

        \paragraph{Examples}
            ~\\
            \verb'    LOAD    S0, #1      ; S0 = 1'\\
            \verb'    LOAD    S1, #2      ; S0 = 2'\\
            \verb''\\
            \verb'    ADD     S0, S1      ; S0 = S0 + S1'\\
            \verb'    ADD     S0, #5      ; S0 = S0 + 5'\\
            \verb'    ADDCY   S0, S1      ; S0 = S0 + S1 + Carry'\\
            \verb'    ADDCY   S0, #5      ; S0 = S0 + 5 + Carry'

        \paragraph{Flags}
            \footnote{\texttt{ADDCY} on PicoBlaze 6 behaves differently: Zero flag is set to 1 if result is zero and zero flag was set prior to the \texttt{ADDCY} instruction, otherwise it set to 0.}
            ~\\\indent
            \begin{tabular}{ll}
                Z & 1 if result is zero, 0 otherwise \\
                C & 1 if results in an overflow, 0 otherwise
            \end{tabular}

        \paragraph{Availability}
            ~\\\indent
            \begin{tabular}{ccccc}
                PicoBlaze 6 & PicoBlaze 3 & PicoBlaze II & PicoBlaze & PicoBlaze CPLD \\
                \yes        & \yes        & \yes         & \yes      & \yes
            \end{tabular}

\clearpage
    \instruction{SUB, SUBCY}
        The \texttt{SUB} instruction performs an 8-bit subtraction of two values without carry, and \texttt{SUBCY} instruction does the same but with carry.

        \paragraph{Syntax}
            ~\\
            \verb'    SUB sX, #kk       '; Subtract immediate value \#kk from sX register (without carry).\\
            \verb'    SUB sX, sY        '; Subtract content of sY register from sX register (without carry).\\
            \verb'    SUBCY sX, #kk     '; Subtract immediate value \#kk from sX register (with carry).\\
            \verb'    SUBCY sX, sY      '; Subtract content of sY register from sX register (with carry).

        \paragraph{Examples}
            ~\\
            \verb'    LOAD    S0, #10     ; S0 = 1'\\
            \verb'    LOAD    S1, #2      ; S0 = 2'\\
            \verb''\\
            \verb'    SUB     S0, S1      ; S0 = S0 - S1'\\
            \verb'    SUB     S0, #5      ; S0 = S0 - 5'\\
            \verb'    SUBCY   S0, S1      ; S0 = S0 - S1 - Carry'\\
            \verb'    SUBCY   S0, #5      ; S0 = S0 - 5 - Carry'

        \paragraph{Flags}
            \footnote{\texttt{SUBCY} on PicoBlaze 6 behaves differently: Zero flag is set to 1 if result is zero and zero flag was set prior to the \texttt{SUBCY} instruction, otherwise it set to 0.}
            ~\\\indent
            \begin{tabular}{ll}
                Z & 1 if result is zero, 0 otherwise \\
                C & 1 if results in an overflow (i.e. the result is negative), 0 otherwise
            \end{tabular}

        \paragraph{Availability}
            ~\\\indent
            \begin{tabular}{ccccc}
                PicoBlaze 6 & PicoBlaze 3 & PicoBlaze II & PicoBlaze & PicoBlaze CPLD \\
                \yes        & \yes        & \yes         & \yes      & \yes
            \end{tabular}

\clearpage
\subsection{Test and Compare}
    \instruction{TEST}
        The \texttt{TEST} instruction performs bit\-wise logical AND operation, and discards its result except for the status flags.

        \paragraph{Syntax}
            ~\\
            \verb'    TEST sX, sY'\\
            \verb'    TEST sX, #kk'

        \paragraph{Examples}
            ~\\
            \verb'    TEST    S0, #0       ; Set Z flag, if S0 == 0.'\\
            \verb'    JUMP    Z, somewhere ; Jump to label "somewhere", if Z flag is set.'

        \paragraph{Flags}
            ~\\\indent
            \begin{tabular}{ll}
                Z & 1 if result is zero, 0 otherwise \\
                C & 1 if the temporary result contains an odd number of 1 bits, 0 otherwise
            \end{tabular}

        \paragraph{Availability}
            ~\\\indent
            \begin{tabular}{ccccc}
                PicoBlaze 6 & PicoBlaze 3 & PicoBlaze II & PicoBlaze & PicoBlaze CPLD \\
                \yes        & \yes        & \no          & \no       & \no
            \end{tabular}

    \instruction{TESTCY}
        The \texttt{TESTCY} instruction performs bit\-wise logical AND operation, and discards its result except for the status flags.

        \paragraph{Syntax}
            ~\\
            \verb'    TESTCY sX, sY'\\
            \verb'    TESTCY sX, #kk'

        \paragraph{Examples}
            ~\\
            \verb'    TESTCY  S0, #0       ; Set Z flag, if S0 == 0.'\\
            \verb'    JUMP    Z, somewhere ; Jump to label "somewhere", if Z flag is set.'

        \paragraph{Flags}
            ~\\\indent
            \begin{tabular}{ll}
                Z & 1 if result is zero and zero flag was set prior to the instruction, otherwise it set to 0 \\
                C & 1 if the temporary result together with the previous state of the carry flag contains an \\
                  & odd number of 1 bits, and 0 otherwise
            \end{tabular}

        \paragraph{Availability}
            ~\\\indent
            \begin{tabular}{ccccc}
                PicoBlaze 6 & PicoBlaze 3 & PicoBlaze II & PicoBlaze & PicoBlaze CPLD \\
                \yes        & \no         & \no          & \no       & \no
            \end{tabular}

\clearpage
    \instruction{COMPARE, CMP}
        The \texttt{COMPARE} instruction performs an 8-bit subtraction of two values. Unlike the \texttt{SUB} instruction, result of this operation is discarded, and only status flags are affected. \texttt{CMP} is only shorthand for \texttt{COMPARE}, these two mnemonics can be used interchangeably.

        \paragraph{Syntax}
            ~\\
            \verb'    COMPARE  sX, #kk   '; Compare immediate value \#kk to content of register sX.\\
            \verb'    COMPARE  sX, sY    '; Compare content of register sY to content of register sX.\\\
            \verb'    CMP      sX, #kk   '; Same as ``COMPARE sX, \#kk''.\\
            \verb'    CMP      sX, sY    '; Same as ``COMPARE sX, sY''.

        \paragraph{Examples}
            ~\\
            \verb'    COMPARE S0, #1       ; Set Z flag, if S0 == 1.'\\
            \verb'    JUMP    Z, somewhere ; Jump to label "somewhere", if Z flag is set.'

        \paragraph{Flags}
            ~\\\indent
            \begin{tabular}{ll}
                Z & 1 if both values are equal (sX = kk or sX = sY), 0 if 1st > 2nd (sX > kk or sX > sY) \\
                C & 1 if 1st < 2nd (sX < kk or sX < sY), 0 if 1st > 2nd (sX > kk or sX > sY)
            \end{tabular}

        \paragraph{Availability}
            ~\\\indent
            \begin{tabular}{ccccc}
                PicoBlaze 6 & PicoBlaze 3 & PicoBlaze II & PicoBlaze & PicoBlaze CPLD \\
                \yes        & \yes        & \no          & \no       & \no
            \end{tabular}

    \instruction{COMPARECY, CMPCY}
        The \texttt{COMPARECY} instruction is \texttt{COMPARE} with carry. \texttt{CMPCY} is only shorthand for \texttt{COMPARECY}, these two mnemonics can be used interchangeably.

        \paragraph{Syntax}
            ~\\
            \verb'    COMPARECY  sX, #kk   '; Compare immediate value \#kk to content of register sX.\\
            \verb'    COMPARECY  sX, sY    '; Compare content of register sY to content of register sX.\\\
            \verb'    CMPCY      sX, #kk   '; Same as ``COMPARECY sX, \#kk''.\\
            \verb'    CMPCY      sX, sY    '; Same as ``COMPARECY sX, sY''.

        \paragraph{Examples}
            ~\\
            \verb'    COMPARECY S0, #1       ; Set Z flag, if S0 == 1.'\\
            \verb'    JUMP      Z, somewhere ; Jump to label "somewhere", if Z flag is set.'

        \paragraph{Flags}
            ~\\\indent
            \begin{tabular}{ll}
                Z & 1 if result is zero and zero flag was set prior to the \texttt{COMPARECY} instruction, 0 otherwise \\
                C & 1 if results in an overflow (i.e. the result is negative), 0 otherwise
            \end{tabular}

        \paragraph{Availability}
            ~\\\indent
            \begin{tabular}{ccccc}
                PicoBlaze 6 & PicoBlaze 3 & PicoBlaze II & PicoBlaze & PicoBlaze CPLD \\
                \yes        & \no         & \no          & \no       & \no
            \end{tabular}

\clearpage
\subsection{Shift and Rotate}
    \instruction{SL0, SL1, SLX, SLA}
        Instructions \texttt{SL0}, \texttt{SL1}, \texttt{SLX}, and \texttt{SLA} all shift contents of the specified register by one bit to the left. The new lsb (least significant bit) depends on the instruction, and the msb (most significant bit) is shifted out to the C flag.

        \paragraph{Syntax}
            ~\\
            \verb'    SL0 sX    '; Shift 0 into the lsb .\\
            \verb'    SL1 sX    '; Shift 1 into the lsb.\\
            \verb'    SLX sX    '; Shift previous lsb back into the new lsb.\\
            \verb'    SLA sX    '; Shift C into the lsb;

        \paragraph{Examples}
            ~\\
            \verb'    LOAD    S0, #0x01 ; S0 = 0b00000001'\\
            \verb'    SL0     S0        ; S0 = 0b00000010'\\
            \verb'    SL1     S0        ; S0 = 0b00000101'\\
            \verb'    SLX     S0        ; S0 = 0b00001011'\\
            \verb'    SLA     S0        ; S0 = 0b0001011C (C is either 0 or 1)'

        \paragraph{Flags}
            ~\\\indent
            \begin{tabular}{ll}
                Z & 1 if result is zero, otherwise  0 \\
                C & the msb (most significant bit) of the original content of the sX register.
            \end{tabular}

        \paragraph{Availability}
            ~\\\indent
            \begin{tabular}{ccccc}
                PicoBlaze 6 & PicoBlaze 3 & PicoBlaze II & PicoBlaze & PicoBlaze CPLD \\
                \yes        & \yes        & \yes         & \yes      & \yes
            \end{tabular}

\clearpage
    \instruction{SR0, SR1, SRX, SRA}
        Instructions \texttt{SR0}, \texttt{SR1}, \texttt{SRX}, and \texttt{SRA} all shift contents of the specified register by one bit to the right. The new msb (most significant bit) depends on the instruction, and the lsb (least significant bit) is shifted out to the C flag.

        \paragraph{Syntax}
            ~\\
            \verb'    SR0 sX    '; Shift 0 into the msb .\\
            \verb'    SR1 sX    '; Shift 1 into the msb.\\
            \verb'    SRX sX    '; Shift previous msb back into the new msb.\\
            \verb'    SRA sX    '; Shift C into the msb;

        \paragraph{Examples}
            ~\\
            \verb'    LOAD    S0, #0x80 ; S0 = 0b10000000'\\
            \verb'    SR0     S0        ; S0 = 0b01000000'\\
            \verb'    SR1     S0        ; S0 = 0b10100000'\\
            \verb'    SRX     S0        ; S0 = 0b11010000'\\
            \verb'    SRA     S0        ; S0 = 0bC1101000 (C is either 0 or 1)'

        \paragraph{Flags}
            ~\\\indent
            \begin{tabular}{ll}
                Z & 1 if result is zero, otherwise  0 \\
                C & the lsb (least significant bit) of the original content of the sX register.
            \end{tabular}

        \paragraph{Availability}
            ~\\\indent
            \begin{tabular}{ccccc}
                PicoBlaze 6 & PicoBlaze 3 & PicoBlaze II & PicoBlaze & PicoBlaze CPLD \\
                \yes        & \yes        & \yes         & \yes      & \yes
            \end{tabular}

\clearpage
    \instruction{RR, RL}
        These instructions rotate the specified register by one bit to the left or right. The bit shifted out of the register and then shifted back to the other side.

        \paragraph{Syntax}
            ~\\
            \verb'    RR sX     '; Rotate Right.\\
            \verb'    RL sX     '; Rotate Left.

        \paragraph{Examples}
            ~\\
            \verb'    LOAD    S0, #0x18 ; S0 = 0b00011000'\\
            \verb'    RR      S0        ; S0 = 0b00001100'\\
            \verb'    RL      S0        ; S0 = 0b00011000'\\
            \verb'    RL      S0        ; S0 = 0b00110000'

        \paragraph{Flags}
            ~\\\indent
            \begin{tabular}{ll}
                Z & 1 if result is zero, otherwise 0 \\
                C & The bit shifted out of the register and then shifted back to the other side.
            \end{tabular}

        \paragraph{Availability}
            ~\\\indent
            \begin{tabular}{ccccc}
                PicoBlaze 6 & PicoBlaze 3 & PicoBlaze II & PicoBlaze & PicoBlaze CPLD \\
                \yes        & \yes        & \yes         & \yes      & \yes
            \end{tabular}

\clearpage
\subsection{Register Bank Selection}
    \instruction{REGBANK, RB}
        Set active register bank. \texttt{RB} is only shorthand for \texttt{REGBANK}, these two mnemonics can be used interchangeably.
        \paragraph{Syntax}
            ~\\
            \verb'    REGBANK A '; Set active bank A.\\
            \verb'    REGBANK B '; Set active bank B.\\
            \verb'    RB A      '; Same as REGBANK A.\\
            \verb'    RB B      '; Same as REGBANK B.\\

        \paragraph{Examples}
            ~\\
            \verb'    RB      A'\\
            \verb'    LD      S0, #0xAA'\\
            \verb''\\
            \verb'    RB      B'\\
            \verb'    LD      S0, #0x55'

        \paragraph{Flags}
            ~\\\indent
            \begin{tabular}{ll}
                Z & no effect \\
                C & no effect
            \end{tabular}

        \paragraph{Availability}
            ~\\\indent
            \begin{tabular}{ccccc}
                PicoBlaze 6 & PicoBlaze 3 & PicoBlaze II & PicoBlaze & PicoBlaze CPLD \\
                \yes        & \no         & \no          & \no       & \no
            \end{tabular}

\clearpage
\subsection{Input/Output}
    \instruction{INPUT, IN}
        The \texttt{INPUT} instruction reads data from a port (i.e. from your hardware design) into the specified register. \texttt{IN} is only shorthand for \texttt{INPUT}, these two mnemonics can be used interchangeably. Please notice the difference between direct port addressing and indirect addressing.

        \paragraph{Syntax}
            ~\\
            \verb'    INPUT sX, pp   '; Copy from port at address pp to register at sX address.\\
            \verb'    IN    sX, pp   '; Same as above.\\
            \verb'    INPUT sX, @sY  '; Copy from port at address given by sY register to register sX.\\
            \verb'    IN    sX, @sY  '; Same as above.

        \paragraph{Examples}
            ~\\
            \verb'    IN      S0, 0x22  ; Read port at address 0x22 and copy its value to S0 reg.'\\
            \verb'    LD      S1, #0x11 ; S1 = 0x11'\\
            \verb''\\
            \verb'    ; Read port at address given by S1 reg. (0x11) and copy its value to S0 reg.'\\
            \verb'    IN      S0, @S1'\\

        \paragraph{Flags}
            ~\\\indent
            \begin{tabular}{ll}
                Z & no effect \\
                C & no effect
            \end{tabular}

        \paragraph{Availability}
            ~\\\indent
            \begin{tabular}{ccccc}
                PicoBlaze 6 & PicoBlaze 3 & PicoBlaze II & PicoBlaze & PicoBlaze CPLD \\
                \yes        & \yes        & \yes         & \yes      & \yes
            \end{tabular}

\clearpage
    \instruction{OUTPUT, OUT}
        The \texttt{OUTPUT} instruction transfers data to a port (i.e. to your hardware design) from the specified register. \texttt{OUT} is only shorthand for \texttt{OUTPUT}, these two mnemonics can be used interchangeably. Please notice the difference between direct port addressing and indirect addressing.

        \paragraph{Syntax}
            ~\\
            \verb'    OUTPUT sX, pp   '; Copy from register at sX address to port at address pp.\\
            \verb'    OUT    sX, pp   '; Same as above.\\
            \verb'    OUTPUT sX, @sY  '; Copy from register sX to port at address given by sY register.\\
            \verb'    OUT    sX, @sY  '; Same as above.

        \paragraph{Examples}
            ~\\
            \verb'    OUT     S0, 0x22  ; Write content of S0 reg. to port at address 0x22.'\\
            \verb'    LD      S1, #0x11 ; S1 = 0x11'\\
            \verb''\\
            \verb'    ; Write content of S0 reg. to port at address given by S1 reg. (0x11). '\\
            \verb'    OUT     S0, @S1'\\

        \paragraph{Flags}
            ~\\\indent
            \begin{tabular}{ll}
                Z & no effect \\
                C & no effect
            \end{tabular}

        \paragraph{Availability}
            ~\\\indent
            \begin{tabular}{ccccc}
                PicoBlaze 6 & PicoBlaze 3 & PicoBlaze II & PicoBlaze & PicoBlaze CPLD \\
                \yes        & \yes        & \yes         & \yes      & \yes
            \end{tabular}

\clearpage
    \instruction{OUTPUTK, OUTK}
        The \texttt{OUTPUTK} instruction is basically the same as the \texttt{OUTPUT} instruction, difference between these two instructions is in write strobe (please refer to PicoBlaze manual) and addressing of the 1st operand. \texttt{OUTPUTK} copies immediate value (value being part of the instruction's opcode) from the 1st operand to the specified port address. \texttt{OUTK} is only shorthand for \texttt{OUTPUTK}, these two mnemonics can be used interchangeably.

        \paragraph{Syntax}
            ~\\
            \verb'    OUTPUTK #kk, p    '; Copy immediate value kk to port at address p (in range 0x0..0xF).\\
            \verb'    OUTK    #kk, p    '; Same as above.\\
            \verb'    OUTPUTK #kk, @sY  '; Copy immediate value kk to port at address given by sY register.\\
            \verb'    OUTK    #kk, @sY  '; Same as above.

        \paragraph{Examples}
            ~\\
            \verb'    OUTK    #0xAA, 0x2  ; Write 0xAA to port at address 0x2.'\\
            \verb''\\
            \verb'    LD      S1, #0x1    ; S1 = 0x1'\\
            \verb'    OUTK    #0x55, @S1  ; Write 0x55 to port at address given by S1 reg. (0x1).'

        \paragraph{Flags}
            ~\\\indent
            \begin{tabular}{ll}
                Z & no effect \\
                C & no effect
            \end{tabular}

        \paragraph{Availability}
            ~\\\indent
            \begin{tabular}{ccccc}
                PicoBlaze 6 & PicoBlaze 3 & PicoBlaze II & PicoBlaze & PicoBlaze CPLD \\
                \yes        & \no         & \no          & \no       & \no
            \end{tabular}

\clearpage
\subsection{Storage}
    \instruction{STORE, ST}
        The \texttt{STORE} instruction copies from the specified register into the scratch pad ram memory (SPR) at the specified address. \texttt{ST} is only shorthand for \texttt{STORE}, these two mnemonics can be used interchangeably.

        \paragraph{Syntax}
            ~\\
            \verb'    STORE sX, ss   '; Copy from register sX to SPR at address ss.\\
            \verb'    STORE sX, @sY  '; Copy from register sX to SPR at address pointed by reg. sY.

        \paragraph{Examples}
            ~\\
            \verb'    ; Store value of reg. S0 in SPR at address pointed by reg. S1.'\\
            \verb'    STORE   S0, @S1'\\
            \verb''\\
            \verb'    ; Store value of reg. S0 in SPR at address 0x22.'\\
            \verb'    STORE   S0, 0x22'

        \paragraph{Flags}
            ~\\\indent
            \begin{tabular}{ll}
                Z & no effect \\
                C & no effect
            \end{tabular}

        \paragraph{Availability}
            ~\\\indent
            \begin{tabular}{ccccc}
                PicoBlaze 6 & PicoBlaze 3 & PicoBlaze II & PicoBlaze & PicoBlaze CPLD \\
                \yes        & \yes        & \no          & \no       & \no
            \end{tabular}

\clearpage
    \instruction{FETCH, FT}
        The \texttt{FETCH} instruction copies from the scratch pad ram memory (SPR) at the specified address into the specified register. \texttt{FT} is only shorthand for \texttt{FETCH}, these two mnemonics can be used interchangeably.

        \paragraph{Syntax}
            ~\\
            \verb'    FETCH sX, ss   '; Copy from SPR at address ss to register sX.\\
            \verb'    FETCH sX, @sY  '; Copy from SPR at address pointed by reg. sY and copy it to reg. sX.

        \paragraph{Examples}
            ~\\
            \verb'    ; Fetch value from SPR at address pointed by reg. S1 and store it in S0 reg.'\\
            \verb'    FETCH   S0, @S1'\\
            \verb''\\
            \verb'    ; Fetch value from SPR at address 0x22 and store it in S0 reg.'\\
            \verb'    FETCH   S0, 0x22'

        \paragraph{Flags}
            ~\\\indent
            \begin{tabular}{ll}
                Z & no effect \\
                C & no effect
            \end{tabular}

        \paragraph{Availability}
            ~\\\indent
            \begin{tabular}{ccccc}
                PicoBlaze 6 & PicoBlaze 3 & PicoBlaze II & PicoBlaze & PicoBlaze CPLD \\
                \yes        & \yes        & \no          & \no       & \no
            \end{tabular}

\clearpage
\subsection{Interrupt group}
    \instruction{RETURNI, RETIE, RETID}
        Return from Interrupt Service Routine (ISR) while enabling or disabling further interrupts. \texttt{RETIE} stands for \textbf{RET}urn from \textbf{I}nterrupt and \textbf{E}nable, RETID stands for \textbf{RET}urn from \textbf{I}nterrupt and \textbf{D}isable

        \paragraph{Syntax}
            ~\\
            \verb'    RETURNI ENABLE    '; Return from ISR and enable interrupts.\\
            \verb'    RETURNI DISABLE   '; Return from ISR and disable interrupts.\\
            \verb'    RETIE             '; Same as ``RETURNI ENABLE''\\
            \verb'    RETID             '; Same as ``RETURNI DISABLE''

        \paragraph{Examples}
            ~\\
            \verb'    ORG     0x3D0       ; Interrupt vector.'\\
            \verb'    LOAD    S0, #0x55   ; (Load register S0 with immediate value 0x55.)'\\
            \verb'    RETURNI DISABLE     ; Return from ISR and disable further interrupts.'

        \paragraph{Flags}
            ~\\\indent
            \begin{tabular}{ll}
                Z & no effect \\
                C & no effect
            \end{tabular}

        \paragraph{Availability}
            ~\\\indent
            \begin{tabular}{ccccc}
                PicoBlaze 6 & PicoBlaze 3 & PicoBlaze II & PicoBlaze & PicoBlaze CPLD \\
                \yes        & \yes        & \yes         & \yes      & \yes
            \end{tabular}

\clearpage
    \instruction{ENABLE/DISABLE INTERRUPT, ENA, DIS}
        Enable or disable interrupts. \texttt{ENA} is only shorthand for \texttt{ENABLE INTERRUPT}, these two mnemonics can be used interchangeably. \texttt{DIS} is only shorthand for \texttt{DISABLE INTERRUPT}, these two mnemonics can be used interchangeably.

        \paragraph{Syntax}
            ~\\
            \verb'    ENABLE INTERRUPT  '; Enable interrupts.\\
            \verb'    DISABLE INTERRUPT '; Disable interrupts.\\
            \verb'    ENA               '; Same as ``ENABLE INTERRUPT''.\\
            \verb'    DIS               '; Same as ``DISABLE INTERRUPT''.

        \paragraph{Examples}
            ~\\
            \verb'    DIS                 ; Timing critical code begins here, disable interrupts.'\\
            \verb'    CALL    something   ; Call subroutine "something".'\\
            \verb'    ENA                 ; Timing critical code ends here, re-enable interrupts.'

        \paragraph{Flags}
            ~\\\indent
            \begin{tabular}{ll}
                Z & no effect \\
                C & no effect
            \end{tabular}

        \paragraph{Availability}
            ~\\\indent
            \begin{tabular}{ccccc}
                PicoBlaze 6 & PicoBlaze 3 & PicoBlaze II & PicoBlaze & PicoBlaze CPLD \\
                \yes        & \yes        & \yes         & \yes      & \yes
            \end{tabular}

\clearpage
\subsection{Program Control}
    \instruction{JUMP}
        Instruction \texttt{JUMP} loads program counter with the address specified by aaa operand or by @(sX, sY).

        \paragraph{Syntax}
            ~\\
            \verb'    JUMP aaa          '; Unconditional jump.\\
            \verb'    JUMP Z, aaa       '; Jump only if the Zero flag is set.\\
            \verb'    JUMP NZ, aaa      '; Jump only if the Zero flag is NOT set.\\
            \verb'    JUMP C, aaa       '; Jump only if the Carry flag is set.\\
            \verb'    JUMP NC, aaa      '; Jump only if the Carry flag is NOT set.\\
            \verb'    JUMP @(sX, sY)    '; Unconditional jump at sX[3..0]sY[7..0].

        \paragraph{Examples}
            ~\\
            \verb'my_label:'\\
            \verb'    ; ... code ...'\\
            \verb'    JUMP    my_label     ; Jump to label "my_label".'\\
            \verb''\\
            \verb'    JUMP    0x300 + 0xff ; Jump to address 3FF hexadecimal.'

        \paragraph{Flags}
            ~\\\indent
            \begin{tabular}{ll}
                Z & no effect \\
                C & no effect
            \end{tabular}

        \paragraph{Availability}
            ~\\\indent
            \begin{tabular}{ccccc}
                PicoBlaze 6 & PicoBlaze 3 & PicoBlaze II & PicoBlaze & PicoBlaze CPLD \\
                \yes        & \yes        & \yes         & \yes      & \yes
            \end{tabular}

\clearpage
    \instruction{CALL}
        Call subroutine at the address specified by aaa operand or by @(sX, sY).

        \paragraph{Syntax}
            ~\\
            \verb'    CALL aaa          '; Unconditional call.\\
            \verb'    CALL Z, aaa       '; Call only if the Zero flag is set.\\
            \verb'    CALL NZ, aaa      '; Call only if the Zero flag is NOT set.\\
            \verb'    CALL C, aaa       '; Call only if the Carry flag is set.\\
            \verb'    CALL NC, aaa      '; Call only if the Carry flag is NOT set.
            \verb'    JUMP @(sX, sY)    '; Unconditional subroutine call at sX[3..0]sY[7..0].

        \paragraph{Examples}
            ~\\
            \verb'subprog:'\\
            \verb'    ADD     S0, S1      ; S0 = S0 + S1'\\
            \verb'    SUB     S1, # 5 * 2 ; S1 = S1 + 7'\\
            \verb'    RETURN'\\
            \verb''\\
            \verb'    CALL    my_subprog'\\
            \verb''\\
            \verb'    CALL    40          ; Call subroutine at address 40 decimal.'

        \paragraph{Flags}
            ~\\\indent
            \begin{tabular}{ll}
                Z & no effect \\
                C & no effect
            \end{tabular}

        \paragraph{Availability}
            ~\\\indent
            \begin{tabular}{ccccc}
                PicoBlaze 6 & PicoBlaze 3 & PicoBlaze II & PicoBlaze & PicoBlaze CPLD \\
                \yes        & \yes        & \yes         & \yes      & \yes
            \end{tabular}

\clearpage
    \instruction{RETURN, RET}
        Return from subroutine. \texttt{RET} is only shorthand for \texttt{RETURN}, these two mnemonics can be used interchangeably.

        \paragraph{Syntax}
            ~\\
            \verb'    RETURN    '; Unconditional return.\\
            \verb'    RETURN Z  '; Return only if the Zero flag is set.\\
            \verb'    RETURN NZ '; Return only if the Zero flag is NOT set.\\
            \verb'    RETURN C  '; Return only if the Carry flag is set.\\
            \verb'    RETURN NC '; Return only if the Carry flag is NOT set.

        \paragraph{Examples}
            ~\\
            \verb'subr:'\\
            \verb'    ADD     S0, S1      ; S0 = S0 + S1'\\
            \verb'    RETURN  Z           ; Return if S0 contains zero value.'\\
            \verb'    LOAD    S0, #1      ; Load S1 with value 1.'\\
            \verb'    RET                 ; Return unconditionally.'\\
            \verb''\\
            \verb'    CALL    subr'

        \paragraph{Flags}
            ~\\\indent
            \begin{tabular}{ll}
                Z & no effect \\
                C & no effect
            \end{tabular}

        \paragraph{Availability}
            ~\\\indent
            \begin{tabular}{ccccc}
                PicoBlaze 6 & PicoBlaze 3 & PicoBlaze II & PicoBlaze & PicoBlaze CPLD \\
                \yes        & \yes        & \yes         & \yes      & \yes
            \end{tabular}

\clearpage
    \instruction{LOAD \& RETURN, LDRET}
        Load the specified register with the specified immediate value and return from subroutine. \texttt{LDRET} is only shorthand for \texttt{LOAD \& RETURN}, these two mnemonics can be used interchangeably.

        \paragraph{Syntax}
            ~\\
            \verb'    LOAD & RETURN     sX, #kk'\\
            \verb'    LD & RET          sX, #kk'\\
            \verb'    LDRET             sX, #kk'\\

        \paragraph{Examples}
            ~\\
            \verb'my_subroutine:'\\
            \verb'    ; ...'\\
            \verb'    LDRET   S0, #0x55'\\
            \verb''\\
            \verb'    CALL    my_subroutine'\\

        \paragraph{Flags}
            ~\\\indent
            \begin{tabular}{ll}
                Z & no effect \\
                C & no effect
            \end{tabular}

        \paragraph{Availability}
            ~\\\indent
            \begin{tabular}{ccccc}
                PicoBlaze 6 & PicoBlaze 3 & PicoBlaze II & PicoBlaze & PicoBlaze CPLD \\
                \yes        & \no         & \no          & \no       & \no
            \end{tabular}

\clearpage
\subsection{Version Control}
    \instruction{HWBUILD}
        Instruction \texttt{HWBUILD} load the specified register with "hwbuild".
        \paragraph{Syntax}
            ~\\
            \verb'    HWBUILD  sX'\\

        \paragraph{Examples}
            ~\\
            \verb'    HWBUILD S0'\\

        \paragraph{Flags}
            ~\\\indent
            \begin{tabular}{ll}
                Z & 1 if loaded value is zero, 0 otherwise \\
                C & 1
            \end{tabular}

        \paragraph{Availability}
            ~\\\indent
            \begin{tabular}{ccccc}
                PicoBlaze 6 & PicoBlaze 3 & PicoBlaze II & PicoBlaze & PicoBlaze CPLD \\
                \yes        & \no         & \no          & \no       & \no
            \end{tabular}


\clearpage
\section{Pseudo Instructions}
    \definecolor{psinstruction_bg}{rgb}{0.6, 0.9, 1.0}
\newcommand{\psinstruction}[1]{~\\[7pt]\addcontentsline{toc}{subsection}{#1}\colorbox{psinstruction_bg}{\parbox{\dimexpr\textwidth-2\fboxsep}{\color{black}\textbf{#1}}}\bigskip}

MDS assembler supports a number of pseudo instructions which can improve understandability of your source code. Assembler will replace these instructions with one or more PicoBlaze instructions to achieve the purpose of that pseudo-instruction.

\psinstruction{NOP}
    No operation. This instruction does not do anything, it just consumes processor time.

    \paragraph{Syntax}
        ~\\
        \verb'    NOP'

    \paragraph{Equivalent}
        ~\\
        \verb'    LOAD  s0, s0'

    \paragraph{Examples}
        ~\\
        \verb'    NOP'\\
        \verb'    NOP'\\
        \verb'    NOP'

\psinstruction{INC}
    Increments the given register value by one.

    \paragraph{Syntax}
        ~\\
        \verb'    INC   sX'

    \paragraph{Equivalent}
        ~\\
        \verb'    ADD   sX, #1'

    \paragraph{Examples}
        ~\\
        \verb'    INC   s0'

\clearpage
\psinstruction{DEC}
    Decrements the given register value by one.

    \paragraph{Syntax}
        ~\\
        \verb'    DEC   sX'

    \paragraph{Equivalent}
        ~\\
        \verb'    SUB   sX, #1'

    \paragraph{Examples}
        ~\\
        \verb'    DEC   s0'

\psinstruction{SETR}
    Sets the given register to binary ones.

    \paragraph{Syntax}
        ~\\
        \verb'    SETR  sX'

    \paragraph{Equivalent}
        ~\\
        \verb'    OR    sX, #0xFF'

    \paragraph{Examples}
        ~\\
        \verb'    SETR  s0'

\psinstruction{CLRR}
    Clears the given register (set it to 0).

    \paragraph{Syntax}
        ~\\
        \verb'    CLRR  sX'

    \paragraph{Equivalent}
        ~\\
        \verb'    AND   sX, #0'

    \paragraph{Examples}
        ~\\
        \verb'    CLRR  s0'

\clearpage
\psinstruction{CPL}
    Performs ones' complement with the given register.

    \paragraph{Syntax}
        ~\\
        \verb'    CPL  sX'

    \paragraph{Equivalent}
        ~\\
        \verb'    XOR   sX, #0xFF'

    \paragraph{Examples}
        ~\\
        \verb'    CPL   s0'

\psinstruction{CPL2}
    Performs two's complement with the given register.

    \paragraph{Syntax}
        ~\\
        \verb'    CPL2  sX'

    \paragraph{Equivalent}
        ~\\
        \verb'    XOR   sX, #0xFF'\\
        \verb'    ADD   sX, #1'

    \paragraph{Examples}
        ~\\
        \verb'    CPL2  s0'

\psinstruction{SETB}
    Sets one bit in the given register.

    \paragraph{Syntax}
        ~\\
        \verb'    SETB  sX, bit ; bit belongs to interval [0,7]'

    \paragraph{Equivalent}
        ~\\
        \verb'    OR    sX, # 1 << bit'

    \paragraph{Examples}
        ~\\
        \verb'    SETB  S0, 3'

\clearpage
\psinstruction{CLRB}
    Clears one bit in the given register.

    \paragraph{Syntax}
        ~\\
        \verb'    CLRB  sX, bit ; bit belongs to interval [0,7]'

    \paragraph{Equivalent}
        ~\\
        \verb'    AND   sX, # ( 0xFF ^ ( 1 << bit ) )'

    \paragraph{Examples}
        ~\\
        \verb'    CLRB  S0, 7'

\psinstruction{NOTB}
    Negates one bit in the given register.

    \paragraph{Syntax}
        ~\\
        \verb'    NOTB  sX, bit ; bit belongs to interval [0,7]'

    \paragraph{Equivalent}
        ~\\
        \verb'    XOR   sX, # ( ~( 1 << bit ) )'

    \paragraph{Examples}
        ~\\
        \verb'    NOTB  S0, 7'

\psinstruction{DJNZ}
    Decrements the given register and jumps at the given label until the register contains zero.

    \paragraph{Syntax}
        ~\\
        \verb'    DJNZ  sX, label'

    \paragraph{Equivalent}
        ~\\
        \verb'    SUB   sX, #1'\\
        \verb'    JUMP  NZ, label'

    \paragraph{Examples}
        ~\\
        \verb'    loop:             '\\
        \verb'        DJNZ  s0, loop'

\clearpage
\psinstruction{IJNZ}
    Increments the given register and jumps at the given label until the register contains zero.

    \paragraph{Syntax}
        ~\\
        \verb'    IJNZ sX, label'

    \paragraph{Equivalent}
        ~\\
        \verb'    ADD   sX, #1'\\
        \verb'    JUMP  NZ, label'

    \paragraph{Examples}
        ~\\
        \verb'    loop:             '\\
        \verb'        IJNZ  s0, loop'


\clearpage
\definecolor{asmdirective_bg}{rgb}{0.7, 1.0, 0.7}
\newcommand{\asmdirective}[1]{~\\[7pt]\addcontentsline{toc}{subsection}{#1}\colorbox{asmdirective_bg}{\parbox{\dimexpr\textwidth-2\fboxsep}{\color{black}\textbf{#1}}}\bigskip}

\section{Directives}

    Assembler directives are commands for the assembler executed at compilation time, their purpose is to instruct the assembler how to compile your code, to define constants, implement conditional compilation, and evaluate various things at compilation time.

    \asmdirective{INCLUDE}
        \index{INCLUDE}
        Compiler copies content of the specified file to line where this directive is used. Included files can include other files. Path of the included file might be specified as either absolute or relative; in case of relative path, the path is always relative to location of file in which the \texttt{INCLUDE} directive appears and optionally to any of the include path list specified as assembler option.

        \paragraph{Syntax}~\\
            \verb'    INCLUDE "file_name"'

        \paragraph{Examples}~\\
            \verb'    INCLUDE "some_file.asm"'\\
            \verb'    INCLUDE "sub_dir/another_file.asm"'\\
            \verb'    INCLUDE "C:/my_dir/my_file.asm"'\\
            \verb'    INCLUDE "C:\\my_dir\\my_file.asm"'\\
            \verb'    INCLUDE "/home/user/project/file.asm"'

    \asmdirective{END}
        \index{END}
        The \texttt{END} directive informs the assembler that it has reached the end of all source code. Assembler then ignores any code following this directive so everything after this directive is threated as comment.

        \paragraph{Syntax}~\\
            \verb'    END'

        \paragraph{Examples}~\\
            \verb'    LOAD  S0, S1'\\
            \verb'    END'\\
            \verb'    LOAD  S0, S1, S2, S3 ; This will not be processed by the assembler.'

    \clearpage
    \asmdirective{EQU}
        \index{EQU}
        \texttt{EQU} stands for \texttt{EQU}als, it defines a symbol and assigns it a numerical value. Such symbol is considered constant and therefore cannot be redefined. Constant symbols defined with directive \texttt{EQU} can be used as register addresses, port addresses, and many others.

        \paragraph{Syntax}~\\
            \verb'    <symbol> EQU <expression>'

        \paragraph{Examples}~\\
            \verb'    First_symb   EQU   0b10011100        ; Binary.'\\
            \verb'    Second_symb  EQU   47                ; Decimal.'\\
            \verb'    Third_symb   EQU   0x39              ; Hexadecimal.'\\
            \verb'    Fourth_symb  EQU   (A - 4) + 18 / B) ; An expression.'\\
            \verb'    Fifth_symb   EQU   0x09 << 2         ; Another expression.'\\
            \verb''\\
            \verb'                 LOAD  S0, #First_symb   ; Loads S0 register with 0b10011100.'

    \asmdirective{CONSTANT}
        \index{CONSTANT}
        This directive is nothing more or less than the \texttt{EQU} directive with another syntax.

        \paragraph{Syntax}~\\
            \verb'    CONSTANT <symbol>, <expression>'

    \clearpage
    \asmdirective{SET}
        \index{SET}
        The \texttt{SET} directive does the same thing as the \texttt{EQU} directive, the only difference is that symbols defined with \texttt{SET} are re-definable while symbols defined with \texttt{EQU} are constant.

        \paragraph{Syntax}~\\
            \verb'    <symbol> SET <expression>'

        \paragraph{Examples}~\\
            \verb'    my_symbol SET   0x10           ; my_symbol = 0x10'\\
            \verb'              LOAD  S0, #my_symbol ; Loads S0 register with immediate value 0x10.'\\
            \verb''\\
            \verb'    my_symbol SET   0x20           ; re-defining my_symbol to new value: 0x20'\\
            \verb'              LOAD  S0, #my_symbol ; Loads S0 register with immediate value 0x20.'

    \asmdirective{VARIABLE}
        \index{VARIABLE}
        This directive is nothing more or less than the \texttt{SET} directive with another syntax.

        \paragraph{Syntax}~\\
            \verb'    VARIABLE <symbol>, <expression>'

        \paragraph{Examples}~\\
            \verb'    VARIABLE  First_symb, 0b10011100        ; Binary.'\\
            \verb'    VARIABLE  Second_symb, 47               ; Decimal.'\\
            \verb'    VARIABLE  Third_symb, 0x39              ; Hexadecimal.'\\
            \verb'    VARIABLE  Fourth_symb, (A -4)+ 18 / B)  ; An expression.'\\
            \verb'    VARIABLE  Fifth_symb, 0x09 << 2         ; Another expression.'\\
            \verb''\\
            \verb'    LOAD      S0, #First_symb               ; Loads S0 with 0b10011100.'

    \clearpage
    \asmdirective{REG}
        \index{REG}
        Symbols defined with the \texttt{REG} directive are considered to be register addresses only and cannot be used for anything else, except for that \texttt{REG} is just another \texttt{EQU}.

        \paragraph{Syntax}~\\
            \verb'    <symbol> REG <address>'

        \paragraph{Examples}~\\
            \verb'    A_reg  REG    s1'\\
            \verb'    B_reg  REG    s2'\\
            \verb'    C_reg  REG    s3'\\
            \verb'    D_reg  REG    0x4'\\
            \verb'    E_reg  REG    0x5'\\
            \verb''\\
            \verb'           LOAD   A_reg, D_reg  ; S0 = S5'\\
            \verb'           LOAD   B_reg, #0x55  ; S2 = 0x55'

    \asmdirective{NAMEREG}
        \index{NAMEREG}
        This directive is nothing more the \texttt{REG} directive with another syntax.

        \paragraph{Syntax}~\\
            \verb'    NAMEREG <symbol>, <address> '

        \paragraph{Examples}~\\
            \verb'    NAMEREG     a, s1'\\
            \verb'    NAMEREG     b, s2'\\
            \verb'    NAMEREG     x, s3'\\
            \verb'    NAMEREG     y, 4'\\
            \verb'    NAMEREG     z, 0xA'
            \verb''\\
            \verb'    LOAD        a, b         ; S1 = S2'\\
            \verb'    LOAD        x, #0x55     ; S3 = 0x55'

    \clearpage
    \asmdirective{DATA}
        \index{DATA}
        Symbols defined with the \texttt{DATA} directive are considered to be scratch-pad ram addresses only and cannot be used for anything else, except for that \texttt{DATA} is just another \texttt{EQU}.

        \paragraph{Syntax}~\\
            \verb'    <symbol> DATA <expression>'

        \paragraph{Examples}~\\
            \verb'    my_location   PORT    0x12'\\
            \verb''\\
            \verb'                  STORE   S0, my_location'

    \asmdirective{CODE}
        \index{CODE}
        Symbols defined with the \texttt{CODE} directive are considered to be program memory addresses only and cannot be used for anything else, except for that \texttt{CODE} is just another \texttt{EQU}.

        \paragraph{Syntax}~\\
            \verb'    <symbol> CODE <expression>'

        \paragraph{Examples}~\\
            \verb'    somewhere     CODE    0x3ff'\\
            \verb'                  ; ...'\\
            \verb'                  ORG     somewhere'\\
            \verb'                  ; ...'\\
            \verb'                  CALL    somewhere'

    \clearpage
    \asmdirective{PORT}
        \index{PORT}
        Symbols defined with the \texttt{PORT} directive are considered to be port addresses only and cannot be used for anything else, except for that \texttt{PORT} is just another \texttt{EQU}.

        \paragraph{Syntax}~\\
            \verb'    <symbol> PORT <expression>'

        \paragraph{Examples}~\\
            \verb'    my_port       PORT    0x22'\\
            \verb''\\
            \verb'                  OUTPUT  S0, my_port'

    \asmdirective{PORTIN}
        \index{PORTIN}
        Symbols defined with the \texttt{PORTIN} behaves the same as if they were defined with the \texttt{PORT} directive but with one exception, \texttt{PORTIN} is intended for specifying input port addresses and some tools might rely on that. Generally it is a good idea to avoid using \texttt{PORT} as much as possible and use \texttt{PORTIN} and \texttt{PORTOUT} instead.

        \paragraph{Syntax}~\\
            \verb'    <symbol> PORTIN <expression>'

        \paragraph{Examples}~\\
            \verb'    my_port       PORTIN  0x22'\\
            \verb''\\
            \verb'                  INPUT   S0, my_port'

    \asmdirective{PORTOUT}
        \index{PORTOUT}
        Symbols defined with the \texttt{PORTOUT} behaves the same as if they were defined with the \texttt{PORT} directive but with one exception, \texttt{PORTOUT} is intended for specifying output port addresses and some tools might rely on that. Generally it is a good idea to avoid using \texttt{PORT} as much as possible and use \texttt{PORTIN} and \texttt{PORTOUT} instead.

        \paragraph{Syntax}~\\
            \verb'    <symbol> PORTOUT <expression>'

        \paragraph{Examples}~\\
            \verb'    my_port       PORTOUT 0x22'\\
            \verb''\\
            \verb'                  OUTPUT  S0, my_port'

    \clearpage
    \asmdirective{AUTOREG}
        \index{AUTOREG}
        It will automatically assign a register at some address starting from 0x00 which is incremented with every other \texttt{AUTOREG} directive by one or, if provided, the size argument. Optionally, you can change starting address counter by adding a parameter after \texttt{AUTOREG} directive. Symbols defined with this directive have the same purpose and limitations as if they were defined with the \texttt{REG} directive. You can check assigned registers in code listing (file .lst) and symbol table (file .sym). This directive may save you some time, you can use it when you don't care which exact register will be used.

        \paragraph{Syntax}~\\
            \verb'    <symbol> AUTOREG <size>'
            \verb'    <symbol> AUTOREG [AT <address>]'

        \paragraph{Examples}~\\
            \verb'    reg_1  AUTOREG                ; reg_1 = 0'\\
            \verb'    reg_2  AUTOREG                ; reg_2 = 1'\\
            \verb'    reg_3  AUTOREG                ; reg_3 = 2'\\
            \verb'    reg_4  AUTOREG AT 10          ; Start counting from 10 so reg_4 =10'\\
            \verb'    reg_5  AUTOREG                ; my_reg_5 = 11'
            \verb''\\
            \verb'           LOAD     reg_3, reg_4  ; S2 = SA'\\
            \verb'           LOAD     reg_1, #0x22  ; S0 = 0x22'

    \asmdirective{AUTOSPR}
        \index{AUTOSPR}
        This directive provides exactly the same functionality as the \texttt{AUTOREG} directive but for addresses in scratch-pad ram. Symbols defined with this directive have the same purpose and limitations as if they were defined with the DATA directive.

        \paragraph{Syntax}~\\
            \verb'    <symbol> AUTOSPR <size>'
            \verb'    <symbol> AUTOSPR [AT <address>]'

        \paragraph{Examples}~\\
            \verb'    my_data  AUTOSPR'\\
            \verb''\\
            \verb'             STORE    S0, my_data'

    \clearpage
    \asmdirective{INITSPR}
        \index{INITSPR}
        Initializes scratch-pad RAM (SPR) with the given value(s), content of such initialized memory is stored in the Secondary Assembler Output (see the compiler configuration dialog, or command line option --secondary).

        \paragraph{Syntax}~\\
            \verb'    <symbol> INITSPR <value>'

        \paragraph{Examples}~\\
            \verb'my_data       INITSPR         "Hello PicoBlaze!"'\\
            \verb'my_data2      INITSPR         0x2b'\\
            \verb''\\
            \verb'              FETCH           S0, my_data'\\
            \verb'              FETCH           S1, my_data + 1'\\
            \verb'              FETCH           S2, my_data + 2'\\
            \verb'              FETCH           S3, my_data + 3'\\
            \verb''\\
            \verb'              FETCH           S8, my_data2'

    \asmdirective{ORGSPR}
        \index{ORGSPR}
        Specify address of origin for scratch-pad RAM initialization (directive \texttt{INITSPR}).

        \paragraph{Syntax}~\\
            \verb'    ORGSPR <address>'

        \paragraph{Examples}~\\
            \verb'              ORGSPR          0x10'\\
            \verb'my_data       INITSPR         "Hello PicoBlaze!" ; <-- address assigned to my_data is 0x10'\\
            \verb''\\
            \verb'              FETCH           S0, my_data'\\
            \verb'              FETCH           S1, my_data + 1'\\
            \verb'              FETCH           S2, my_data + 2'

    \clearpage
    \asmdirective{MERGESPR}
        \index{MERGESPR}
        Merge scratch-pad RAM initialization with program memory initialization at the specified address.

        \paragraph{Syntax}~\\
            \verb'    MERGESPR <address>'

        \paragraph{Examples}~\\
            \verb'              MERGESPR        0x280'\\
            \verb'my_data       INITSPR         "Hello PicoBlaze!"'\\
            \verb''\\
            \verb'              FETCH           S0, my_data'\\
            \verb'              FETCH           S1, my_data + 1'\\
            \verb'              FETCH           S2, my_data + 2'

    \asmdirective{STRING}
        \index{STRING} \label{STRING directive}
        Defines a named character string (sequence of characters) which can later be used with \texttt{LOAD \& RETURN} and \texttt{OUTPUTK} instructions, and with \texttt{DB} directive.

        \paragraph{Syntax}~\\
            \verb'    <name> STRING "<string>"'
            \verb'    STRING <name>, "<string>"'

        \paragraph{Examples}~\\
            \verb'my_string     STRING          "Hello PicoBlaze!"'\\
            \verb''\\
            \verb'              LOAD & RETURN   S0, my_string'\\
            \verb'              OUTPUTK         my_string, 2'\\
            \verb'              DB              my_string'

    \clearpage
    \asmdirective{DEFINE}
        \index{DEFINE}
        Define and expression which is evaluated every time separately when used in the code. These expressions can handle unlimited number of parameters, parameters are defined in curly brackets and are numbered from 0 to infinity (in decimal radix), using expressions with parameters resembles calling a function in C language, please see the example below.

        \paragraph{Syntax}~\\
            \verb'    <symbol> DEFINE <expression>'

        \paragraph{Examples}~\\
            \verb'    A     EQU     10              ; A = 10 (decimal)'\\
            \verb'    B     SET     25              ; B = 25 (decimal)'\\
            \verb'    C     DEFINE  ( A + B ) * 2   ; Value of C is unknown for now.'\\
            \verb''\\
            \verb'          LOAD    S0, #C          ; Load S0 with ( ( 10 + 25 ) * 2 ) = 70.'\\
            \verb''\\
            \verb'    B     SET     11              ; B = 11 (decimal)'\\
            \verb'          LOAD    S0, #C          ; Now load S0 with ( ( 10 + 11 ) * 2 ) = 42.'\\
            \verb''\\
            \verb''\\
            \verb'    X     DEFINE  ( {0} + {1} )   ; Value of C is unknown for now.'\\
            \verb'          LOAD    S0, #X(4, 5)    ; Now load S0 with ( 4 + 5 ) = 9.'

    \asmdirective{ORG, ADDRESS}
        \index{ORG} \index{ADDRESS}
        The assembler maintains a location counter for program memory, this location counter is incremented with each assembled instruction. With \texttt{ORG} or \texttt{ADDRESS} directive this location counter can be changed to instruct the assembler to start writing the code following the \texttt{ORG} directive at the new location counter position.

        \paragraph{Syntax}~\\
            \verb'    ORG     <expression>'\\
            \verb'    ADDRESS <expression>'


        \paragraph{Examples}~\\
            \verb'    ORG   0x3ff             ; Suppose that 0x3ff is the address for ISR.'\\
            \verb'    JUMP  handle_interrupt'

    \clearpage
    \asmdirective{SKIP}
        \index{SKIP}
        Do not initialize the given number of program memory words and skip to the next nearest location.

        \paragraph{Syntax}~\\
            \verb'    SKIP <expression>'

        \paragraph{Examples}~\\
            \verb'    ORG   0'\\
            \verb'    LOAD  S0, #0x22  ; Put opcode at address 0x0.'\\
            \verb'    LOAD  S0, #0x22  ; Put opcode at address 0x1.'\\
            \verb'    LOAD  S0, #0x22  ; Put opcode at address 0x2.'\\
            \verb'    LOAD  S0, #0x22  ; Put opcode at address 0x3.'\\
            \verb'    SKIP  5          ; Skip next 5 program memory locations.'\\
            \verb'    LOAD  S0, #0x22  ; Put opcode at address 0x8.'\\
            \verb'    LOAD  S0, #0x22  ; Put opcode at address 0x9.'

    \asmdirective{UNDEFINE, UNDEF}
        \index{UNDEFINE} \index{UNDEF}
        All symbols can be undefined, undefined symbols are be deleted from the symbol table and compiler will not recognize them.

        \paragraph{Syntax}~\\
            \verb'    UNDEFINE <symbol>'\\
            \verb'    UNDEF    <symbol>'

        \paragraph{Examples}~\\
            \verb'    symbol  SET     15'\\
            \verb'            LOAD    S0, #symbol'\\
            \verb'            UNDEF   symbol'\\
            \verb'            LOAD    s0, #symbol  ; This will cause compilation error.'

    \clearpage
    \asmdirective{DB}
        \index{DB}
        This directive initializes the program memory directly, it can be used for direct writing of instruction opcodes. Memory is initialized in two different ways: in case of string given as argument to the directive, program memory will be initialized byte by byte; in case of constants and expressions, each constant or expression initializes one instruction word. If instruction word is 18 bits wide, the MSB of the byte triplet will be trimmed to 2 bits, making entire triplet only 18 bits wide instead of 24.

        \paragraph{Syntax}~\\
            \verb'    ; Expresion syntax'\\
            \verb'    DB  <expression1>  [, <expression2>, ...]'\\
            \verb''\\
            \verb'    ; String syntax'\\
            \verb'    DB <"string">'\\
            \verb''\\
            \verb'    ; Combination of string(s) and expression(s)'\\
            \verb'    DB <"string"> [, <expression1>, ...]'

            Parameter can be unlimited number of string characters, or expressions divided by comma.


        \paragraph{Examples}~\\
            \verb'    DB  0x060FC                 ; Constant.'\\
            \verb'    DB  "my string"             ; String.'\\
            \verb'    DB  "my string", 2+1, 3     ; Combination of string, expression, and constant.'

    \asmdirective{LIMIT}
        \index{LIMIT}
        Imposes user defined limit on size of register file, scratch-pad ram, or program memory. In the example below if you use 8 registers or \texttt{JUMP} to address higher than 512, compiler reports such attempt as error.

        \paragraph{Syntax}~\\
            \verb'    LIMIT  D, <number> ; Size of scratch-pad RAM (D stands for data).'\\
            \verb'    LIMIT  R, <number> ; Number of registers (R stands for registers).'\\
            \verb'    LIMIT  C, <number> ; Size of program memory (C stands for code).'

        \paragraph{Examples}~\\
            \verb'    LIMIT  R, 8   ; Limit maximum register address to 7.'\\
            \verb'    LIMIT  D, 32  ; Limit maximum address in scratch-pad ram to 31. '\\
            \verb'    LIMIT  C, 512 ; Limit maximum address in program memory to 511.'

    \clearpage
    \asmdirective{DEVICE}
        \index{DEVICE}
        Normally, you choose the target architecture when you are creating a project. But you can also specify target architecture with \texttt{DEVICE} directive. This will affect predefined symbols.

        \paragraph{Syntax}~\\
            \verb'    DEVICE <device_name>'

        \paragraph{Examples}~\\
            \verb'    DEVICE kcpsm6'\\
            \verb'    DEVICE kcpsm3'\\
            \verb'    DEVICE kcpsm2'\\
            \verb'    DEVICE kcpsm1'\\
            \verb'    DEVICE kcpsm1cpld'

    \asmdirective{LIST, NOLIST}
        \index{LIST} \index{NOLIST}
        Temporarily turns on and off output to the code listing.

        \paragraph{Syntax}~\\
            \verb'    LIST    ; Turn code listing ON.'\\
            \verb'    NOLIST  ; Turn code listing OFF.'

        \paragraph{Examples}~\\
            \verb'    NOLIST'\\
            \verb'    INCLUDE "some_file.asm" ; The included file will not appear in the code listing.'\\
            \verb'    LIST'

    \clearpage
    \asmdirective{TITLE}
        \index{TITLE}
        Set title for code listing.

        \paragraph{Syntax}~\\
            \verb'    TITLE  "<title text>"'

        \paragraph{Examples}~\\
            \verb'    TITLE  "My program for something, etc."'

    \asmdirective{MESSAGE}
        \index{MESSAGE}
        Print compiler message, the message will be printed by the compiler in the same was as errors and warnings are. Such message, however, is not considered to be neither error nor warning.

        \paragraph{Syntax}~\\
            \verb'    MESSAGE "some message..."'

        \paragraph{Examples}~\\
            \verb'    MESSAGE "text text text..."'

    \asmdirective{ERROR}
        \index{ERROR}
        This directive does the same things as the \texttt{MESSAGE} directive but in this case the printed message is considered as an error and causes the assembler to consider the entire compilation unsuccessful.

        \paragraph{Syntax}~\\
            \verb'    ERROR "error message"'

        \paragraph{Examples}~\\
            \verb'    ERROR "my error message"'

    \asmdirective{WARNING}
        \index{WARNING}
        This directive does the same things as the \texttt{MESSAGE} directive but in this case the printed message is considered as a warning.

        \paragraph{Syntax}~\\
            \verb'WARNING "warning message"'

        \paragraph{Examples}~\\
            \verb'WARNING "my warning message"'

    \clearpage
    \asmdirective{REPEAT}
        \index{REPEAT}
        Repeats the specified block of code for the specified number of times. \texttt{REPT} is shortcut for \texttt{REPT}, and \texttt{ENDR} is shortcut for \texttt{ENDREPEAT}.

        \paragraph{Syntax}~\\
            \verb'    REPEAT <number-of-repeats>'\\
            \verb'        <code>'\\
            \verb'    ENDREPEAT'\\
            \verb''\\
            \verb'    REPT <number-of-repeats>'\\
            \verb'        <code>'\\
            \verb'    ENDR'

        \paragraph{Examples}~\\
            \verb'    REPT          5'\\
            \verb'        SR0       sF'\\
            \verb'    ENDR'\\
            \verb''\\
            \verb'    ; Equivalent to.'\\
            \verb'    SR0           sF'\\
            \verb'    SR0           sF'\\
            \verb'    SR0           sF'\\
            \verb'    SR0           sF'\\
            \verb'    SR0           sF'

    \clearpage
    \asmdirective{\#WHILE}
        \index{\#WHILE} \index{\#ENDWHILE} \index{\#ENDW}
        Repeats the specified block of code until expression equals to zero. \texttt{\#ENDW} is shortcut for \texttt{\#ENDWHILE}.

        \paragraph{Syntax}~\\
            \verb'    #WHILE <expression>'\\
            \verb'        <code>'\\
            \verb'    #ENDWHILE'\\
            \verb''\\
            \verb'    #WHILE <expression>'\\
            \verb'        <code>'\\
            \verb'    #ENDW'

        \paragraph{Examples}~\\
            \verb'                ld          S0, #0x44       ; (value to output)'\\
            \verb'    addr        set         0               ; (starting address)'\\
            \verb''\\
            \verb'            #while ( addr < 5 )             ; Repeat while "addr" is lower than 5.'\\
            \verb'                out         S0, addr'\\
            \verb'    addr        set         addr + 1        ; Redefine "addr": addr := addr + 1.'\\
            \verb'            #endwhile                       ; End the while loop.'

    \asmdirective{FAILJMP, DEFAULT\_JUMP}
        \index{FAILJMP} \index{DEFAULT\_JUMP}
        Fills program memory with jumps to the specified address. Purpose of this directive is to provide a simple means of protection against random errors.

        \paragraph{Syntax}~\\
            \verb'    FAILJMP     <expression>'\\
            \verb'    DEFAULT_JMP <expression>'

        \paragraph{Examples}~\\
            \verb'    something_is_wrong:'\\
            \verb'          ; ... do something ...'\\
            \verb''\\
            \verb'    FAILJMP  something_is_wrong'

    \clearpage
    \asmdirective{ENTITY}
        \index{ENTITY}
        Specifies VHDL entity name to use when filling VHDL template, by default the entity name is the base name of your source code file (without file extension, case sensitive). Entity name is case sensitive and has to be enclosed in double quotes ("). Assembler does not check whether the entity name is a valid VHDL identifier!

        \paragraph{Syntax}~\\
            \verb'    ENTITY  "<name>"'\\

        \paragraph{Examples}~\\
            \verb'    entity   "my_entity_abc"'

\clearpage
\section{Code generation directives}
    MDS assembler supports several special directives for automated generation of run-time loops and conditions. Note that condition and loop blocks may contain any other code including other loops and conditions.

    \paragraph{Loops:} Instead of writing loops with loads, compares, and jumps, you might find it to be more straightforward to use the assembler to generate them for you. You can use three types of \texttt{FOR} loop and one type of \texttt{WHILE} loop.

    \paragraph{Conditions:} Instead of writing conditional branching using compares and jumps, you can let the assembler do at least some this work for you with \texttt{IF}, \texttt{ELSEIF}, \texttt{ELSE}, and \texttt{ENDIF} directives. This feature resembles C language but don't forget that you are still working with assembler, these branching directives are merely a "syntax sugar", they are translated as compare and conditional jump, nothing more.

    \paragraph{Condition syntax}
        ~\\``A'' and ``B'' can be either register address or immediate value, in case of immediate value it has to be prefixed with ``\texttt{\#}''. So immediate constants are specified with ``\texttt{\#}'' prefix. A value without ``\texttt{\#}'' is considered to be a register address.

        \begin{table}[h!]
            \centering
            \begin{tabular}{|c|l|l|}
                \hline
                \textbf{Syntax} & \textbf{Description} & \textbf{Example} \\\hline
                \verb'A == B' & equal to         & \verb'S0 == S1' \\
                \verb'A != B' & not equal to     & \verb'S0 != #0xA5' \\
                \verb'A >  B' & greater than     & \verb'#(0x5A + 2) > S0' \\
                \verb'A <  B' & lower than       & \verb'#A < my_reg' \\
                \verb'A >= B' & greater or equal & \verb'A >= #B' \\
                \verb'A <= B' & lower or equal   & \verb'S4 <= #B' \\
                \verb'A &  B' & bitwise AND      & \verb'#A & S0' \\
                \verb'A !& B' & bitwise NAND     & \verb'S0 !& S0' \\\hline
            \end{tabular}
            \caption{Condition syntax used for all code generation directives.}
        \end{table}

    \paragraph{Availability}
        ~\\\indent
        \begin{tabular}{ccccc}
            PicoBlaze 6 & PicoBlaze 3 & PicoBlaze II & PicoBlaze & PicoBlaze CPLD \\
            \yes        & \yes        & \no          & \no       & \no
        \end{tabular}

    \clearpage
    \asmdirective{IF, ELSEIF, ELSE, ENDIF}
        \index{IF} \index{ELSEIF} \index{ELSE} \index{ENDIF}
        To implement run-time conditions you can use \texttt{IF}, \texttt{ELSEIF}, \texttt{ELSE}, and \texttt{ENDIF} directives for better readability of you code. Assembler translates these directives as predefined macros containing \texttt{COMPARE}, \texttt{TEST}, and \texttt{JUMP} instructions. You can use registers and immediate constants in conditions.

        \paragraph{Syntax}~\\
            \verb'    IF      <condition>'\\
            \verb'        <code>'\\
            \verb'    ELSEIF  <condition>'\\
            \verb'        <code>'\\
            \verb'    ELSE'\\
            \verb'        <code>'\\
            \verb'    ENDIF'

        \subsubsection{Example}
            \verb'    IF          s0 == #10         ; Register to immediate value.'\\
            \verb'        LOAD    s0, #10h'\\
            \verb'    ELSEIF      B >= S1           ; Register to register.'\\
            \verb'        SR0     s0'\\
            \verb'    ELSE        #0x5 >= #0x6      ; Immediate value to immediate value.'\\
            \verb'        INPUT   s0,RX_data'\\
            \verb'    ENDIF'

            ~\\In this example, the first condition compares register S0 to immediate value of 10 (decimal). The second condition compares register at address given by ``B'' symbol to register S1, and the third condition compares two immediate values (in this case the result of comparison is known in advance and the assembler with exploit that fact and print warning).

    \clearpage
    \asmdirective{WHILE, ENDWHILE}
        \index{WHILE} \index{ENDW} \index{ENDWHILE}
        To implement run-time loops you can use the \texttt{WHILE} directive. Assembler translates the \texttt{WHILE} directive to \texttt{COMPARE}, \texttt{TEST}, and \texttt{JUMP} instructions to implement the loop. Directive \texttt{ENDWHILE} closes the loop body; \texttt{ENDW} is only shortcut for \texttt{ENDWHILE}, they can be used interchangeably.

        \paragraph{Syntax}~\\
            \verb'    WHILE <condition>'\\
            \verb'          <code>'\\
            \verb'    ENDWHILE'

        \subsubsection{Example}
            \verb'    load        S0, #0xAA       ; (value to output)'\\
            \verb'    load        S1, #0          ; (starting address)'\\
            \verb''\\
            \verb'    while       S1 < #5         ; C: while ( S1 < 5 ) {'\\
            \verb'        output  S0, @S1         ; C:     S0 = *S1;'\\
            \verb'        inc     S1              ; C:     S1++;'\\
            \verb'    endwhile                    ; C: }'

    \clearpage
    \asmdirective{FOR, ENDFOR}
        \index{FOR} \index{ENDFOR} \index{ENDF}
        The \texttt{FOR} directive provides another way to relatively easily implement run-time program loops, it is best demonstrated on examples (see below). Directive \texttt{ENDFOR} closes the loop body; \texttt{ENDF} is only shortcut for \texttt{ENDFOR}, they can be used interchangeably.

        \paragraph{Syntax}~\\
            \verb'    FOR   <iterator-register>, <number-of-iterations>'\\
            \verb'          <code>'\\
            \verb'    ENDFOR'\\
            \verb''\\
            \verb'    FOR   <iterator-register>, <start> .. <end>'\\
            \verb'          <code>'\\
            \verb'    ENDFOR'\\
            \verb''\\
            \verb'    FOR   <iterator-register>, <start> .. <end>, <step>'\\
            \verb'          <code>'\\
            \verb'    ENDFOR'

        \paragraph{Examples}~\\
            \verb'    ; S0 starts from 0 and goes up to 9 (S0 is incremented after each iteration).'\\
            \verb'    FOR s0, 10'\\
            \verb'        NOP'\\
            \verb'    ENDF'\\
            \verb''\\
            \verb'    ; S0 goes from 10 to 15 (6 iterations: 10, 11, 12, 13, 14, 15).'\\
            \verb'    FOR s0, 10..15'\\
            \verb'        NOP'\\
            \verb'    ENDF'\\
            \verb''\\
            \verb'    ; S0 goes from 10 to 50 by steps of 10 (5 iterations: 10, 20, 30, 40, 50).'\\
            \verb'    FOR s0, 10..50, 10'\\
            \verb'        NOP'\\
            \verb'    ENDF'

\clearpage
\section{Conditional Assembly Directives}
    \index{Conditional Assembly}
    \index{\#IF} \index{\#IFN} \index{\#IFB} \index{\#IFNB} \index{\#IFDEF} \index{\#IFNDEF} \index{\#ELSE}
    \index{\#ELSEIF} \index{\#ELSEIFN} \index{\#ELSEIFB} \index{\#ELSEIFNB} \index{\#ELSEIFDEF} \index{\#ELSEIFNDEF}
    \index{\#ENDIF}

    The aim of the conditional assembly to to assemble certain parts of the code if and only if certain arithmetically expressed condition is met. This feature can prove useful particularly when the user want to make the code somehow ``configurable''. Note that condition blocks may contain any other code including other conditions. This assembler provides these directives to cope with conditional assembly:

    \begin{description}
        \item[\texttt{\#IF} <expression>]~\\
            Compiles the following block only if the expression value is not zero.
        \item[\texttt{\#IFN} <expression>]~\\
            Means ``If Not'', compiles the following block only if the expression value is zero.
        \item[\texttt{\#IFB} <macro-parameter>]~\\
            Means ``If Blank'', compiles the following block only if the macro-parameter is blank.
        \item[\texttt{\#IFNB} <macro-parameter>]~\\
            Means ``IF Not Blank'', compiles the following block only if the macro-parameter is not blank.
        \item[\texttt{\#IFDEF}  <symbol>]~\\
            Means ``IF DEFined'', compiles the following block only if the symbol is defined.
        \item[\texttt{\#IFNDEF} <symbol>]~\\
            Means ``IF Not DEFined'', compiles the following block only if the symbol is not defined.
        \item[\texttt{\#ELSE}]~\\
            Compiles the following block only if none of the previous conditions was met.
        \item[\texttt{\#ELSEIF} <expression>]~\\
            Compiles the following block only if none of the previous conditions was met and if the expression value is not zero.
        \item[\texttt{\#ELSEIFN} <expression>]~\\
            Compiles the following block only if none of the previous conditions was met and if the expression value is zero.
        \item[\texttt{\#ELSEIFB} <macro-parameter>]~\\
            Compiles the following block only if none of the previous conditions was met and if the macro-parameter is blank.
        \item[\texttt{\#ELSEIFNB} <macro-parameter>]~\\
            Compiles the following block only if none of the previous conditions was met and if the macro-parameter is not blank.
        \item[\texttt{\#ELSEIFDEF} <symbol>]~\\
            Compiles the following block only if none of the previous conditions was met and if the symbol is defined.
        \item[\texttt{\#ELSEIFNDEF} <symbol>]~\\
            Compiles the following block only if none of the previous conditions was met and if the symbol is not defined.
        \item[\texttt{\#ENDIF}]~\\
            Closes the tree of conditions, using this directive is mandatory.
    \end{description}

    \subsection{Example}
        ~\\
        {
            \fontsize{8pt}{10pt}\selectfont
            \verb'    abc     equ     14           ; Assign number 14 to symbol abc.'\\
            \verb'    xyz     equ     10           ; Assign number 10 to symbol abc.'\\
            \verb''\\
            \verb'    #ifdef abc                   ; <--+ Assemble only if symbol abc has been defined.'\\
            \verb'      #if ( abc = 13 )           ;    | <--+ Assemble if 13 has been assigned to symbol abc.'\\
            \verb'            load     S0, #0b1101 ;    |    |'\\
            \verb'      #elseif ( abc = 14 )       ;    | <--+ Assemble if 14 has been assigned to symbol abc.'\\
            \verb'            load     S0, #0x21   ;    |    |'\\
            \verb'      #elseifn ( abc % 2 )       ;    | <--+ Assemble if the value assigned to symbol abc is even.'\\
            \verb'            load     S0, #abc    ;    |    |'\\
            \verb'      #else                      ;    | <--+ Else ..'\\
            \verb'            load     S0, #077    ;    |    |'\\
            \verb'      #endif                     ;    | <--+'\\
            \verb'    #elseifndef xyz              ; <--+ Assemble if symbol xyz has NOT been defined.'\\
            \verb'            clrr     S0          ;    |'\\
            \verb'    #else                        ; <--+ Else ...'\\
            \verb'      #ifn ( xyz mod 2 )         ;    | <--+ Assemble if ( yxz modulo 2 ) is 0.'\\
            \verb'            load     S0, #128    ;    |    |'\\
            \verb'      #endif                     ;    | <--+'\\
            \verb'    #endif                       ; <--+'
        }

\clearpage
\section{Macro processing directives}
    \index{Macro}
    \index{MACRO} \index{ENDM} \index{EXITM} \index{EXPAND} \index{NOEXPAND}
    Macro is a sequence of instructions which can be expanded anywhere in the code and for any number of times. That may reduce necessity of repeating code fragments as well as source code size and make the solved task easier to comprehend and solve. Unlike subprograms macros do not add extra run-time overhead, repeating usage of macros may significantly increase size of the resulting machine code. Note that macros may contain any other code including other macro expansions and/or definitions.

    Macros can have no parameters or any number of parameters, number of parameters is unlimited. All parameters are optional, parameters which has not been substituted with corresponding arguments are filled with blank values. Blank values have special meaning, cannot be used in arithmetical expressions or as operands but can be checked if they are blank or not using \texttt{\#IFB} and \texttt{\#IFNB} directives.

    \subsection{Syntax}
        \verb'    <name-of-macro>  MACRO  [<parameter1>]  [, <parameter2> ...]'\\
        \verb'                     <code>'\\
        \verb'    ENDM'

    \subsection{Directives}
        \begin{table}[h!]
            \begin{tabular}{ll}
                \texttt{MACRO}      & Define a new macro. \\
                \texttt{ENDM}       & End of macro definition. \\
                \texttt{EXITM}      & Exit macro expansion. \\
                \texttt{EXPAND}     & Disable macro expansions.\\
                \texttt{NOEXPAND}   & (Re-)enable macro expansions.\\
            \end{tabular}
        \end{table}

    \asmdirective{LOCAL}
        \index{LOCAL}
        Directive \texttt{LOCAL} declares a symbol as local for the macro in which it appears.

        \paragraph{Syntax}~\\
            \verb'    LOCAL <symbols>'

        \subsubsection{Example}
            \verb'    MACRO            my_macro'\\
            \verb'            LOCAL    wait'\\
            \verb''\\
            \verb'        wait:'\\
            \verb'            SUBCY    S0, #0x10'\\
            \verb'            SUB      S0, #0x01' \\
            \verb'            LOAD     S0, #0xF0'\\
            \verb'            JUMP     C,  wait'\\
            \verb'    ENDM'

    \clearpage
    \subsection{Examples}
        \paragraph{Macro without parameters}
            ~\\
            \verb'    abc     macro'\\
            \verb'            load    s2, s0'\\
            \verb'            add     s2, #1'\\
            \verb'            load    s1, s2'\\
            \verb'    endm'\\
            \verb''\\
            \verb'            abc           ; Expand macro "abc" here'\\
            \verb'            abc           ; And here...'\\
            \verb'            abc           ; And here...'\\
            \verb''\\
            \verb'    ; This produces the same result as if you wrote this:'\\
            \verb'    ;       load    s2, s0'\\
            \verb'    ;       add     s2, #1'\\
            \verb'    ;       load    s1, s2'\\
            \verb'    ;'\\
            \verb'    ;       load    s2, s0'\\
            \verb'    ;       add     s2, #1'\\
            \verb'    ;       load    s1, s2'\\
            \verb'    ;'\\
            \verb'    ;       load    s2, s0'\\
            \verb'    ;       add     s2, #1'\\
            \verb'    ;       load    s1, s2'\\

        \paragraph{Macro with parameters}
            ~\\
            \verb'    abc         macro   x, y'\\
            \verb'                load    x, #y'\\
            \verb'                load    x, y'\\
            \verb'    endm'\\
            \verb''\\
            \verb'                abc     s2, 3'\\
            \verb'        ; This produces this result:'\\
            \verb'        ;       load    s2, #3'\\
            \verb'        ;       load    s2, 3'

        \enlargethispage{5\baselineskip}
        \paragraph{Using blank parameters}
            ~\\
            \verb'    abc         macro   x, y'\\
            \verb'              #ifb      y     ; If blank...'\\
            \verb'                load    x, S0'\\
            \verb'              #else           ; Else...'\\
            \verb'                load    x, y'\\
            \verb'              #endif          ; End of condition.'\\
            \verb'    endm'\\
            \verb''\\
            \verb'                abc     S0, S1        ; Parameter y is S1 here.'\\
            \verb'        ; Produces this result:'\\
            \verb'        ;       load    S0, S1'\\
            \verb''\\
            \verb'                abc     S0            ; Parameter y is "blank" here.'\\
            \verb'        ; Produces this result:'\\
            \verb'        ;       load    S0, S0'\\

        \clearpage
        \paragraph{Premature end of macro expansion}
            ~\\
            \verb'    abc         macro   x, y'\\
            \verb'                load    x, #y'\\
            \verb'            #if y > 2'\\
            \verb'                exitm'\\
            \verb'            #endif'\\
            \verb'                load    x, y'\\
            \verb'    endm'\\
            \verb''\\
            \verb'                abc     s0, 1'\\
            \verb'        ; Produces:'\\
            \verb'        ;       load    s0, #1'\\
            \verb'        ;       load    s0, 1'\\
            \verb''\\
            \verb'                abc     s0, 3'\\
            \verb'        ; Produces:'\\
            \verb'        ;       load    s0, #1'\\

        \paragraph{A few simple practical examples}
            ~\\
            \verb'    ; Copy content of registers at addresses [source, source+4]'\\
            \verb'    ; to registers at addresses [target, target+4].'\\
            \verb'    copy5       macro       target, source'\\
            \verb'                load        target + 0, source + 0'\\
            \verb'                load        target + 1, source + 1'\\
            \verb'                load        target + 2, source + 2'\\
            \verb'                load        target + 3, source + 3'\\
            \verb'                load        target + 4, source + 4'\\
            \verb'    endm'\\
            \verb''\\
            \verb'                ; Copy [S0..S4] to [S5..S9]'\\
            \verb'                copy5       S5, S0'\\
            \verb''\\
            \verb''\\
            \verb'    ; Wait for the given number number of instruction cycles,'\\
            \verb'    ; and use the given register as iterator for the delay loop.'\\
            \verb'    wait        macro       register, cycles'\\
            \verb'                for         register, ( cycles - 1 ) / 4'\\
            \verb'                    nop'\\
            \verb'                endfor'\\
            \verb'    endm'\\
            \verb''\\
            \verb'                ; Wait 100 instruction cycles here.'\\
            \verb'                wait        S0, 100'


\clearpage
\section{Output files}
    \subsection{Generated VHDL and Verilog files}
        As you know, the PicoBlaze microprocessor is primarily designed for use in a VHDL design. MDS will generate all the necessary files that are needed for implementation in FPGA. The compiler will read a VHDL template and insert the generated machine code for PicoBlaze processor to complete the definition of program ROM and write the result into a new VHDL file that is ready for synthesis and simulation.

        Template can be modified to define alternative memory definitions. However, you are responsible for ensuring that the template is correct, the compiler does not perform any validity checks on the VHDL template.

        The compiler identifies certain strings enclosed by ``\texttt{\{\}}'' characters (marks), and substitutes these character strings with corresponding values. The MDS assembler replaces instances of \verb'{timestamp}', \verb'{name}', \verb'{INIT_X}', \verb'{INITP_X}', \verb'{INIT64_X}', \verb'{INIT128_X}', \verb'{INIT256_X}', \verb'{[8:0]_INIT_X}', \verb'{[8:0]_INITP_X}', \verb'{[17:9]_INIT_X}', \verb'{[17:9]_INITP_X}', and \verb'{begin template}'. Templates have to contain these marks for the compiler to work correctly.

    \subsection{MEM File}
        MEM file contains machine code generated by the assembler. There are 17 columns 4 bytes wide, the first column starts with ``\verb'@''' and represents address, other columns contain instruction opcodes. Unused locations are filled with zeros.

        \subsubsection{Example}
            \verb'@0000 000011F7 00001299 0000132E ... ... ... 00004577 00007789 000015A4'\\
            \verb'...'\\
            \verb'...'\\
            \verb'...'\\
            \verb'@0040 000004DF 000047F4 00000000 ... ... ... 00000000 00000000 00000000'\\
            \verb'@0080 00000000 00000000 00000000 ... ... ... 00000000 00000000 00000000'\\
            \verb'@00C0 00000000 00000000 00000000 ... ... ... 00000000 00000000 00000000'\\
            \verb'@0100 00000000 00000000 00000000 ... ... ... 00000000 00000000 00000000'\\
            \verb'...'\\
            \verb'...'\\
            \verb'...'\\
            \verb'@3F80 000011F7 00001299 0000132E ... ... ... 00000000 00000000 00000000'\\
            \verb'@3FC0 000011F7 00001299 0000132E ... ... ... 00000000 00000000 00000000'\\

    \clearpage
    \subsection{Raw Hex Dump file}
        Raw Hex Dump is very simple, file starts with hexadecimal representation of opcode of the first instruction in your program at address 0x0 then it is followed by opcode at address 0x1, 0x2, and so on. Unused locations are filled with zeros.

        \subsubsection{Example}
            ~\\
            \verb'011F7'\\
            \verb'01299'\\
            \verb'0132E'\\
            \verb'19101'\\
            \verb'00000'\\
            \verb'19201'\\
            \verb'00000'\\
            \verb'19301'\\
            \verb'00000'\\
            \verb'00000'\\
            \verb'00...'\\
            \verb'.....'\\
            \verb'...00'

    \subsection{Raw binary file}
        Raw binary file contains machine code generated by the assembler in raw form. It contains instruction opodes formatted either as a sequence of byte triplets (in case of 18-bit instruction opcodes), or sequence of byte pairs (in case of 16-bit instructions opcodes), byte order for these sequences is big-endian. Unused memory locations and higher bits of bytes which are not used in their entire width are filled with binary zeros, start address for the entire file is 0.

    \subsection{String table}
        String table is a human readable text file containing table of strings defined in your source code using the \texttt{STRING} directive.

        \subsubsection{File format}
            The table of strings consists of a number of lines with following format:

            \begin{tabular}{|ccc|}
                \hline
                <Name> & <Location> & <"Value"> \\\hline
            \end{tabular}

            \begin{description}
                \item[Name]~\\
                    Name of the string.
                \item[Location]~\\
                    Location of the string definition, formatted as:\\
                    \texttt{<file-name>:<line-number>.<column>.<line-number>.<column>}
                \item[Value]~\\
                    The assigned character string.
            \end{description}

    \subsection{Symbol table}
        Symbol table is a human readable text file containing table of symbols defined in your source code using labels, symbol definition directives (\texttt{EQU}, \texttt{SET}, \texttt{REG}, \texttt{DATA}, \texttt{CODE}, \texttt{AUTOREG}, \texttt{AUTOSPR}), and implicitly defined symbols for your processor.

        \subsubsection{File format}
            The table of symbols consists of a number of lines with following format:

            \begin{tabular}{|cccccc|}
                \hline
                <Name> & <Type> & <Value> & <Usage> & <Attribute> & <Location> \\\hline
            \end{tabular}

            \begin{description}
                \item[Name]~\\
                    Symbol name.
                \item[Type]~\\
                    Symbol type.
                    \begin{itemize}
                        \item PORT: PORT\_ID indicator.
                        \item DATA: Scratch-pad memory address.
                        \item LABEL: Address in program memory.
                        \item REGISTER: Address of an internal register.
                        \item EXPRESION: An expression.
                        \item NUMBER: Symbol is a general number.
                    \end{itemize}
                \item[Value]~\\
                    Value assigned to the symbol.
                \item[Usage]~\\
                    ``USED'' or ``NOT USED''.
                \item[Attribute]~\\
                    \begin{itemize}
                        \item IMPLICIT: Symbol is defined implicitly for your processor.
                        \item LOCAL: Local symbol in macro.
                        \item REDEFINABLE: Symbol is re-definable, i.e. is not constant.
                        \item CONSTANT: Symbol cannot be redefined, i.e. is constant.
                    \end{itemize}
                \item[Location]~\\
                    Location of symbol definition formatted as:\\
                    \texttt{<file-name>:<line-number>.<column>.<line-number>.<column>}
            \end{description}

    \clearpage
    \subsection{Macro table}
        Macro table is a human readable text file containing table of macros defined in your source code using the \texttt{MACRO} directive.

        \subsubsection{File format}
            The table of macros consists of a number of lines with following format:

            \begin{tabular}{|cccc|}
                \hline
                <Name> & ( <Parameters> ) & <Usage> & <Location> \\\hline
            \end{tabular}

            \begin{description}
                \item[Name]~\\
                    Name of macro.
                \item[Parameters]~\\
                    Macro parameters.
                \item[Usage]~\\
                    Information about how many times the macro was used.
                \item[Location]~\\
                    Location of the macro definition formatted as:\\
                    \texttt{<file-name>:<line-number>.<column>.<line-number>.<column>}
            \end{description}

    \enlargethispage{3\baselineskip}
    \subsection{Intel 8 HEX}
        Intel 8 HEX is a popular object file format capable of containing up to 64kB of data. Hex files have usually extension .hex or .ihx. These files are text files consisting of a sequence of records, each line line can contain at most one record. Records starts with ``\texttt{:}'' (colon) character at the beginning of the line and ends by end of the line. Everything else besides records should be ignored. Records consist of a sequence of 8-bit hexadecimal numbers (e.g. ``a2'' or ``8c''). These numbers are divided into ``fields'' with different meaning, see the example below.

        For PicoBlaze, opcodes are divided into either 3 bytes (in case of 18-bit opcode, unused high order bits are filled with zeros) or 2 bytes (in case of 16-bit opcode), these bytes are placed in the file in big-endian byte order.

        ~\\
        \indent\texttt{\colorbox{Goldenrod}{:}\colorbox{green}{0F}\colorbox{blue}{0000}\colorbox{Apricot}{00}\colorbox{GreenYellow}{E580F4F590E580F4F590E580F4F590}\colorbox{Lavender}{57}}\\
        \indent\texttt{\colorbox{Goldenrod}{:}\colorbox{green}{0F}\colorbox{blue}{000F}\colorbox{Apricot}{00}\colorbox{GreenYellow}{E580F4F590E580F4F590E580F4F590}\colorbox{Lavender}{48}}\\
        \indent\texttt{\colorbox{Goldenrod}{:}\colorbox{green}{0F}\colorbox{blue}{001E}\colorbox{Apricot}{00}\colorbox{GreenYellow}{E580F4F590E580F4F590E580F4F590}\colorbox{Lavender}{39}}\\
        \indent\texttt{\colorbox{Goldenrod}{:}\colorbox{green}{10}\colorbox{blue}{002D}\colorbox{Apricot}{00}\colorbox{GreenYellow}{E580F4F5907410B3758010B2907410B3}\colorbox{Lavender}{30}}\\
        \indent\texttt{\colorbox{Goldenrod}{:}\colorbox{green}{10}\colorbox{blue}{003D}\colorbox{Apricot}{00}\colorbox{GreenYellow}{758010B2902694052600940426940526}\colorbox{Lavender}{0A}}\\
        \indent\texttt{\colorbox{Goldenrod}{:}\colorbox{green}{10}\colorbox{blue}{004D}\colorbox{Apricot}{00}\colorbox{GreenYellow}{00940426009404269405E580F4F59026}\colorbox{Lavender}{8A}}\\
        \indent\texttt{\colorbox{Goldenrod}{:}\colorbox{green}{0B}\colorbox{blue}{005D}\colorbox{Apricot}{00}\colorbox{GreenYellow}{009404269405E580F4F590}\colorbox{Lavender}{63}}\\
        \indent\texttt{\colorbox{Goldenrod}{:}\colorbox{green}{00}\colorbox{blue}{0000}\colorbox{Apricot}{01}\colorbox{Lavender}{FF}}\\\\
        \indent\colorbox{Goldenrod}{\color{Goldenrod}X} Start code\\
        \indent\colorbox{green}{\color{green}X} Byte count\\
        \indent\colorbox{blue}{\color{blue}X} Address\\
        \indent\colorbox{Apricot}{\color{Apricot}X} Record type\\
        \indent\colorbox{GreenYellow}{\color{GreenYellow}X} Data\\
        \indent\colorbox{Lavender}{\color{Lavender}X} Checksum\footnote{Checksum is two's complement of 8-bit sum of entire record, except for the start code and the checksum itself.}

    \clearpage
    \subsection{S-Rec format}
        S-records are a form of simple ASCII encoding for binary data. An S-record file consists of a sequence of specially formatted ASCII character strings. An S-record will be less than or equal to 78 bytes in length. The order of S-records within a file is of no significance and no particular order may be assumed.

        For PicoBlaze, opcodes are divided into either 3 bytes (in case of 18-bit opcode, unused high order bits are filled with zeros) or 2 bytes (in case of 16-bit opcode), these bytes are placed in the file in big-endian byte order.

        \subsubsection{File format}
            The Motorola S-Rec file consists of a number of lines with following format:

            \begin{tabular}{|ccccc|}
                \hline
                <Type> & <Count> & <Address> & <Data> & <Checksum> \\\hline
            \end{tabular}

            \begin{description}
                \item[Type]~\\
                    A char[2] field. These characters describe the type of record (S0, S1, S2, S3, S5, S7, S8, or S9).

                \item[Count]~\\
                    A char[2] field. These characters when paired and interpreted as a hexadecimal value, display the count of remaining character pairs in the record.

                \item[Address]~\\
                    A char[4,6, or 8] field. These characters grouped and interpreted as a hexadecimal value, display the address at which the data field is to be loaded into memory. The length of the field depends on the number of bytes necessary to hold the address. A 2-byte address uses 4 characters, a 3-byte address uses 6 characters, and a 4-byte address uses 8 characters.

                \item[Data]~\\
                    A char [0-64] field. These characters when paired and interpreted as hexadecimal values represent the memory loadable data or descriptive information.

                \item[Checksum]~\\
                    A char[2] field. These characters when paired and interpreted as a hexadecimal value display the least significant byte of the ones complement of the sum of the byte values represented by the pairs of characters making up the count, the address, and the data fields.
            \end{description}

            Each record is terminated with a line feed. If any additional or different record terminator(s) or delay characters are needed during transmission to the target system it is the responsibility of the transmitting program to provide them.

        \subsubsection{Record types}
            \begin{description}
                \item[S0]~\\
                    The type of record is S0 (0x5330). The address field is unused and will be filled with zeros (0x0000). The header information within the data field is divided into the following subfields.

                    \begin{itemize}
                        \item mname is char[20] and is the module name.
                        \item ver is char[2] and is the version number.
                        \item rev is char[2] and is the revision number.
                        \item description is char[0-36] and is a text comment.
                    \end{itemize}

                    Each of the subfields is composed of ASCII bytes whose associated characters when paired, represent one byte hexadecimal values in the case of the version and revision numbers, or represent the hexadecimal values of the ASCII characters comprising the module name and description.

                \item[S1]~\\
                    The type of record field is S1 (0x5331). The address field is interpreted as a 2-byte address. The data field is composed of memory loadable data.

                \item[S2]~\\
                    The type of record field is S2 (0x5332). The address field is interpreted as a 3-byte address. The data field is composed of memory loadable data.

                \item[S3]~\\
                    The type of record field is S3 (0x5333). The address field is interpreted as a 4-byte address. The data field is composed of memory loadable data.

                \item[S5]~\\
                    The type of record field is S5 (0x5335). The address field is interpreted as a 2-byte value and contains the count of S1, S2, and S3 records previously transmitted. There is no data field.

                \item[S7]~\\
                    The type of record field is S7 (0x5337). The address field contains the starting execution address and is interpreted as a 4-byte address. There is no data field.

                \item[S8]~\\
                    The type of record field is S8 (0x5338). The address field contains the starting execution address and is interpreted as a 3-byte address. There is no data field.

                \item[S9]~\\
                    The type of record field is S9 (0x5339). The address field contains the starting execution address and is interpreted as a 2-byte address. There is no data field.
            \end{description}

            \paragraph{Example}
                ~\\
                \verb'S00600004844521B'\\
                \verb'S1130000285F245F2212226A000424290008237C2A'\\
                \verb'S11300100002000800082629001853812341001813'\\
                \verb'S113002041E900084E42234300182342000824A952'\\
                \verb'S107003000144ED492'\\
                \verb'S5030004F8'\\
                \verb'S9030000FC'

                ~\\This file consists of one S0 record, four S1 records, one S5 record and an S9 record.

            \paragraph{The S0 record is comprised as follows:}
                ~\\
                \begin{itemize}
                    \item S0 S-record type S0, indicating it is a header record.
                    \item 06 Hexadecimal 06 (decimal 6), indicating that six character pairs (or ASCII bytes) follow.
                    \item 00 00 Four character 2-byte address field, zeros in this example.
                    \item 48 44 52 ASCII H, D, and R - "HDR".
                    \item 1B The checksum.
                \end{itemize}

            \paragraph{The first S1 record is comprised as follows:}
                ~\\
                \begin{itemize}
                    \item S1 S-record type S1, indicating it is a data record to be loaded at a 2-byte address.
                    \item 13 Hexadecimal 13 (decimal 19), indicating that nineteen character pairs, representing a 2 byte address, 16 bytes of binary data, and a 1 byte checksum, follow.
                    \item 00 00 Four character 2-byte address field; hexadecimal address 0x0000 where the data which follows is to be loaded.
                    \item 28 5F 24 5F 22 12 22 6A 00 04 24 29 00 08 23 7C Sixteen character pairs representing the actual binary data.
                    \item 2A The checksum.
                \end{itemize}

            \paragraph{The second and third S1 records are comprised as follows:}
                ~\\
                The second and third S1 records each contain 0x13 (19) character pairs and are ended with checksums of 13 and 52, respectively. The fourth S1 record contains 07 character pairs and has a checksum of 92.

            \paragraph{The S5 record is comprised as follows:}
                ~\\
                \begin{itemize}
                    \item S5 S-record type S5, indicating it is a count record indicating the number of S1 records
                    \item 03 Hexadecimal 03 (decimal 3), indicating that three character pairs follow.
                    \item 00 04 Hexadecimal 0004 (decimal 4), indicating that there are four data records previous to this record.
                    \item F8 The checksum.
                \end{itemize}

            \paragraph{The S9 record is comprised as follows:}
                ~\\
                \begin{itemize}
                    \item S9 S-record type S9, indicating it is a termination record.
                    \item 03 Hexadecimal 03 (decimal 3), indicating that three character pairs follow.
                    \item 00 00 The address field, hexadecimal 0 (decimal 0) indicating the starting execution address.
                    \item FC The checksum.
                \end{itemize}

    \subsection{Code Listing}
        Code listing serves as an additional information about the assembled code and the progress of the assembly process. It contains information about all symbols defined in the code. Where and how they have been defined, what are their values and whether they are used in the code. Also detailed information about all macros defined in the code and/or expanded in the code. Conditional compilation configuration, instruction opcodes, address space reservations, inclusion of code from other files. And all warnings, errors, and notes generated during the assembly by the assembler. There are assembler directives which alters formatting of the code listing file.

        \subsubsection{A simple code listing.}
            ~\\
            \verb'                         1     ; Comment.'\\
            \verb'                         2             org     0'\\
            \verb'                         3'\\
            \verb'0000 01001               4     main:   load    s0, #1'\\
            \verb'0001 01102               5             load    s1, #2'\\
            \verb'0005 01203               6             load    s2, #3'\\
            \verb'0007 3E000               7             jump    main'\\
            \verb'                         8'\\
            \verb'                         9             end'\\

        Code listing contains entire source code which has been assembled but with each line prefixed with line number and some additional information which will be explained later. Each line of the code listing which contains an original source code line may contain besides line number also some additional information regarding the compilation of the given line of code. Such a additional information might look like this and is composed of these parts:

        \subsubsection{Format of code listing}
            ~\\
            \verb'  '{\color{highlight_lst_number}\verb'00055'}\verb'                  '{\color{highlight_lst_line}\verb'1'}\verb'      '{\color{highlight_comment}\verb'X           EQU         0x55'}\\
            \verb'                         '{\color{highlight_lst_line}\verb'2'}\verb'      '{\color{highlight_comment}\verb'            INCLUDE     "file.asm"'}\\
            {\color{highlight_lst_address}\verb'0007'}\verb' '{\color{highlight_lst_code}\verb'00100'}\verb'      '{\color{highlight_label}\verb'=1'}\verb'       '{\color{highlight_lst_line}\verb'3'}\verb' '{\color{highlight_constant}\verb'+1'}\verb'   '{\color{highlight_comment}\verb'label:      LOAD        S1, S0'}\\
            ~\\
            \colorbox{highlight_lst_number}{\color{highlight_lst_number}X} Expression value\\
            \colorbox{highlight_lst_address}{\color{highlight_lst_address}X} Address in program memory\\
            \colorbox{highlight_lst_code}{\color{highlight_lst_code}X} Machine code\\
            \colorbox{highlight_label}{\color{highlight_label}X} Level of file inclusion\\
            \colorbox{highlight_lst_line}{\color{highlight_lst_line}X} Line number\\
            \colorbox{highlight_constant}{\color{highlight_constant}X} Level of macro expansion\\
            \colorbox{highlight_comment}{\color{highlight_comment}X} Original line

        \clearpage
        \subsubsection{A more complex example of code listing}
            ~\\
            {\color{highlight_lst_number}\verb' 0001C'}{\color{highlight_lst_line}\verb'               1'}\verb'     '{\color{highlight_constant}\verb'abc'}\verb'     '{\color{highlight_directive}\verb'equ'}\verb'     '{\color{highlight_symbol}\verb'('}\verb' '{\color{highlight_unknown_base}\verb'14'}\verb' '{\color{highlight_symbol}\verb'*'}\verb' '{\color{highlight_unknown_base}\verb'2'}\verb' '{\color{highlight_symbol}\verb')'}\verb'      '{\color{highlight_comment}\verb'; Define symbol abc.'}\\
            {\color{highlight_lst_line}\verb'                     2'}\verb'             '{\color{highlight_directive}\verb'org'}\verb'     '{\color{highlight_unknown_base}\verb'0'}\verb'               '{\color{highlight_comment}\verb'; Code at address 0.'}\\
            {\color{highlight_lst_line}\verb'                     3'}\\
            {\color{highlight_lst_include}\verb'             =1'}{\color{highlight_lst_line}\verb'      4'}\verb'             '{\color{highlight_directive}\verb'include'}\verb' '{\color{highlight_string}\verb''\verb"'"\verb'macros.asm'\verb"'"\verb' '{\color{highlight_comment}\verb'; Include macros.asm'}}\\
            {\color{highlight_lst_include}\verb'             =1'}{\color{highlight_lst_line}\verb'      5'}\verb'     '{\color{highlight_comment}\verb'; This is the beginning of file macros.asm.'}\\
            {\color{highlight_lst_include}\verb'             =1'}{\color{highlight_lst_line}\verb'      6'}\verb'     '{\color{highlight_macro}\verb'xyz'}\verb'     '{\color{highlight_directive}\verb'macro'}\verb'    '{\color{highlight_constant}\verb'arg'}\\
            {\color{highlight_lst_include}\verb'             =1'}{\color{highlight_lst_line}\verb'      7'}\verb'             '{\color{highlight_instruction}\verb'load'}\verb'     '{\color{highlight_sfr}\verb's1'}{\color{highlight_oper_sep}\verb','}\verb' '{\color{highlight_constant}\verb'arg'}\\
            {\color{highlight_lst_include}\verb'             =1'}{\color{highlight_lst_line}\verb'      8'}\verb'             '{\color{highlight_instruction}\verb'nop'}\\
            {\color{highlight_lst_include}\verb'             =1'}{\color{highlight_lst_line}\verb'      9'}\verb'             '{\color{highlight_instruction}\verb'load'}\verb'     '{\color{highlight_constant}\verb'arg'}{\color{highlight_oper_sep}\verb','}\verb' '{\color{highlight_sfr}\verb's1'}\\
            {\color{highlight_lst_include}\verb'             =1'}{\color{highlight_lst_line}\verb'     10'}\verb'     '{\color{highlight_directive}\verb'endm'}\\
            {\color{highlight_lst_include}\verb'             =1'}{\color{highlight_lst_line}\verb'     11'}\verb'     '{\color{highlight_comment}\verb'; This is the end of file macros.asm.'}\\
            {\color{highlight_lst_line}\verb'                    12'}\\
            {\color{highlight_lst_line}\verb'                    13'}\verb'     '{\color{highlight_label}\verb'main:'}\verb'   '{\color{highlight_macro}\verb'xyz'}\verb'     '{\color{highlight_sfr}\verb's0'}\verb'          '{\color{highlight_comment}\verb'; Expand macro xyz here.'}\\
            {\color{highlight_lst_address}\verb'0000'}{\color{highlight_lst_code}\verb' 00100'}{\color{highlight_lst_line}\verb'          14'}{\color{highlight_lst_macro}\verb' +1'}\verb'                  '{\color{highlight_instruction}\verb'load'}\verb'     '{\color{highlight_sfr}\verb's1'}{\color{highlight_oper_sep}\verb','}\verb' '{\color{highlight_sfr}\verb's0'}\\
            {\color{highlight_lst_address}\verb'0002'}{\color{highlight_lst_code}\verb' 00000'}{\color{highlight_lst_line}\verb'          15'}{\color{highlight_lst_macro}\verb' +1'}\verb'                  '{\color{highlight_instruction}\verb'nop'}\\
            {\color{highlight_lst_address}\verb'0003'}{\color{highlight_lst_code}\verb' 00010'}{\color{highlight_lst_line}\verb'          16'}{\color{highlight_lst_macro}\verb' +1'}\verb'                  '{\color{highlight_instruction}\verb'load'}\verb'     '{\color{highlight_sfr}\verb's0'}{\color{highlight_oper_sep}\verb','}\verb' '{\color{highlight_sfr}\verb's1'}\\
            {\color{highlight_lst_address}\verb'0005'}{\color{highlight_lst_code}\verb' 22000'}{\color{highlight_lst_line}\verb'          17'}\verb'             '{\color{highlight_instruction}\verb'jump'}\verb'    '{\color{highlight_constant}\verb'main'}\verb'        '{\color{highlight_comment}\verb'; Jump back to main:.'}\\
            {\color{highlight_lst_line}\verb'                    18'}\verb'             '{\color{highlight_directive}\verb'end'}\verb'                 '{\color{highlight_comment}\verb'; End of assembly.'}

\clearpage
\section{Assembler messages}
    This chapter lists the messages generated by the MDS assembler. The following sections include a brief description of the possible error and warning messages along with a description of the error or warning and any corrective actions you can take to avoid or eliminate the error or warning. Errors terminate the assembly and generate a message that is displayed on the console. Warnings generate a message that is displayed on the console but do not terminate the assembly. All messages are recorded in the code listing file.

    \begin{description}
        \item[Unable to open file: X]~\\
            The given file (X) cannot be opened, probably does not exist or your operating system refuses to grad the access to it. Check whether the file exists, and check your permissions.
        \item[Unable to write to file: X]~\\
            It is not possible to write into the given file (X). This is in most cases caused by wrong permissions set on the file or directory, or nonexistent directory in the file path.
        \item[Unable to save file: X]~\\
            The given file (X) cannot be saved. This might indicate that there is something badly wrong, like not enough space left on your storage device (HDD, etc.).
        \item[The resulting machine code is too big to be stored in a file]~\\
            Size of the resulting machine code is bigger than your processor could handle in its current configuration.
        \item[Some of the source code files were apparently changed during compilation]~\\
            Please do not change source files during the compilation, wait for the compilation finish first.
        \item[User defined memory limit for X memory exceeded]~\\
            You have exceeded boundary of some memory space (X) defined by the \texttt{LIMIT} directive.
        \item[Instruction X requires operand \#Y to be of type(s) Z ...]~\\
            This means for example that you are trying to use a symbol defined as port address in place of register address. ``\#Y'' stands for operand number, starting from 1.
        \item[Macro not defined: X]~\\
            The macro you are attempting to use (X) has not (yet) been defined, possibly you are trying to expand a macro before its definition.
        \item[Too many arguments given, expecting at most X arguments]~\\
            Directives require certain number of arguments, when you provide a different number of arguments, the directive makes no sense to the assembler.
        \item[Attempting to use unavailable space in X memory at address: Y]~\\
            Suppose you have 8 registers and you try for example to write to register at address 10, then the assembler will give you this error. The same, of course, apply also to program memory, or any other memory space.
        \item[The last error was critical, compilation aborted]~\\
            Normally the assembler tries to carry on compiling for as long as possible despite errors, it is implemented that way to provide you with as many error message at the time as possible. But in certain cases compilation has to aborted instantaneously.
        \item[Device not supported: X]~\\
            The given processor (X) is not supported, this usually happens due to some typo.
        \item[Device specification code is already loaded ]~\\
            The processor architecture has to be specified only once, multiple specifications for one source code would not make sense, they could "collide" with each other.
        \item[Limit value X is not valid]~\\
            Invalid memory size limit for the \texttt{LIMIT} directive, only -1, 0, and positive integers are valid.
        \item[Directive `LOCAL' cannot appear outside of macro definition ]~\\
            Directive \texttt{LOCAL} can be only used inside macro definition, outside macro definition it has no meaning and therefore cannot be used.
        \item[Directive `EXITM' cannot appear outside macro definition]~\\
            Directive \texttt{EXITM} can be only used inside macro definition, outside macro definition it has no meaning and therefore cannot be used.
        \item[Maximum macro expansion depth (X) reached]~\\
            Macros can be expanded in other macros, and they can be expanded in other macros, and so on. To avoid infinite loops or other kinds of undesirable behavior related to macro expansion occurring in another macro expansion, the assembler limits maximum depth of macro expansion. You can change your compiler settings and allow higher expansion depth. By default, maximum macro expansion depth is set to 1024.
        \item[Maximum file inclusion depth (X) reached]~\\
            A source file may include another source file(s) using the \texttt{INCLUDE} directive, and these files can in the same way include other source files, and so on. Although cycles in the inclusion tree (infinite loop of inclusion) are checked for by the assembler, and theoretically cannot possibly happen, the assembler limits maximum inclusion depth by default to 1024 for stability reasons. You can change your compiler settings and allow higher inclusion depth.
        \item[Maximum number of \#WHILE directive iterations (X) reached]~\\
            When using the \texttt{\#WHILE} directive, it is easy, by mistake, to make it an enormously long loop, or even an infinite loop; in order to prevent that from happening, the assembler limits maximum number of the \#while loop iterations to 1024.
        \item[Maximum number of REPEAT directive iterations (X) reached]~\\
            When using the \texttt{REPEAT} directive, it is easy, by mistake, to make it an enormously long loop, or even an infinite loop; in order to prevent that from happening, the assembler limits maximum number of the repeat loop iterations to 1024.
        \item[Instruction word is only X bits wide, value Y trimmed to Z]~\\
            You have exceeded instruction word length, opcode value has been trimmed from left to X bits which reduces value Y to value Z.
        \item[Symbol already defined: X]~\\
            The given symbol (X) has already been defined, there cannot coexist two or more symbols with the same name in one compilation unit.
        \item[Symbol not defined: X]~\\
            The given symbol has not been defined, maybe you are trying to use it prior to its definition.
        \item[Symbol X has been already defined with type: Y]~\\
            The given symbol has been already defined with type Y, you are trying to define two or more symbols with the same name but with different types. This is not allowed in this assembler, such practice often tends to lower software quality.
        \item[Cannot remove predefined symbol: X]~\\
            Predefine symbols like \_\_MDS\_VERSION\_\_, \_\_DATE\_\_, \_\_TIME\_\_, \_\_FILE\_\_, and \_\_LINE\_\_ cannot be redefined or deleted.
        \item[Cannot change value of predefined symbol: X]~\\
            Predefine symbols like \_\_MDS\_VERSION\_\_, \_\_DATE\_\_, \_\_TIME\_\_, \_\_FILE\_\_, and \_\_LINE\_\_ cannot be redefined or deleted.
        \item[Undefined value]~\\
            You are attempting to use some value whose numerical representation is unknown to the assembler.
        \item[Real numbers are not supported in this assembler]~\\
            Real numbers (1.8, 22.65, etc.) are not supported by this assembler.
        \item[Undefined symbol: X]~\\
            Symbol X has been undefined, i.e. deleted from the symbol table using the \texttt{UNDEFINE} directive and therefore can be no longer used.
        \item[This value is not valid inside of expression ]~\\
            Expressions in this assembler can be calculated only from integer values; other values like strings, etc. are not allowed.
        \item[Division by zero]~\\
            During evaluation of an expression, the assembler encountered division by zero. Division by zero results in value not representable by this assembler therefore is reported as error. Examples of such case might be: ``\texttt{LOAD S0, \#10 / 0}'', or ``\texttt{A  EQU  ( 2 * A / B ) ; where B = 0}''.
        \item[Unable to resolve this expression]~\\
            The given expression cannot be resolved, please check if the expression is syntactically correct.
        \item[Syntax not understood]~\\
            There is a syntax error in your source code; in this case, the error is completely ununderstandable for the assembler.
        \item[Character constant has to have 8 bits]~\\
            Character constants are supposed to be only one letter long.
        \item[Unterminated string or character constant]~\\
            Strings and character constants start with ``\texttt{"}'' respectively ``\texttt{'}'', and has to end with the same character, i.e. ``\texttt{"}'' respectively ``\texttt{'}''. In this case you have apparently left a string or character constant unterminated by the appropriate character.
        \item[Unrecognized escape sequence: X]~\\
            Escape sequence X was not understood by the assembler, please check the table of escape sequences for reference (section \ref{Escape sequences}).
        \item[No file name specified]~\\
            You probably forgot to specify name of file for the \texttt{INCLUDE} directive.
        \item[Unable to open the specified file: X]~\\
            The specified file (X) cannot be opened, probably does not exists or your operating system refuses to grad you access to it. Please check whether the file exists, and check your permissions.
        \item[Unrecognized token: X]~\\
            Lexical analyzer was unable to recognize this token (X). Normally this does not happen; but if your code contains some binary values with no printable representation or something like that, this may happen.
        \item[Maximum number of messages reached, suppressing compiler message...]~\\
            Source code contains a huge number of errors, further error messages are from now on suppressed in order to prevent enormous size of code listing file, and enormous and impractical assembler console output.
        \item[Macro expansion has been disabled, macro X will not be expanded]~\\
            The given macro (X) cannot be expanded because macro expansion has been temporarily disabled by \texttt{NOEXPAND} directive. You can use \texttt{EXPAND} directive to re-enable the macro expansion.
        \item[Parameter X substituted for blank value]~\\
            Macro parameters are optional, those parameters which has not been substituted with arguments are filled with blank values. Parameters can be checked if they are ``blanks'' using the \texttt{\#IFB} and \texttt{\#IFNB} directives.
        \item[Symbol X already declared as local]~\\
            You are trying to declare a symbol as local macro symbol, which itself is perfectly valid operation unless you do it multiple times for the same symbol.
        \item[Reusing already reserved space in X memory at address: Y]~\\
            There is a collision in two different attempts to utilize certain location in memory space. For instance with the \texttt{ORG} directive you can easily, by mistake, attempt to put two instruction at the same location in the program memory. Or using the \texttt{REG} directive you can also, by mistake, use the same register for two completely different things which could result in "mysterious" behavior of your program.
        \item[Limit value -1 means unlimited]~\\
            Using -1 as size argument for the \texttt{LIMIT} directive removes the limit set by that directive, there is nothing bad about that but the assembler notifies you about it just to make sure you know what you are doing.
        \item[Symbol X declared as local but never used, declaration ignored]~\\
            What is the point of having a local symbol inside a macro and do not use it? This is not an error but its possible side effect from some error or mistake, that is why the compiler warns you about it.
        \item[Comparing two immediate constants, result is always X]~\\
            When using autogenerated run-time conditions, it does not make sense to write a condition which result is always known in advance, that is why you get this warning message. Please check if you are using correct addressing modes.
        \item[Sign overflow. Result is negative number lower than the lowest negative...]~\\
            Two's complement signed arithmetic overflow occurred during expression evaluation.
        \item[File name contains a null character]~\\
            This is a very unlikely error message, it means that the provided file name contains an invalid character, in this case it is the NULL character.
        \item[Result is negative number X, this will be represented as X-bit number...]~\\
            Result is negative number, in two's complement signed arithmetic that means that the number contains binary ones in high order bits; after the value is trimmed to fit binary width required for the specific purpose, it might result in a different value (even a positive number).
        \item[Value out of range, allowed range is {[X,Y]} (trimmed to Z bits) which makes it N]~\\
            Value exceeds the required range for the specific purpose.
        \item[Architecture not supported for the selected language]~\\
            The specified processor architecture is not supported by this compiler for the selected programming language, this cannot normally happen.
        \item[Programming language not supported]~\\
            The specified programming language is not supported by this compiler, this does not normally happen.
        \item[Failure: X]~\\
            This message indicates some kind of low-level failure, something wrong with your operating system, etc. This does not normally happen.
        \item[Programming language not specified]~\\
            Since this compiler is implemented in a way that it can be extended with support for additional programming languages via compiler modules, programming language has to be specified as a compilation option.
        \item[Target architecture not specified]~\\
            Since this compiler is implemented in a way that it can be extended with support for additional target processor architectures via compiler modules, target architecture has to be specified as a compilation option.
        \item[Source code file not specified]~\\
            You are trying to compile something but you did not provide any input file name.
        \item[Empty string used as source code file name]~\\
            If you are running the compiler from a script, this probably indicates an error in that script; otherwise, it should not normally happen.
        \item[File not found: X]~\\
            Some of the specified source files does not exist.
        \item[Invalid unicode character: X]~\\
            Invalid value used for Unicode escape sequence, please check the Unicode table for reference.
        \item[Too big number: X]~\\
            Some numeric literal in your source code represents a value too high to be usable in the compiler's internal logic.
        \item[Invalid token: X]~\\
            Compiler's lexical analyzer encountered a token which it cannot use.
        \item[Identifier cannot start with a digit: X]~\\
            Assembler encountered a token which it cannot recognize; in this case it might be a numeric literal, or an identifier. For example: ``0abc'' might be meant as hexadecimal number but correct syntax is with radix specifier, i.e.: 0x0abc or 0abcH; or it may be identifier (like a label or so) but in that case it cannot start with a digit.
        \item[Too many arguments given to macro X, expecting at most Y argument(s)]~\\
            You have defined some macro with parameters (X), this macro is defined with certain number (Y) of parameters. When you expand this macro with lower number of arguments than is the number of parameters, the remaining parameters will be filled with ``blank'' values; but when you provide higher number of arguments, the compiler will not know what to do with them.
        \item[String already defined: X]~\\
            You are attempting to override, already defined character string (X). Character strings are defined with the \texttt{STRING} directive.
        \item[Blank value]~\\
            Blank value used in expression; since blank values are not numbers, they cannot be use for calculations. Blank values are automatically generated by the assembler when some of macro parameters was not substituted with the corresponding argument.
        \item[This value is not valid inside an expression]~\\
            This indicates an attempt to use nonnumerical value (like a string) in an expression. Since such value is not a number, it cannot be use for calculation.
        \item[Invalid number of operands, instruction X takes Y operand(s)]~\\
            Each instruction takes an exact number of operands; for example \texttt{LOAD} always takes two operands, it cannot be used with only one operand, or with three, four, etc. operands.
        \item[Invalid number of operands, instruction X takes Y or Z operand(s)]~\\
            Each instruction takes an exact number of operands; for example \texttt{LOAD} always takes two operands, it cannot be used with only one operand, or with three, four, etc. operands.
        \item[Cannot declare a label before X directive]~\\
            For various reasons with certain directives it is not allowed to define a label at the same line.
        \item[Directive X requires a single argument]~\\
            This means that directive X cannot be used with different number of arguments than one.
        \item[Directive X takes no arguments]~\\
            This means that directive X cannot be used with different number of arguments than zero.
        \item[Directive X requires an identifier for the symbol (or macro) which it defines]~\\
            Symbol, string, and macro definition directives always require name of the symbol, string, or macro which they are suppose to define.
        \item[Directive X requires `AT' operator before the start address]~\\
            Missing ``\texttt{AT}'' operator, please refer to the directive syntax for details.
        \item[Comma (`,') expected between operands]~\\
            There is a syntax error in your code, and the assembler guesses that it might be caused by missing comma between instruction operands.
        \item[Instruction not supported on the this device: X]~\\
            Various processors have various instruction sets, this error message indicates that you are trying to use some instruction (X) which is not supported by the processor you are compiling your code for.
        \item[No user defined limit for the program memory size, fail jump cannot be used]~\\
            When using the \texttt{FAILJMP} directive (or similar directive), there has to be defined exact size of the program memory. Otherwise, the assembler cannot know up to which address it is supposed to fill the program memory with \texttt{JUMP}s.
        \item[Invalid size: X]~\\
            Size cannot be zero or negative number.
        \item[Cannot merge DATA memory (SPR) with the CODE memory at address: X]~\\
            When using the \texttt{MERGESPR} directive, the merge address has to be valid. Zero and negative numbers are not valid: zero would collide with the reset address, and negative number does not make sense at all.
        \item[Address is crossing CODE memory size boundary: Y]~\\
            When using the \texttt{MERGESPR} directive, the merge address has to be within boundaries of the progra memory space.
        \item[Directive `EXITM' cannot appear inside special macro...]~\\
            Directive \texttt{EXITM} cannot appear inside special macro because it would break its pairing rules. Special macros include directives: \texttt{IF}, \texttt{ELSEIF}, \texttt{ELSE}, \texttt{FOR}, and \texttt{WHILE}. Term pairing rules refers to pairing of \texttt{IF--ENDIF}, \texttt{FOR--ENDFOR}, and \texttt{WHILE--ENDWHILE}.
        \item[Only one fail jump may be specified in the code]~\\
            When using the \texttt{FAILJMP} directive (or similar directive), it may be used only once in the compilation unit. Multiple usage of such directive would result either in mutual overrides or in collisions, for that reason it is not allowed.
        \item[String X not defined]~\\
            You are attempting to use a string (X) which has not been defined in the table strings using the \texttt{STRING} directive. Probably this is just a typo, please check if you are using the correct syntax for the \texttt{OUTPUTK} or \texttt{LOAD \& RETURN} instruction.
        \item[Assembler feature X is supported only on X and higher]~\\
            Certain special assembler features like autogenerated run-time conditions, etc. are supported only on certain processors because they dependent on specific instructions.
        \item[Attempting to override string: X]~\\
            A string defined using the \texttt{STRING} directive, would be overridden by symbol. This action is not allowed.
        \item[Attempting to override symbol: X]~\\
            A symbol defined using the \texttt{EQU}, \texttt{SET}, etc. directive, would be overridden by string (\texttt{STRING} directive). This action is not allowed.
        \item[Redefinition of macro X, original definition is at: Y]~\\
            This indicates that the macro X originally defined at location Y, will be overridden by this definition, expansions of this macro from this line on will use the new definition, while the preceding expansions will not be affected.
        \item[Macro parameter X eclipses global symbol Y, defined at: Z]~\\
            This message means that your macro uses a parameter with the same name as some of the already defined symbols. There is nothing wrong with that but it is generally better to avoid it.
        \item[Macro parameter X eclipses global symbol Y]~\\
            This message means that your macro uses a parameter with the same name as some of the already defined symbols. There is nothing wrong with that but it is generally better to avoid it.
        \item[Argument \#X not used]~\\
            Expression defined using the \texttt{DEFINE} directive contains parameter which is not used during the expression evaluation.
        \item[Exact device not specified, using X by default]~\\
            If you do not use the \texttt{DEVICE} directive to specify the exact processor, assembler will use something (X) by default.
        \item[Processor type (X) should be specified without double quotes...]~\\
            The \texttt{DEVICE} takes processor name as it argument, the processor name should not be enclosed by ``\texttt{"}'' (double quote) characters.
        \item[Reuse of iterator register in nested for loop]~\\
            The two loops will affect each other via their iterator registers.
        \item[Comparing a register with itself, result is always X]~\\
            Result your auto-generated run-time condition is always known in advance because it compares the given register with the same register.
        \item[Register is expected here but given type is: X]~\\
            Only register addresses are recommended for the \texttt{FOR} loops iterators.
        \item[Generic number is expected here but given type is: X]~\\
            Only generic numbers are recommended for specifying the \texttt{FOR} loops intervals.
        \item[No target file for SPR initialization specified]~\\
            When you wish the compiler to perform scratch-pad memory initialization, you either need to specify target file for the initialization, or specify merge address in the program memory.
        \item[Jump address is not specified by a label]~\\
            When using the \texttt{FAILJMP} directive, argument for this directive is recommended to be a ``label''.
        \item[Instruction word is only X bits wide, value Y trimmed to Z]~\\
            When using the \texttt{DB} directive to directly initialize the program memory, values exceeding bit width of the instruction opcode on your processor are automatically trimmed from left to fit the opcode width.
        \item[Unable to locate file X in base path Y, or include path(s): Z]~\\
            This indicates that one of your include files (i.e. files included into your source code using the \texttt{INCLUDE} directive) cannot be located in the directories which you specified to search for included files (include directories).
        \item[File X is already opened, you might have an inclusion loop in your code]~\\
            This usually happens when you include a file from another file which includes the first file, or when a file includes itself.
        \item[I/O error, cannot read the source file properly]~\\
            Some of the source files is not readable, this might be caused by permissions set on that file.
    \end{description}
