\section{Main simulator panel}
    This panel is the main part of the simulator user interface. Simulator mimics functionality of PicoBlaze instruction set and provide detailed view in its registers, overview of input and output ports, etc., and displays warning when the simulated program behaves strangely i.e. does something which under normal circumstances results an error. Following picture shows main window of the simulator. You can find here internal registers, scratch-pad ram, input and output ports, stack, program counter, elapsed time and cycles, current clock and internal flags: carry and zero. All those features can be edited during processor simulation. If some value changes state, it will turn yellow. You can drag out whole simulation panel and place outside of the main window, for example another display on your computer.

   \begin{figure}[h!]
        \centering
        \includegraphics[width=\textwidth]{img/bottom_panel.png}
        \caption{Main simulator panel}
    \end{figure}

    \begin{description}
        \item [Registers]~\\
            You can find here values of internal processor registers. The first column represents name, second value in decimal, third hexadecimal value, and last binary value.
        \item [Scratch-pad RAM]~\\
            This hexadecimal editor represents values stored in the scratch-pad RAM. Address of a particular field is addition of the line and column numbers.
        \item [Ports]~\\
            Output and input ports are represented by hexadecimal editors in their full address range. Besides viewing and changing input and output values, you can also see the status of PicoBlaze RD, WR, and WRK signals (read and write strobes).
        \item [Stack]~\\
            Stack window represents values stored in stack. You can change the stack content by pushing and popping values, and view its virtual stack pointer.
        \item [Info panel]~\\
            Panel on the right side of simulation panel gives you general information about simulated code. You can find there the program counter, number of executed cycles, clock frequency, elapsed time and status flags Carry and Zero, plus normally hidden Interrupt Enable flag. You can also invoke an interrupt by clicking on the Interrupt button. Buttons showing status flags or other flags can also change them (click on such button to invert the flag). Green text in ``flag button'' indicates that the flag is set to 1, black text indicates that value of the flag is 0.
    \end{description}

\section{Modes of simulation}
    The simulator can operate three different modes:
    \begin{description}
        \item [Step]
            \index{Step mode}
            Step by step simulation, executes only one instruction when invoked, no matter how many machine cycles it will take (this does not apply for macro-instruction, in that case each instruction of the macro is executed separately).
        \item [Animation]
            \index{Animation mode}
            Similar to the step mode but in a loop, instructions are executed one after another until stopped on user request, or by a simulator waring message.
        \item [Run]
            \index{Run mode}
            Run is generally the same as animation but much faster, GUI is not updated so often, and changes in the registers, ports, etc. are not shown until the run is halted.
    \end{description}

\section{Simulation cursor}
    Simulation cursor shows what instruction will be executed in next step. Two previously executed instructions are highlighted but with lower and fading intensity; you can use that to easier track previous steps. Following figure shows actual cursor on line 8, one step before is on line 5, and two steps before is on 4.

    \begin{figure}[h!]
        \centering
        \includegraphics[width=\textwidth]{img/simulationcursor1.png}
        \caption{Simulation cursor}
    \end{figure}

    \clearpage

    When simulator encounters macro expansion, for better orientation, cursor will appear on two or more locations. On macro expansion and in macro definition, showing what is being executed and where it came from. When you have macro expanded in another macro, you will see three cursors and so on. For better understanding, the following figure shows simulation cursor behavior with macros.

    \begin{figure}[h!]
        \centering
        \includegraphics[width=\textwidth]{img/simulationcursor2.png}
        \caption{Simulation cursor (with macros)}
    \end{figure}


\section{Simulator messages}
    The simulator can display several warnings to handle some common mistakes in software development like program counter overflow, stack overflow, and some others; it is meant to make your debugging process less painful. To adjust settings related simulator warnings, please go to [Main~Menu]~-> [Interface]~-> [Configure]~-> [Simulator~warnings].

    \subsection{Memory group}
        The memory group relates to suspicious and invalid attempts to access program memory, register file, scratch-pad memory, and even ports of the simulated processor.
        \begin{description}
            \item[Write to nonexistent memory location]~\\ % ERR_MEM_WR_NONEXISTENT
                Attempt to write at a memory location exceeding the size of that memory.
            \item[Read from nonexistent memory location]~\\ % ERR_MEM_RD_NONEXISTENT
                Attempt to read from a memory location exceeding the size of that memory.
            \item[Write to unimplemented memory location]~\\ % ERR_MEM_WR_NOT_IMPLEMENTED
                Attempt to write at location in memory which has been disabled. For instance you disable scratch-pad memory by setting its size to 0, and then you attempt to write in there.
            \item[Read from unimplemented memory location]~\\ % ERR_MEM_RD_NOT_IMPLEMENTED
                Attempt to read from location in memory which has been disabled. For instance you disable scratch-pad memory by setting its size to 0, and then you attempt to read from there.
        \end{description}

    \subsection{Stack group}
        The stack group is related to processor call stack.
        \begin{description}
            \item[Stack overflow]~\\ % ERR_STACK_OVERFLOW
                This happens when the program exceed the stack capacity. It might happen when the call stack capacity is insufficient to handle so many subroutine calls as your program requires, it may also easily happen when the program uses recursion, or when there are nonsense calls the program.
            \item[Stack underflow]~\\ % ERR_STACK_UNDERFLOW
                This happens when there is a request to pop value from call stack but the stack is already empty.
        \end{description}

    \subsection{CPU group}
        The CPU group contains warnings related control flow and interrupt handling.
        \begin{description}
            \item[Program counter overflow]~\\ % ERR_CPU_PC_OVERFLOW
                The program counter has exceeded its maximum value, this is usually a serious error in the simulated program.
%             \item[Program counter underflow]~\\ % ERR_CPU_PC_UNDERFLOW
%                 The program counter has exceeded its minimum value (which is zero). This happens due to invalid relative jumps, etc.
            \item[System fatal error]~\\ % ERR_CPU_SYS_FATAL
                This indicates an error in the simulator itself, normally this never happens (the simulator is tested by vast series of automated tests); but if that happens, please report this as a bug in the MDS.
            \item[Invalid opcode]~\\ % ERR_CPU_INVALID_OPCODE
                Indicates that the simulator encountered instruction opcode which is cannot decode and execute, this is very unlikely to happen but if you use the \texttt{DB} directive to directly write opcodes in the program memory, this might happen.
%             \item[Invalid jump]~\\ % ERR_CPU_INVALID_JUMP
%             \item[Invalid call]~\\ % ERR_CPU_INVALID_CALL
            \item[Invalid IRQ]~\\ % ERR_CPU_INVALID_IRQ
                Invalid interrupt request. For instance you are trying to interrupt an interrupt service routine.
            \item[Invalid return]~\\ % ERR_CPU_INVALID_RET
                Attempt to return from subroutine while there is no subroutine to return from.
            \item[Invalid return from interrupt]~\\ % ERR_CPU_INVALID_RETI
                Attempt to return from ISR (interrupt service routine) while there is no ISR in progress.
%             \item[Invalid opset]~\\ % ERR_CPU_INVALID_OPSET
%             \item[Unsupported instruction]~\\ % ERR_CPU_UNSUPPORTED_INST
%             \item[Instruction ignored]~\\ % ERR_CPU_INST_IGNORED
        \end{description}
