\section{Short introduction}
    The MCU simulator is a tool designed to mimics behavior of a real MCU as much as possible. But it can have certain limitations, the most expectable limitation is of course the speed of simulation. This simulator is very slow, but ofers some extra features.

\section{Modes of simulation}
    There are 4 modes of simulation:
    \begin{itemize}
        \item STEP: Execute exactly one intruction, no matter how many machine cycles it will take. This does not apply for macro-instruction, in that case each instruction of the macro is executed separately.
        \item STEP OVER: Execute as many instructions as possible until simulator cursor changes its location from one line of source code to another.
        \item ANIMATE: Do the same as step, but in a loop, one after another until stopped by a waring condition or user request.
        \item RUN: This is generally the same as animate, but much faster, because GUI is not updated so often as in the animate mode.
        \item STEP BACK: Take back the last performed step. There is limited number of step which can be taken back.
    \end{itemize}
