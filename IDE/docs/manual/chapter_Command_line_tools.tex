Assembler, Disassembler, and Assembler Translator may also be invoked from command line and therefore used in your scripts. On Linux versions there is also man page for each one of them.

\section{Assembler}
    \subsection{Description}
        Macro-assembler for PicoBlaze soft-core processors, this tool takes one or more source code files and produces compiled machine code file usable in JTAG loaders, processors simulators, and similar tools; along with these files containing compiled machine code compiler also produces extensive debugging output. MDS macro-assembler is made to run fast and extensively tested for greater reliability, it feature set includes various special macros and user defined macros support making it one of the world most advanced assemblers available on the market today for PicoBlaze processors.

        By default the compiler does not generate any output files, that might be useful when you simply want to check a file for errors but you do not want that when you actually need to compile something a use the resulting machine code. When you need the machine code or any other file output, you have to specify which file or files you want the compiler to generate by providing the corresponding options.

    \paragraph{Usage}~\\
        \verb'mds-compiler  <OPTIONS>  [ -- ] [ source-file ...  ]'

    \paragraph{Options}~\\
        \begin{description}
            \item[-{}-architecture, -a <architecture>]~\\
                \textbf{(REQUIRED)} Specify target architecture, supported architectures are:
                \begin{description}
                    \item[PicoBlaze] KCPSM soft-core processor
                \end{description}

            \item[-{}-language, -l <programming language>]~\\
                \textbf{(REQUIRED)} Specify programming language, supported languages are:
                \begin{description}
                    \item[asm] Assembly language
                \end{description}

            \item[-{}-hex, -x <intel HEX-file>]~\\
                Specify output file with machine code generated as a result of compilation, data will be stored in Intel 8 Hex format.

            \item[-{}-debug, -g <MDS-native-debug-file>]~\\
                Specify output file with code for MCU simulator and other debugging tools.

            \item[-{}-srec <Motorola-S-REC-file>]~\\
                Specify output file with machine code generated as a result of compilation, data will be stored in Motorola S-record format.

            \item[-{}-binary <binary-file>]~\\
                Specify output file with machine code generated as a result of compilation, data will be stored in raw binary format.

            \item[-{}-lst <code-listing>]~\\
                Specify output file where code listing generated during compilation will be stored.

            \item[-{}-mtable, -m <table-of-macros>]~\\
                Specify file in which the compiler will put table of macros defined in your code.

            \item[-{}-stable, -s <table-of-symbols>]~\\
                Specify file in which the compiler will put table of symbols defined in your code.

            \item[-{}-strtable <table of symbols>]~\\
                Specify file in which the compiler will put table of strings defined in your code.

            \item[-{}-help, -h]~\\
                Print help message.

            \item[-{}-version, -V]~\\
                Print compiler version and exit.

            \item[-{}-check, -c]~\\
                Do not perform the actual compilation, do only lexical and syntax analysis of the the provided source code and exit.

            \item[-{}-no-backup]~\\
                Don't generate backup files.

            \item[-{}-brief-msg]~\\
                Print only unique messages.

            \item[-{}-no-strict]~\\
                Disable certain error and warning messages.

            \item[-{}-no-warnings]~\\
                Do not print any warnings.

            \item[-{}-no-errors]~\\
                Do not print any errors.

            \item[-{}-no-remarks]~\\
                Do not print any remarks.

            \item[-{}-silent]~\\
                Do not print any warnings, errors, or any other messages, stay completely silent.

            \item[-{}-include, -I <directory>]~\\
                Add directory where the compiler will search for include files.

            \item[-{}-device, -d <device>]~\\
                Specify exact target device, options are:
                \begin{description}
                    \item[For PicoBlaze]~\\
                        kcpsm1, kcpsm1cpld, kcpsm2, kcpsm3, and kcpsm6
                \end{description}

            \item[-{}-precompile, -P <.prc-file>]~\\
                Specify target file for generation of precompiled code.

            \item[-{}-vhdl <.vhd-file>]~\\
                Specify target file for generation of VHDL code.

            \item[-{}-vhdl-tmpl <.vhd-file>]~\\
                Specify VHDL template file.

            \item[-{}-verilog <.v-file>]~\\
                Specify target file for generation of verilog code.

            \item[-{}-verilog-tmpl <.v-file>]~\\
                Specify verilog template file.

            \item[-{}-mem <.mem-file>]~\\
                Specify target file for generation of MEM file.

            \item[-{}-raw-hex-dump <.hex-file>]~\\
                Specify target file for Raw Hex Dump (sequence of 5 digit long hexadecimal numbers separated by CRLF sequence).

            \item[-{}-secondary <.hex file>]~\\
                Specify target file for SPR initialization, output format is Intel HEX.

            \item[-{}-define, -D <name{[}=value{]}>]~\\
                Define a symbol with the given name. Value is optional, the default value is 1; value has to be a decimal number, if specified. Symbol defined using this option is of type ``number'' and is re-definable. This option is particularly useful in conjunction with the conditional compilation directives (\texttt{\#IFDEF}, \texttt{\#IF}, \texttt{\#ELSE}, etc.) Example:~\\
                \verb'mds-compiler -D SKIP_LOOPS -D DEFAULT_TIMEOUT=10 ...'

        \end{description}

    \paragraph{Notes}~\\
        \begin{itemize}
            \item `-{}-' marks the end of options, it becomes useful when you want to compile file(s) which name(s) could be mistaken for a command line option.
            \item When multiple source files are specified, they are compiled as one unit in the order in which they appear on the command line (from left to right).
        \end{itemize}

    \paragraph{Examples}~\\
        \begin{itemize}
            \item \verb'mds-compiler --architecture=PicoBlaze --language=asm --hex=abc.hex abc.asm'\\
                Compile source code file `abc.asm' (-{}-file=abc.asm) for architecture PicoBlaze (-{}-arch=PicoBlaze) written in assembly language (-{}-language=asm), and create file `abc.hex' containing machine code generated generated by the compiler.

            \item \verb'mds-compiler --language asm --architecture PicoBlaze --hex abc.hex abc.asm'\\
                Do the same at the above, only in this case we have used another variant of usage of switches with argument.

            \item \verb'mds-compiler -l asm -a PicoBlaze -x abc.hex abc.asm'\\
                Do the same at the above, only in this case we have used short version of the switches.
        \end{itemize}

\section{Disassembler}
    \subsection{Description}
        Disassembler is a tool intended to generate assembly language code from an object file. In other words it has certain level of capability of reversing the assembly process and regaining the original source code from any object code.

    \paragraph{Usage}~\\
        \verb'mds-disasm  <OPTIONS>  [ -- ] <input-file>'

    \paragraph{Options}~\\
        \begin{description}
            \item[-{}-arch, -a <architecture>]~\\
                \textbf{(REQUIRED)} Specify target architecture, supported architectures are: \textbf{PicoBlaze}

            \item[-{}-family, -f <family>]~\\
                \textbf{(REQUIRED)} Specify processor family, supported families for the given architectures are:
                \begin{description}
                    \item[For PicoBlaze]~\\
                        kcpsm1, kcpsm1cpld, kcpsm2, kcpsm3, and kcpsm6
                \end{description}

            \item[-{}-out, -o <output-file>]~\\
                Specify output file where the resulting assembly language code will be stored.

            \item[-{}-type, -t <input-file-format>]~\\
                Type of the input file; if none provided, disassembler will try to guess the format from input file extension, supported types are:
                \begin{description}
                    \item [hex] Intel 8 HEX, or Intel 16 HEX.
                    \item [srec] Motorola S-Record.
                    \item [bin] Raw binary file.
                    \item [mem] Xilinx MEM file (for PicoBlaze only).
                    \item [vhd] VHDL file (for PicoBlaze only).
                    \item [v] Verilog file (for PicoBlaze only).
                \end{description}

            \item[-{}-cfg-ind <indentation>]~\\
                Indent with:
                \begin{description}
                    \item [spaces] Indent with spaces (default).
                    \item [tabs] Indent with tabs.
                \end{description}

            \item[-{}-cfg-tabsz <n>]~\\
                Consider tab to be displayed at most n spaces wide, here it is by default 8.

            \item[-{}-cfg-eof <end-of-line-character>]~\\
                Specify line separator, available options are:
                \begin{description}
                    \item [clrf] (WINDOWS) use sequence of carriage return and line feed characters
                    \item [If] (UNIX) use a single line feed character (0x0a = ') (default),
                    \item [cr] (APPLE) use single carriage return (0x0d = '').
                \end{description}

            \item[-{}-cfg-sym <symbols>]~\\
                Which kind of addresses should be translated to symbolic names, and which should remain to be represented as numbers, available options are:
                \begin{description}
                    \item [c] Program memory.
                    \item [d] Data memory.
                    \item [r] Register file.
                    \item [p] Port address.
                    \item [k] Immediate constants.
                \end{description}
                This options takes any combination of these, i.e. for example "cdr" stands for program memory + data memory + register file.

            \item[-{}-cfg-lc <character>]~\\
                Use uppercase, or lowercase characters:
                \begin{description}
                    \item [l] Lowercase.
                    \item [u] Uppercase.
                \end{description}

            \item[-{}-cfg-radix <radix>]~\\
                Radix, available options are:
                \begin{description}
                    \item [h] Hexadecimal.
                    \item [d] Decimal.
                    \item [b] Binary.
                    \item [o] Octal.
                \end{description}

            \item[-{}-help, -h]~\\
                Print help message.

            \item[-{}-version, -V]~\\
                Print disassembler version and exit.
        \end{description}

    \paragraph{Notes}~\\
        \begin{itemize}
            \item `-{}-' marks the end of options, it becomes useful when you want to disassemble file which name could be mistaken for a command line option.
        \end{itemize}

\section{Assembler translator}
    \subsection{Description}
        mds-translator is tool for translating assembly language source code from one dialect to another; in this case it translates into assembly language used by MDS (Multitarget Development System) from dialect used by other companies or projects.

    \paragraph{Usage}~\\
        \verb'mds-translator  <OPTIONS>  [ -- ] <input-file>'

    \paragraph{Options}~\\
        \begin{description}
            \item[-{}-type, -t <asm-variant>]~\\
                \textbf{(REQUIRED)} Specify variant of assembly language used in the input file, possible options are:
                \begin{description}
                    \item [1] Xilinx KCPSMx
                    \item [2] Mediatronix KCPSMx
                    \item [3] openPICIDE KCPSMx
                \end{description}

            \item[-{}-output, -o <file.asm>]~\\
                \textbf{(REQUIRED)} Specify output file.

            \item[-{}-cfg-ind <indentation>]~\\
                Indent with:
                \begin{description}
                    \item [keep] Do not alter original indentation.
                    \item [spaces] Indent with spaces.
                    \item [tabs] Indent with tabs.
                \end{description}

            \item[-{}-cfg-tabsz <n>]~\\
                Consider tab to be displayed at most n spaces wide, here it is by default 8.

            \item[-{}-cfg-eof <end-of-line-character>]~\\
                Specify line separator, available options are:
                \begin{description}
                    \item [clrf] (WINDOWS) use sequence of carriage return and line feed characters
                    \item [If] (UNIX) use a single line feed character (0x0a = ') (default),
                    \item [cr] (APPLE) use single carriage return (0x0d = '').
                \end{description}

            \item[-{}-short-inst <true|false>]~\\
                Use instruction shortcuts, e.g. `in' instead of `input', etc. (default: false).

            \item[-{}-cfg-lc-sym <character>]~\\
                Use uppercase, or lowercase characters for symbols:
                \begin{description}
                    \item [l] Lowercase.
                    \item [u] Uppercase.
                \end{description}

            \item[-{}-cfg-lc-dir <character>]~\\
                Use uppercase, or lowercase characters for directives:
                \begin{description}
                    \item [l] Lowercase.
                    \item [u] Uppercase.
                \end{description}

            \item[-{}-cfg-lc-inst <character>]~\\
                Use uppercase, or lowercase characters for instructions:
                \begin{description}
                    \item [l] Lowercase.
                    \item [u] Uppercase.
                \end{description}

            \item[-{}-backup, -b]~\\
                Enable generation of backup files.

            \item[-{}-version, -V]~\\
                Print version and exit.

            \item[-{}-help, -h]~\\
                Print help message.
        \end{description}

    \paragraph{Notes}~\\
        \begin{itemize}
            \item `-{}-' marks the end of options, it becomes useful when you want to disassemble file which name could be mistaken for a command line option.
        \end{itemize}

    \paragraph{Examples}~\\
        \begin{itemize}
            \item \verb'mds-translator --type=1 --output=final_file.asm my_file.psm'\\
        \end{itemize}

\clearpage
\section{Simulator}
    \subsection{Description}
        The main purpose of this tool is to provide means to use the MDS's processor simulator in user written scripts in which the user needs to simulate a processor. For instance you can write a script in Tcl, Python, Bash, etc. and use the mds-proc-sim to test your programs written for PicoBlaze. You can do things like: feed your program with some input and read its output, connect several processors together and make them exchange data, you can even view contents of registers, and all other memories, watch subroutine calls and interrupts, set breakpoints, and many other tasks.

        This tool implements command line interface for the simulator engine. It listens to simple commands, like ``\texttt{sim~step}'' or ``\texttt{get~pc}''. Each command must be on separate line, empty lines are ignored, and everything after the `\texttt{\#}' character is threated as comment and ignored. Response to each command is either ``\texttt{done}'', or ``\texttt{Error: <message>}'' or ``\texttt{Warning: <message>}''. Some commands print some values, in that case these values are printed before the `done' string. This command line interface is case sensitive.

        Input consists of sequence of text lines separated by LF, each line may contains several words separated by spaces or tabs. Input is taken from standard input, output is written to standard output, and error and warning messages are written to standard error output (if possible). End of line character is LF (Line Feed), CR characters at the end of line are ignored as white space.

    \subsection{Invocation}
        \verb'mds-proc-sim  <OPTIONS>'

        \paragraph{Options}~\\
            \begin{description}
                \item[-g, -{}-debug-file <full file name>]~\\
                    \textbf{(REQUIRED)} Specify MDS native debug file.

                \item[-d, -{}-device <device>]~\\
                    \textbf{(REQUIRED)} Specify exact device for simulation.
                    \begin{description}
                        \item[For PicoBlaze]~\\
                            kcpsm1, kcpsm1cpld, kcpsm2, kcpsm3, and kcpsm6
                    \end{description}

                \item[-c, -{}-code-file <full file name>]~\\
                    \textbf{(REQUIRED)} Specify file with machine code for simulation.

                \item[-t, -{}-code-file-type <file type>]~\\
                    \textbf{(REQUIRED)} Specify type of the machine code file, supported are:
                    \begin{description}
                        \item [hex] Intel 8 HEX, or Intel 16 HEX.
                        \item [srec] Motorola S-Record.
                        \item [bin] Raw binary file.
                        \item [mem] Xilinx MEM file (for PicoBlaze only).
                        \item [vhd] VHDL file (for PicoBlaze only).
                        \item [v] Verilog file (for PicoBlaze only).
                        \item [rawhex] Raw HEX dump (for PicoBlaze only).
                    \end{description}

                \item[-p, -{}-proc-def-file <processor definition file>]~\\
                    Specify architecture for user defined processor, this option makes sense only if --device=Adaptable

                \item[-h, -{}-help]~\\
                    Print a brief help message.

                \item[-V, -{}-version]~\\
                    Print version information and exit.

                \item[-s, -{}-silent]~\\
                    Be less verbose
            \end{description}

        \paragraph{Examples}~\\
            \verb'mds-proc-sim -d kcpsm6 -g Example1.dbg -c Example1.ihex -t hex'

    \subsection{Commands}
        Commands might have several subcommands and might take arguments, arguments might be optional; if an argument is a number, it might be hexadecimal, decimal, octal, or binary; radix is specified with either prefix or suffix notation in the same way as in the assembler (examples: \texttt{0xff}, \texttt{255}, \texttt{0377}, \texttt{0b11111111}; or \texttt{ffh}, \texttt{255d}, \texttt{377q}, \texttt{11111111b}). When simulator responses to a command with a number, this number is always decimal; when there are more that one number in the response, these numbers are separated by a single space. Each response is written on separate line.

        \begin{description}
            \item[set]~\\
                Set something, subcommands:
                \begin{description}
                    \item[pc \texttt{<address>}]~\\
                        Set program counter, i.e. jump somewhere, e.g. ``set pc 0x3ff'' - jump to 0x3ff.
                    \item[flag \texttt{<flag> <value>}]~\\
                        Set processor flag, flag is one of: {C, Z, IE, I, pC, pZ}, value is either 0 or 1; pC is pre-Carry and pZ is pre-Zero (i.e. before ISR), I is interrupt, and IE is interrupt enable.
                    \item[memory \texttt{<space> <address> <value>}]~\\
                        Set some location in memory to some value, space is one of: { portin, portout, register, data, code, stack} (where ``code'' is program memory). E.g. ``set memory register 0x0 0x22'' - set S0 to 0x22.
                    \item[size \texttt{<space> <value>}]~\\
                        Resize the specified memory. E.g. ``set size code 128'' - reduces size of program memory to 128.
                    \item[breakpoint \texttt{<file:line> {[} <value> {]}}]~\\
                        Set breakpoint, value is either 0 or 1, value is optional and 1 is default, 1 means set and 0 means unset. Please use ``get locations'' to see where breakpoints can actually be set.
                \end{description}

            \item[get]~\\
                Get something, subcommands:
                \begin{description}
                    \item[pc]~\\
                        Get current value of program counter.
                    \item[flag \texttt{<flag>}]~\\
                        Get processor flag (see ``set flag'' command for details).
                    \item[memory \texttt{<space> <address> {[} <end-address> {]}}]~\\
                        Read memory, when end-address is specified
                    \item[size \texttt{<space>}]~\\
                        Get size of some memory.
                    \item[cycles]~\\
                        Get total number of machine cycles executed on the simulated processor.
                    \item[breakpoints]~\\
                        List breakpoints.
                    \item[locations]~\\
                        List locations where it makes sense to set a breakpoint.
                \end{description}

            \item[file]~\\
                Load or save a file, subcommands:
                \begin{description}
                    \item[load \texttt{<space> <type> <file>}]~\\
                        Load contents of the specified file (\texttt{<file>}) into the specified memory (\texttt{<space>}), \texttt{<type>} is type of file, \texttt{<file>} is file name.
                    \item[save \texttt{<space> <type> <file>}]~\\
                        Save contents of the specified memory (\texttt{<space>}) into the specified file (\texttt{<file>}).
                \end{description}

            \item[sim]~\\
                Simulate program.
                \begin{description}
                    \item[step \texttt{{[} <steps> {]}}]~\\
                        Step, optionally execute \texttt{<steps>} steps.
                    \item[run]~\\
                        Run program. (This forks from the main thread, further commands are protected by mutexes.)
                    \item[animate]~\\
                        Animate program. (This forks from the main thread, further commands are protected by mutexes.)
                    \item[halt]~\\
                        Halt program animation or run.
                    \item[reset]~\\
                        Reset processor.
                    \item[irq]~\\
                        Interrupt request.
                \end{description}

            \item[exit \texttt{{[} <code> {]}}]~\\
                Exit with exit code \texttt{<code>}; if \texttt{<code>} is not specified, exit with code 0.

            \item[help \texttt{{[} <command> {]}}]~\\
                Print a brief help message about the commands; if optional \texttt{<command>} is specified, print message concerning that command.
        \end{description}

        \paragraph{Example}~\\
            \begin{tabular}{l|l}
                \textbf{Input}          & \textbf{Output} \\
                \hline
                \verb'get cycles'       & \\
                                        & \verb'0'\\
                                        & \verb'done'\\
                \hline
                \verb'sim step'         & \\
                                        & \verb'done'\\
                \hline
                \verb'get cycles'       & \\
                                        & \verb'1'\\
                                        & \verb'done'\\
                \hline
                \verb'sim step 0xA'     & \\
                                        & \verb'done'\\
                \hline
                \verb'get cycles'       & \\
                                        & \verb'11'\\
                                        & \verb'done'\\
            \end{tabular}

    \subsection{Events}
        When something changes in the simulator, for instance content of some register, simulator prints a message like: ``\texttt{>{}>{}>~(register)~mem\_inf\_wr\_val\_written~@~0}''. Event message always starts with ``\texttt{>{}>{}> }'' sequence so it can be easily identified and handled by a parser. All numbers present in an even message are represented in decimal radix.

        \paragraph{Event message syntax:}~\\
            \verb'>>> [ (<attribute>) ] <event-id> [ <value> ] [ @ <location> ] [ : <detail> ]'\\

            Tokens enclosed by square brackets may not be present. \texttt{<attribute>} and \texttt{<event-id>} are strings; \texttt{<value>}, \texttt{<location>}, and \texttt{<detail>} are always numbers.

        \paragraph{Examples of event messages:}~\\
            \verb'    >>> cycles 2'\\
            \verb'    >>> cpu_pc_changed 2'\\
            \verb'    >>> flags_c_changed : 1'\\
            \verb'    >>> flags_z_changed : 0'\\
            \verb'    >>> (register) mem_inf_wr_val_written @ 0'\\
            \verb'    >>> (register) mem_inf_wr_val_changed @ 0'\\

        \begin{description}
            \item[General simulator events]~\\
                ...
                \begin{description}
                    \item[event-id: \texttt{cycles}]~\\
                        Total number of \texttt{<value>} machine cycles has been executed so far.
                    \item[event-id: \texttt{breakpoint}]~\\
                        Breakpoint reached at address specified in \texttt{<location>}.
                \end{description}
            \item[Processor flags related events]~\\
                All have \texttt{<detail>} which is either \texttt{0} or \texttt{1} and indicates the new value of the concerned flag.
                \begin{description}
                    \item[event-id: \texttt{flags\_z\_changed}]~\\
                        Zero flag has been changed.
                    \item[event-id: \texttt{flags\_c\_changed}]~\\
                        Carry flag has been changed.
                    \item[event-id: \texttt{flags\_pz\_changed}]~\\
                        Pre-zero flag has been changed.
                    \item[event-id: \texttt{flags\_pc\_changed}]~\\
                        Pre-carry flag has been changed.
                    \item[event-id: \texttt{flags\_ie\_changed}]~\\
                        Interrupt enable flag has been changed.
                    \item[event-id: \texttt{flags\_int\_changed}]~\\
                        Interrupt flag has been changed.
                \end{description}
%                 \item[Stack]~\\
% out << ">>> stack_overflow @ " << locationOrReason << " : " << detail << std::endl;
% out << ">>> stack_underflow @ " << locationOrReason << " : " << detail << std::endl;
% out << ">>> stack_sp_changed : " << locationOrReason << std::endl;
%                 \item[Interrupts]~\\
        \end{description}
