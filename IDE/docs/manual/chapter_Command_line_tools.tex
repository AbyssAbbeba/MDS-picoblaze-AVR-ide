Assembler, Disassembler, and Assembler Translator may also be invoked from command line and therefore used in your scripts. On Linux versions there is also man page for each one of them.

\section{Assembler}
    \subsection{Description}
        Macro-assembler for PicoBlaze soft-core processors, this tool takes one or more source code files and produces compiled machine code file usable in JTAG loaders, processors simulators, and similar tools; along with these files containing compiled machine code compiler also produces extensive debugging output. MDS macro-assembler is made to run fast and extensively tested for greater reliability, it feature set includes various special macros and user defined macros support making it one of the world most advanced assemblers available on the market today for PicoBlaze processors.

        By default the compiler does not generate any output files, that might be useful when you simply want to check a file for errors but you do not want that when you actually need to compile something a use the resulting machine code. When you need the machine code or any other file output, you have to specify which file or files you want the compiler to generate by providing the corresponding options.

    \paragraph{Usage}~\\
        \verb'mds-compiler  <OPTIONS>  [ -- ] [ source-file ...  ]'

    \paragraph{Options}~\\
        \begin{description}
            \item[-{}-architecture, -a <architecture>]~\\
                \textbf{(REQUIRED)} Specify target architecture, supported architectures are:
                \begin{description}
                    \item[PicoBlaze] KCPSM soft-core processor
                \end{description}

            \item[-{}-language, -l <programming language>]~\\
                \textbf{(REQUIRED)} Specify programming language, supported languages are:
                \begin{description}
                    \item[asm] Assembly language
                \end{description}

            \item[-{}-hex, -x <intel HEX-file>]~\\
                Specify output file with machine code generated as a result of compilation, data will be stored in Intel 8 Hex format.

            \item[-{}-debug, -g <MDS-native-debug-file>]~\\
                Specify output file with code for MCU simulator and other debugging tools.

            \item[-{}-srec <Motorola-S-REC-file>]~\\
                Specify output file with machine code generated as a result of compilation, data will be stored in Motorola S-record format.

            \item[-{}-binary <binary-file>]~\\
                Specify output file with machine code generated as a result of compilation, data will be stored in raw binary format.

            \item[-{}-lst <code-listing>]~\\
                Specify output file where code listing generated during compilation will be stored.

            \item[-{}-mtable, -m <table-of-macros>]~\\
                Specify file in which the compiler will put table of macros defined in your code.

            \item[-{}-stable, -s <table-of-symbols>]~\\
                Specify file in which the compiler will put table of symbols defined in your code.

            \item[-{}-strtable <table of symbols>]~\\
                Specify file in which the compiler will put table of strings defined in your code.

            \item[-{}-help, -h]~\\
                Print help message.

            \item[-{}-version, -V]~\\
                Print compiler version and exit.

            \item[-{}-check, -c]~\\
                Do not perform the actual compilation, do only lexical and syntax analysis of the the provided source code and exit.

            \item[-{}-no-backup]~\\
                Don't generate backup files.

            \item[-{}-brief-msg]~\\
                Print only unique messages.

            \item[-{}-no-strict]~\\
                Disable certain error and warning messages.

            \item[-{}-no-warnings]~\\
                Do not print any warnings.

            \item[-{}-no-errors]~\\
                Do not print any errors.

            \item[-{}-no-remarks]~\\
                Do not print any remarks.

            \item[-{}-silent]~\\
                Do not print any warnings, errors, or any other messages, stay completely silent.

            \item[-{}-include, -I <directory>]~\\
                Add directory where the compiler will search for include files.

            \item[-{}-device, -d <device>]~\\
                Specify exact target device, options are:
                \begin{description}
                    \item[For PicoBlaze]~\\
                        kcpsm1, kcpsm1cpld, kcpsm2, kcpsm3, and kcpsm6
                \end{description}

            \item[-{}-precompile, -P <.prc-file>]~\\
                Specify target file for generation of precompiled code.

            \item[-{}-vhdl <.vhd-file>]~\\
                Specify target file for generation of VHDL code.

            \item[-{}-vhdl-tmpl <.vhd-file>]~\\
                Specify VHDL template file.

            \item[-{}-verilog <.v-file>]~\\
                Specify target file for generation of verilog code.

            \item[-{}-verilog-tmpl <.v-file>]~\\
                Specify verilog template file.

            \item[-{}-mem <.mem-file>]~\\
                Specify target file for generation of MEM file.

            \item[-{}-raw-hex-dump <.hex-file>]~\\
                Specify target file for Raw Hex Dump (sequence of 5 digit long hexadecimal numbers separated by CRLF sequence).

            \item[-{}-secondary <.hex file>]~\\
                Specify target file for SPR initialization, output format is Intel HEX.

            \item[-{}-define, -D <name{[}=value{]}>]~\\
                Define a symbol with the given name. Value is optional, the default value is 1; value has to be a decimal number, if specified. Symbol defined using this option is of type ``number'' and is re-definable. This option is particularly useful in conjunction with the conditional compilation directives (\texttt{\#IFDEF}, \texttt{\#IF}, \texttt{\#ELSE}, etc.) Example:~\\
                \verb'mds-compiler -D SKIP_LOOPS -D DEFAULT_TIMEOUT=10 ...'

        \end{description}

    \paragraph{Notes}~\\
        \begin{itemize}
            \item `-{}-' marks the end of options, it becomes useful when you want to compile file(s) which name(s) could be mistaken for a command line option.
            \item When multiple source files are specified, they are compiled as one unit in the order in which they appear on the command line (from left to right).
        \end{itemize}

    \paragraph{Examples}~\\
        \begin{itemize}
            \item \verb'mds-compiler --architecture=PicoBlaze --language=asm --hex=abc.hex abc.asm'\\
                Compile source code file `abc.asm' (-{}-file=abc.asm) for architecture PicoBlaze (-{}-arch=PicoBlaze) written in assembly language (-{}-language=asm), and create file `abc.hex' containing machine code generated generated by the compiler.

            \item \verb'mds-compiler --language asm --architecture PicoBlaze --hex abc.hex abc.asm'\\
                Do the same at the above, only in this case we have used another variant of usage of switches with argument.

            \item \verb'mds-compiler -l asm -a PicoBlaze -x abc.hex abc.asm'\\
                Do the same at the above, only in this case we have used short version of the switches.
        \end{itemize}

\section{Disassembler}
    \subsection{Description}
        Disassembler is a tool intended to generate assembly language code from an object file. In other words it has certain level of capability of reversing the assembly process and regaining the original source code from any object code.

    \paragraph{Usage}~\\
        \verb'mds-disasm  <OPTIONS>  [ -- ] <input-file>'

    \paragraph{Options}~\\
        \begin{description}
            \item[-{}-arch, -a <architecture>]~\\
                \textbf{(REQUIRED)} Specify target architecture, supported architectures are: \textbf{PicoBlaze}

            \item[-{}-family, -f <family>]~\\
                \textbf{(REQUIRED)} Specify processor family, supported families for the given architectures are:
                \begin{description}
                    \item[For PicoBlaze]~\\
                        kcpsm1, kcpsm1cpld, kcpsm2, kcpsm3, and kcpsm6
                \end{description}

            \item[-{}-out, -o <output-file>]~\\
                Specify output file where the resulting assembly language code will be stored.

            \item[-{}-type, -t <input-file-format>]~\\
                Type of the input file; if none provided, disassembler will try to guess the format from input file extension, supported types are:
                \begin{description}
                    \item [hex] Intel 8 HEX, or Intel 16 HEX.
                    \item [srec] Motorola S-Record.
                    \item [bin] Raw binary file.
                    \item [mem] Xilinx MEM file (for PicoBlaze only).
                    \item [vhd] VHDL file (for PicoBlaze only).
                    \item [v] Verilog file (for PicoBlaze only).
                \end{description}

            \item[-{}-cfg-ind <indentation>]~\\
                Indent with:
                \begin{description}
                    \item [spaces] Indent with spaces (default).
                    \item [tabs] Indent with tabs.
                \end{description}

            \item[-{}-cfg-tabsz <n>]~\\
                Consider tab to be displayed at most n spaces wide, here it is by default 8.

            \item[-{}-cfg-eof <end-of-line-character>]~\\
                Specify line separator, available options are:
                \begin{description}
                    \item [clrf] (WINDOWS) use sequence of carriage return and line feed characters
                    \item [If] (UNIX) use a single line feed character (0x0a = ') (default),
                    \item [cr] (APPLE) use single carriage return (0x0d = '').
                \end{description}

            \item[-{}-cfg-sym <symbols>]~\\
                Which kind of addresses should be translated to symbolic names, and which should remain to be represented as numbers, available options are:
                \begin{description}
                    \item [c] Program memory.
                    \item [d] Data memory.
                    \item [r] Register file.
                    \item [p] Port address.
                    \item [k] Immediate constants.
                \end{description}
                This options takes any combination of these, i.e. for example "cdr" stands for program memory + data memory + register file.

            \item[-{}-cfg-lc <character>]~\\
                Use uppercase, or lowercase characters:
                \begin{description}
                    \item [l] Lowercase.
                    \item [u] Uppercase.
                \end{description}

            \item[-{}-cfg-radix <radix>]~\\
                Radix, available options are:
                \begin{description}
                    \item [h] Hexadecimal.
                    \item [d] Decimal.
                    \item [b] Binary.
                    \item [o] Octal.
                \end{description}

            \item[-{}-help, -h]~\\
                Print help message.

            \item[-{}-version, -V]~\\
                Print disassembler version and exit.
        \end{description}

    \paragraph{Notes}~\\
        \begin{itemize}
            \item `-{}-' marks the end of options, it becomes useful when you want to disassemble file which name could be mistaken for a command line option.
        \end{itemize}

\section{Assembler translator}
    \subsection{Description}
        mds-translator is tool for translating assembly language source code from one dialect to another; in this case it translates into assembly language used by MDS (Multitarget Development System) from dialect used by other companies or projects.

    \paragraph{Usage}~\\
        \verb'mds-translator  <OPTIONS>  [ -- ] <input-file>'

    \paragraph{Options}~\\
        \begin{description}
            \item[-{}-type, -t <asm-variant>]~\\
                \textbf{(REQUIRED)} Specify variant of assembly language used in the input file, possible options are:
                \begin{description}
                    \item [1] Xilinx KCPSMx
                    \item [2] Mediatronix KCPSMx
                    \item [3] openPICIDE KCPSMx
                \end{description}

            \item[-{}-output, -o <file.asm>]~\\
                \textbf{(REQUIRED)} Specify output file.

            \item[-{}-cfg-ind <indentation>]~\\
                Indent with:
                \begin{description}
                    \item [keep] Do not alter original indentation.
                    \item [spaces] Indent with spaces.
                    \item [tabs] Indent with tabs.
                \end{description}

            \item[-{}-cfg-tabsz <n>]~\\
                Consider tab to be displayed at most n spaces wide, here it is by default 8.

            \item[-{}-cfg-eof <end-of-line-character>]~\\
                Specify line separator, available options are:
                \begin{description}
                    \item [clrf] (WINDOWS) use sequence of carriage return and line feed characters
                    \item [If] (UNIX) use a single line feed character (0x0a = ') (default),
                    \item [cr] (APPLE) use single carriage return (0x0d = '').
                \end{description}

            \item[-{}-short-inst <true|false>]~\\
                Use instruction shortcuts, e.g. `in' instead of `input', etc. (default: false).

            \item[-{}-cfg-lc-sym <character>]~\\
                Use uppercase, or lowercase characters for symbols:
                \begin{description}
                    \item [l] Lowercase.
                    \item [u] Uppercase.
                \end{description}

            \item[-{}-cfg-lc-dir <character>]~\\
                Use uppercase, or lowercase characters for directives:
                \begin{description}
                    \item [l] Lowercase.
                    \item [u] Uppercase.
                \end{description}

            \item[-{}-cfg-lc-inst <character>]~\\
                Use uppercase, or lowercase characters for instructions:
                \begin{description}
                    \item [l] Lowercase.
                    \item [u] Uppercase.
                \end{description}

            \item[-{}-backup, -b]~\\
                Enable generation of backup files.

            \item[-{}-version, -V]~\\
                Print version and exit.

            \item[-{}-help, -h]~\\
                Print help message.
        \end{description}

    \paragraph{Notes}~\\
        \begin{itemize}
            \item `-{}-' marks the end of options, it becomes useful when you want to disassemble file which name could be mistaken for a command line option.
        \end{itemize}

    \paragraph{Examples}~\\
        \begin{itemize}
            \item \verb'mds-translator --type=1 --output=final_file.asm my_file.psm'\\
        \end{itemize}

\clearpage
\section{Simulator}
    \subsection{Description}
        The main purpose of this tool is to provide means to use the MDS's processor simulator in user written scripts in which the user needs to simulate a processor. For instance you can write a script in Tcl, Python, Bash, etc. and use the mds-proc-sim to test your programs written for PicoBlaze. You can do things like: feed your program with some input and read its output, connect several processors together and make them exchange data, you can even view contents of registers, and all other memories, watch subroutine calls and interrupts, set breakpoints, and many other tasks.

        This tool implements command line interface for the simulator engine. It listens to simple commands, like ``\texttt{sim~step}'' or ``\texttt{get~pc}''. Each command must be on separate line, empty lines are ignored, and everything after the `\texttt{\#}' character is threated as comment and ignored. Response to each command is either ``\texttt{done}'', or ``\texttt{Error: <message>}'' or ``\texttt{Warning: <message>}''. Some commands print some values, in that case these values are printed before the `done' string. This command line interface is case sensitive.

        Input consists of sequence of text lines separated by LF, each line may contains several words separated by spaces or tabs. Input is taken from standard input, output is written to standard output, and error and warning messages are written to standard error output (if possible). End of line character is LF (Line Feed), CR characters at the end of line are ignored as white space.

    \subsection{Invocation}
        \verb'mds-proc-sim  <OPTIONS>'

        \paragraph{Options}~\\
            \begin{description}
                \item[-g, -{}-debug-file <full-file-name>]~\\
                    \textbf{(REQUIRED)} Specify MDS native debug file.

                \item[-d, -{}-device <device>]~\\
                    \textbf{(REQUIRED)} Specify exact device for simulation.
                    \begin{description}
                        \item[For PicoBlaze]~\\
                            kcpsm1, kcpsm1cpld, kcpsm2, kcpsm3, and kcpsm6
                    \end{description}

                \item[-c, -{}-code-file <full-file-name>]~\\
                    \textbf{(REQUIRED)} Specify file with machine code for simulation.

                \item[-t, -{}-code-file-type <file-type>]~\\
                    \textbf{(REQUIRED)} Specify type of the machine code file, supported are:
                    \begin{description}
                        \item [hex] Intel 8 HEX, or Intel 16 HEX.
                        \item [srec] Motorola S-Record.
                        \item [bin] Raw binary file.
                        \item [mem] Xilinx MEM file (for PicoBlaze only).
                        \item [vhd] VHDL file (for PicoBlaze only).
                        \item [v] Verilog file (for PicoBlaze only).
                        \item [rawhex] Raw HEX dump (for PicoBlaze only).
                    \end{description}

                \item[-p, -{}-proc-def-file <processor definition file>]~\\
                    Specify architecture for user defined processor, this option makes sense only if --device=Adaptable

                \item[-h, -{}-help]~\\
                    Print a brief help message.

                \item[-V, -{}-version]~\\
                    Print version information and exit.

                \item[-s, -{}-silent]~\\
                    Be less verbose
            \end{description}

        \paragraph{Examples}~\\
            \verb'mds-proc-sim -d kcpsm6 -g Example1.dbg -c Example1.ihex -t hex'

    \subsection{Commands}
        Commands might have several subcommands and might take arguments, arguments might be optional; if an argument is a number, it might be hexadecimal, decimal, octal, or binary; radix is specified with either prefix or suffix notation in the same way as in the assembler (examples: \texttt{0xff}, \texttt{255}, \texttt{0377}, \texttt{0b11111111}; or \texttt{ffh}, \texttt{255d}, \texttt{377q}, \texttt{11111111b}). When simulator responses to a command with a number, this number is always decimal; when there are more that one number in the response, these numbers are separated by a single space. There are no negative or real numbers, all numbers have to be nonnegative integers in both input and output. Each response is written on separate line.

        \begin{description}
            \item[set]~\\
                Set something in the simulator, this command has the following subcommands:
                \begin{description}
                    \item[pc \texttt{<address>}]~\\
                        Set program counter to \texttt{<address>}. For example ``\texttt{set~pc~0x3ff}'' executes unconditional jump at address 0x3ff.
                    \item[flag \texttt{<flag> <value>}]~\\
                        Set processor flag. \texttt{<flag>} has to be one of \{C, Z, IE, I, pC, pZ\} (C is Carry, Z is Zero, pC is pre-Carry (Carry before ISR), pZ is pre-Zero (Zero before ISR), I is interrupt, and IE is interrupt enable), \texttt{<value>} has to be either ``0'' or ``1''.
                    \item[memory \texttt{<space> <address> <value>}]~\\
                        Change content of the \texttt{<space>} memory or I/O at \texttt{<address>}. \texttt{<space>} has to be one of \{portin, portout, register, data, code, stack\} where ``code'' is program memory, ``data'' is scratch-pad ram, and others should be obvious). For example ``\texttt{set memory register 0 0x22}''  sets register at address 0 (i.e. S0) to value 0x22.

                        When dealing with memory banks, address is always absolute i.e. not respecting boundary of bank size; this can be demonstrated on an example: suppose there are 2 banks, bank size is 16 and memory size is 32, address 1 in the 1st bank is 1 while the address in the other bank is $16+1=17$.

                        When the provided value has higher bit width than what can be stored at the given location, the value is automatically silently trimmed from the right; for example when the given value is 0x123 while it is supposed to be 8-bit number, the value stored will be 0x23.
                    \item[size \texttt{<space> <size>}]~\\
                        Resize the \texttt{<space>} memory or I/O to new size of \texttt{<size>}. (Use with care.) For example ``\texttt{set size code 128}'' reduces size of the program memory to 128.
                    \item[breakpoint \texttt{<file:line> {[} <value> {]}}]~\\
                        Set breakpoint at \texttt{<file:line>}, \texttt{<value>} is optional and may be set either to 0 or 1, 1 is default, 1 means set and 0 means unset. Please use command ``\texttt{get locations}'' to see at which locations breakpoints can actually be set. Breakpoints are effective only in run mode or animation mode.
                    \item[config \texttt{<key> <value>}]
                        Alter processor or simulator configuration.
                        ~\\\texttt{<key>} might be:
                        \begin{itemize}
                            \item \texttt{hwbuild} for which the \texttt{<value>} is an 8-bit number.
                            \item \texttt{interrupt\_vector} for which the \texttt{<value>} is an address to program memory.
                        \end{itemize}
                \end{description}

            \item[get]~\\
                Get some information from the simulator, this command has the following subcommands:
                \begin{description}
                    \item[pc]~\\
                        Get current value of the program counter.
                    \item[flag \texttt{<flag>}]~\\
                        Get processor flag, please see the ``\texttt{set~flag}'' command for details concerning which flags can be retrieved.
                    \item[memory \texttt{<space> <address> {[} <end-address> {]}}]~\\
                        Read memory or I/O. When \texttt{<end-address>} is specified, this command will output a space separated list of decimal values read from the memory in range {[}\texttt{<address>},~\texttt{<end-address>}{]}.
                    \item[size \texttt{<space>}]~\\
                        Get size of the specified memory or I/O, please see ``\texttt{set~size}'' command for additional details.
                    \item[cycles]~\\
                        Get total number of machine cycles executed on the simulated processor so far. This value is automatically set to zero when the simulator is reseted using the ``\texttt{sim~reset}'' command.
                    \item[breakpoints]~\\
                        List breakpoints set by user using the ``\texttt{set~breakpoint}'' command.
                    \item[locations]~\\
                        List source locations (i.e. files and line numbers) where it is possible to set a breakpoint.
                \end{description}

            \item[file]~\\
                Load or save a file with memory dump; this commands recognizes the same file types at the ``\texttt{-{}-code-file-type}'' command line option, and memory spaces are the same as with the ``\texttt{set~memory}'' command.
                This command has the following subcommands:
                \begin{description}
                    \item[load \texttt{<space> <type> <file>}]~\\
                        Load contents of the specified file (\texttt{<file>}) into the specified memory (\texttt{<space>}), \texttt{<type>} is type of file, \texttt{<file>} is file name.
                    \item[save \texttt{<space> <type> <file>}]~\\
                        Save contents of the specified memory (\texttt{<space>}) into the specified file (\texttt{<file>}).
                \end{description}

            \item[sim]~\\
                Simulate program.
                \begin{description}
                    \item[step \texttt{{[} <steps> {]}}]~\\
                        Step, optionally execute \texttt{<steps>} steps.
                    \item[run]~\\
                        Run program. Program animation will run in a separate thread while this tool continues to listen and answer to commands. It is safe to use all other commands while the simulator is running, e.g. ``\texttt{set~memory~...}'', etc.
                    \item[animate]~\\
                        Animate program. Program animation will run in a separate thread while this tool continues to listen and answer to commands. It is safe to use all other commands while the simulator is running, e.g. ``\texttt{set~memory~...}'', etc.
                    \item[halt]~\\
                        Halt program animation or run.
                    \item[reset]~\\
                        Reset simulated processor.
                    \item[irq]~\\
                        Invoke an interrupt request.
                \end{description}

            \item[exit \texttt{{[} <code> {]}}]~\\
                Exit this command line interface with exit code \texttt{<code>}; if the \texttt{<code>} is not specified, default exit code with value of 0 will be used.

            \item[help \texttt{{[} <command> {]}}]~\\
                Print a brief help message about the commands; if optional \texttt{<command>} is specified, this command prints a message concerning specifically the given \texttt{<command>}.
        \end{description}

        \paragraph{Example}~\\
            \begin{tabular}{l|l}
                \textbf{Input}          & \textbf{Output} \\
                \hline
                \verb'get cycles'       & \\
                                        & \verb'0'\\
                                        & \verb'done'\\
                \hline
                \verb'sim step'         & \\
                                        & \verb'done'\\
                \hline
                \verb'get cycles'       & \\
                                        & \verb'1'\\
                                        & \verb'done'\\
                \hline
                \verb'sim step 0xA'     & \\
                                        & \verb'done'\\
                \hline
                \verb'get cycles'       & \\
                                        & \verb'11'\\
                                        & \verb'done'\\
            \end{tabular}

    \subsection{Events}
        When something changes in the simulator, for instance content of some register, simulator prints a message like: ``\texttt{>{}>{}>~(register)~mem\_inf\_wr\_val\_written~@~0}''. Event message always starts with ``\texttt{>{}>{}> }'' sequence so it can be easily identified and handled by a parser. All numbers present in an even message are represented in decimal radix.

        \paragraph{Event message syntax:}~\\
            \verb'>>> [ (<attribute>) ] <event-id> [ <value> ] [ @ <location> ] [ : <detail> ]'\\

            Tokens enclosed by square brackets may not be present. \texttt{<attribute>} and \texttt{<event-id>} are strings; \texttt{<value>}, \texttt{<location>}, and \texttt{<detail>} are always numbers.

        \paragraph{Examples of event messages:}~\\
            \verb'    >>> cycles 2'\\
            \verb'    >>> cpu_pc_changed 2'\\
            \verb'    >>> flags_c_changed : 1'\\
            \verb'    >>> flags_z_changed : 0'\\
            \verb'    >>> (register) mem_inf_wr_val_written @ 0'\\
            \verb'    >>> (register) mem_inf_wr_val_changed @ 0'\\

        \begin{description}
            \item[General simulator events]~\\
                These events are not related to any specific part of the simulated processor.
                \begin{description}
                    \item[\texttt{>{}>{}> cycles <value>}]~\\
                        Total number of \texttt{<value>} machine cycles has been executed so far.
                    \item[\texttt{>{}>{}> breakpoint @ <location>}]~\\
                        Breakpoint reached at address specified by \texttt{<location>}.
                \end{description}

            \item[Processor flags group]~\\
                All these events have \texttt{<detail>} which is either \texttt{0} or \texttt{1} and indicates the new value of the concerned flag.
                \begin{description}
                    \item[\texttt{>{}>{}> flags\_z\_changed : <detail>}]~\\
                        Zero flag has been changed to \texttt{<detail>}.
                    \item[\texttt{>{}>{}> flags\_c\_changed : <detail>}]~\\
                        Carry flag has been changed to \texttt{<detail>}.
                    \item[\texttt{>{}>{}> flags\_pz\_changed : <detail>}]~\\
                        Pre-zero flag has been changed to \texttt{<detail>}.
                    \item[\texttt{>{}>{}> flags\_pc\_changed : <detail>}]~\\
                        Pre-carry flag has been changed to \texttt{<detail>}.
                    \item[\texttt{>{}>{}> flags\_ie\_changed : <detail>}]~\\
                        Interrupt enable flag has been changed to \texttt{<detail>}.
                    \item[\texttt{>{}>{}> flags\_int\_changed : <detail>}]~\\
                        Interrupt flag has been changed to \texttt{<detail>}.
                \end{description}

            \item[Call stack group]~\\
                These events are related solely to content of the processor call stack, they does not contain detailed information about interrupt services and subroutine calls.
                \begin{description}
                    \item[\texttt{>{}>{}> stack\_overflow @ <location> : <detail>}]~\\
                        Call stack overflow occurred when PC was \texttt{<location>}, new virtual stack pointer value is \texttt{<detail>}.
                    \item[\texttt{>{}>{}> stack\_underflow @ <location> : <detail>}]~\\
                        Call stack underflow occurred when PC was \texttt{<location>}, new virtual stack pointer value is \texttt{<detail>}.
                    \item[\texttt{>{}>{}> stack\_sp\_changed : <detail>}]~\\
                        Content of the call stack has been changed, now the stack contains \texttt{<detail>} return addresses.
                \end{description}

            \item[Interrupt group]~\\
                These are interrupt service routine (ISR) handling related events.
                \begin{description}
                    \item[\texttt{>{}>{}> int\_irq\_denied}]~\\
                        Interrupt request denied.
                    \item[\texttt{>{}>{}> int\_entering\_interrupt}]~\\
                        Entering ISR (Interrupt Service Routine).
                    \item[\texttt{>{}>{}> int\_leaving\_interrupt}]~\\
                        Leaving ISR (Interrupt Service Routine).
                \end{description}

            \item[Instruction set group]~\\
                These events are related to instruction opcode decoding and instruction execution in general.
                \begin{description}
                    \item[\texttt{>{}>{}> cpu\_undefined\_opcode @ <location>}]~\\
                        Reading instruction opcode from uninitialized memory location at address \texttt{<location>}.
                    \item[\texttt{>{}>{}> cpu\_invalid\_opcode @ <location> : <detail>}]~\\
                        Unable to decode instruction opcode: \texttt{<detail>}, read from address: \texttt{<location>}.
                    \item[\texttt{>{}>{}> cpu\_invalid\_irq @ <location>}]~\\
                        Attempt to invoke interrupt service from \texttt{<location>} while it is not possible.
                    \item[\texttt{>{}>{}> cpu\_invalid\_ret @ <location>}]~\\
                        Attempt to return from subroutine at \texttt{<location>} while there is no subroutine to return from.
                    \item[\texttt{>{}>{}> cpu\_invalid\_reti @ <location>}]~\\
                        Attempt to return from interrupt service at \texttt{<location>} while there is no interrupt service subroutine to return from.
                    \item[\texttt{>{}>{}> cpu\_pc\_overflow <value>}]~\\
                        Program counter overflow, new value of PC is \texttt{<value>}.
                    \item[\texttt{>{}>{}> cpu\_pc\_underflow <value>}]~\\
                        Program counter underflow, new value of PC is \texttt{<value>}.
                    \item[\texttt{>{}>{}> cpu\_pc\_changed <value>}]~\\
                        Program counter changed, new value of PC is \texttt{<value>}.
                    \item[\texttt{>{}>{}> cpu\_call @ <location> : <detail>}]~\\
                        Calling subroutine at address \texttt{<detail>} from address \texttt{<location>}.
                    \item[\texttt{>{}>{}> cpu\_return @ <location>}]~\\
                        Returning from subroutine, return request occurred at address \texttt{<location>}.
                    \item[\texttt{>{}>{}> cpu\_irq @ <location>}]~\\
                        Calling interrupt service subroutine from address \texttt{<location>}.
                    \item[\texttt{>{}>{}> cpu\_return\_from\_isr @ <location>}]~\\
                        Returning from interrupt service subroutine, return request occurred at address \texttt{<location>}.
                \end{description}

            \item[Memory group]~\\
                \texttt{<attribute>} indicates memory space where the event occurred:
                \begin{description}
                    \item[\texttt{register}]~\\
                        Register file.
                    \item[\texttt{data}]~\\
                        Scratch-pad ram.
                    \item[\texttt{stack}]~\\
                        Call stack
                    \item[\texttt{code}]~\\
                        Program memory.
%                     \item[\texttt{eeprom}]~\\
                \end{description}

                Events are:
                \begin{description}
                    \item[\texttt{>{}>{}> (<attribute>) mem\_inf\_wr\_val\_changed @ <location>}]~\\
                        Content of the \texttt{<attribute>} memory has been changed at address \texttt{<location>}.
                    \item[\texttt{>{}>{}> (<attribute>) mem\_inf\_wr\_val\_written @ <location>}]~\\
                        Value written to \texttt{<attribute>} memory at address \texttt{<location>}.
                    \item[\texttt{>{}>{}> (<attribute>) mem\_inf\_rd\_val\_read @ <location>}]~\\
                        Value read from \texttt{<attribute>} memory at address \texttt{<location>}.
                    \item[\texttt{>{}>{}> (<attribute>) mem\_err\_rd\_nonexistent @ <location>}]~\\
                        Attempt to read from nonexistent memory (i.e. memory with zero size), memory space: \texttt{<attribute>}, address: \texttt{<location>}.
                    \item[\texttt{>{}>{}> (<attribute>) mem\_err\_wr\_nonexistent @ <location>}]~\\
                        Attempt to write to nonexistent memory (i.e. memory with zero size), memory space: \texttt{<attribute>}, address: \texttt{<location>}.
                    \item[\texttt{>{}>{}> (<attribute>) mem\_wrn\_rd\_undefined @ <location>}]~\\
                        Attempt to read from uninitialized location in \texttt{<attribute>} memory at address \texttt{<location>}.
                    \item[\texttt{>{}>{}> (<attribute>) mem\_rd\_addr\_overflow @ <location> : <detail>}]~\\
                        Address overflow occurred during read, former address: \texttt{<location>}, new address: \texttt{<detail>}.
                    \item[\texttt{>{}>{}> (<attribute>) mem\_wr\_addr\_overflow @ <location> : <detail>}]~\\
                        Address overflow occurred during write, former address: \texttt{<location>}, new address: \texttt{<detail>}.
%                     \item[\texttt{>{}>{}> (<attribute>) mem\_err\_rd\_not\_implemented @ <location>}]~\\
%                     \item[\texttt{>{}>{}> (<attribute>) mem\_err\_wr\_not\_implemented @ <location>}]~\\
%                     \item[\texttt{>{}>{}> (<attribute>) mem\_err\_rd\_access\_denied @ <location>}]~\\
%                     \item[\texttt{>{}>{}> (<attribute>) mem\_err\_wr\_access\_denied @ <location>}]~\\
%                     \item[\texttt{>{}>{}> (<attribute>) mem\_wrn\_rd\_default @ <location>}]~\\
%                     \item[\texttt{>{}>{}> (<attribute>) mem\_wrn\_rd\_write\_only @ <location>}]~\\
%                     \item[\texttt{>{}>{}> (<attribute>) mem\_wrn\_wr\_read\_only @ <location>}]~\\
%                     \item[\texttt{>{}>{}> (<attribute>) mem\_wrn\_rd\_par\_write\_only @ <location>}]~\\
%                     \item[\texttt{>{}>{}> (<attribute>) mem\_wrn\_wr\_par\_read\_only @ <location>}]~\\
%                     \item[\texttt{>{}>{}> (<attribute>) mem\_wrn\_rd\_reserved\_read @ <location>}]~\\
%                     \item[\texttt{>{}>{}> (<attribute>) mem\_wrn\_wr\_reserved\_written @ <location>}]~\\
%                     \item[\texttt{>{}>{}> (<attribute>) mem\_sys\_fatal\_error}]~\\
                \end{description}

            \item[I/O group]~\\
                Input and output related events.
                \begin{description}
%                     \item[\texttt{>{}>{}> io\_write @ <location> : <detail>}]~\\
%                     \item[\texttt{>{}>{}> io\_indeterminable\_log @ <location>}]~\\
                    \item[\texttt{>{}>{}> plio\_read @ <location>}]~\\
                        Commencing read cycle at port address \texttt{<location>}.
                    \item[\texttt{>{}>{}> plio\_read\_end}]~\\
                        Finishing the read cycle.
                    \item[\texttt{>{}>{}> plio\_write @ <location> : <detail>}]~\\
                        Commencing write cycle at port address \texttt{<location>} with value \texttt{<detail>}.
                    \item[\texttt{>{}>{}> plio\_outputk @ <location> : <detail>}]~\\
                        Commencing \texttt{OUTPUTK} type write cycle at port address \texttt{<location>} with value \texttt{<detail>}.
                    \item[\texttt{>{}>{}> plio\_write\_end}]~\\
                        Finishing the write cycle.
                \end{description}
        \end{description}

    \subsection{Notes}
        \paragraph{Safety}~\\
            The simulator never crashes no matter what the input might be. Error states are always responded with appropriate error message or are just silently ignored.

        \paragraph{Usability}~\\
            This simulator interface is not meant for the user to hand write commands and directly read results, it is intended to be used in scripts.

        \paragraph{Complexity}~\\
            This tool is generally relatively modest in terms of memory usage and runs very fast in run mode (command ``\texttt{sim~run}''). The only thing which might heavily slow it down is the textual input and output; if you plan to use this tool in your scripts, please try to use the run mode as much as possible to preserve some speed advantage.
