Multitarget Development System (MDS) for PicoBlaze is a graphical integrated development environment (IDE) for Xilinx's PicoBlaze soft-core processors and their compatible clones. MDS is intended to be used mainly by developers and by education institutions, for students and hobbyists there is also noncommercial version available under a different license.

MDS provides all the necessary functionality to develop software part of a PicoBlaze application, including source code editor, assembler, disassembler, and simulator. Besides that there is also a number of tools and functions to make your work easier, the sole purpose of MDS is to save your time and enable development of more complex applications. User is our main concern. We believe you will feel relatively comfortable while working with this tool.

\section{Main features}
    \begin{itemize}
        \item Text editor optimized for writing source code.
        \item Project manager for creating and maintaining your projects.
        \item Very fast simulator for all available versions of Picoblaze.
        \item Macro-assembler supporting wide range of output file formats, including MEM, HEX, and VHDL.
        \item PicoBlaze disassembler capable of reading from HEX, VHDL, MEM, and other file formats.
        \item Tool called Assembler Translator for compatibility with your current tools.
        \item Command line tools, and a number of graphical utilities.
    \end{itemize}

\clearpage
\section{Requirements}
    This software is compiled for x86 processors in both 32b and 64b variants with SSE instruction set, that means you
    need a processor equipped with SEE feature (most processors manufactured after the year 2003 are). For GNU/Linux,
    MDS also requires GLIBC version >= 2.14.

    Supported host operating systems are:
    \begin{itemize}
        \item Microsoft Windows 7 (32b and 64b),
        \item Microsoft Windows XP (32b),
        \item Ubuntu 13.10 and higher (32b and 64b),
        \item CentOS 7 (64b),
        \item openSUSE 13.1 and higher (32b and 64b),
        \item Fedora 20 and higher (32b and 64b).
    \end{itemize}
